\documentclass[a4paper,12pt]{scrartcl}
\usepackage[utf8x]{inputenc}
\usepackage[T1]{fontenc} % avec T1 comme option  d'encodage c'est ben mieux, surtout pour taper du français.
%\usepackage{lmodern,textcomp} % fortement conseillé pour les pdf. On peut mettre autre chose : kpfonts, fourier,...
\usepackage[french]{babel} %Sans ça les guillemets, amarchpo
\usepackage{amsmath}
\usepackage{multicol}
\usepackage{amssymb}
\usepackage{tkz-tab}
\usepackage{exercice_sheet}

%\trait
%\section*{}
%\exo{}
%\question{}
%\subquestion{}

\date{}


% Title Page
\title{Formulaire, polynômes du second degré à une inconnue}

\author{\textsc{Mathématiques}}

\begin{document}

\maketitle




On considère une fonction polynomiale du second degré $P$ telle que $P(x) = ax^2 + bx + c$ où $a$, $b$ et $c$ sont des nombres réels. 

\section*{Calcul du discriminant}

$\Delta = b^2 - 4ac$

\section*{Solutions de l'équation $P(x) = 0$}

\begin{itemize}
\item $\Delta > 0$

L'équation a 2 solutions : 

$\mathcal{S} = \left\lbrace \dfrac{-b-\sqrt{\Delta}}{2a} ; \dfrac{-b+\sqrt{\Delta}}{2a} \right\rbrace$

\item $\Delta = 0$

L'équation a une unique solution :

$\mathcal{S} = \left\lbrace \dfrac{-b}{2a}\right\rbrace$

\item $\Delta < 0$

L'équation n'a pas de solution :

$\mathcal{S} = \emptyset$

\end{itemize}

\section*{Factorisation de $P(x)$}

\begin{itemize}
\item $\Delta > 0$

L'équation $P(x) = 0$ a 2 solutions que l'on appelle $x_1$ et $x_2$: 

On peut donc écrire la forme factorisée de $P(x)$: $P(x) = a(x - x_1)(x - x_2)$

\item $\Delta = 0$

L'équation $P(x) = 0$ a une unique solution que l'on appelle $x_1$:

On peut donc écrire la forme factorisée de $P(x)$: $P(x) = a(x - x_1)^2$

\item $\Delta < 0$

L'équation $P(x) = 0$ n'a pas de solution, on ne peut donc pas factoriser $P(x)$.

\end{itemize}


\trait

\begin{center}
Fin.
\end{center}

\end{document}
