\section*{Exponentielle et logarithme}

$a$ et $b$ sont des nombres réels.

\begin{center}

\begin{tabular}{|c|c|}
\hline 
\textbf{fonction exponentielle} & \textbf{fonction logarithme népérien} \\ 
\hline 
$e^{a+b} = e^{a} \times e^{b}$ & $\ln(a \times b) = \ln a + \ln b$ \\ 
\hline 
$e^{a-b} = \dfrac{e^{a}}{e^{b}}$ & $\ln \left( \dfrac{a}{b} \right) = \ln a - \ln b$ \\ 
\hline 
$e^{-a} = \dfrac{1}{e^{a}}$ & $\ln \left( \dfrac{1}{a} \right) = - \ln a$ \\ 
\hline 
$(e^{a})^{b} = e^{a \times b}$ & $\ln(a^n) = n \ln a$, $n \in \mathbb{R}$, $a > 0$ \\ 
\hline 
$e^{1} = e$ & $\ln e = 1$ \\ 
\hline 
$e^{0} = 1$ & $\ln 1 = 0$ \\ 
\hline 
\end{tabular} 

\end{center}

\subsection*{Quelques propriétés}

\begin{itemize}

\item les 2 fonctions sont réciproques: $\ln \left( e^{a} \right) = a$ pour $a \in \mathbb{R}$; $e^{\ln a} = a$ pour $a > 0$ (\og $a$ strictement positif \fg{}).

\item ensemble de définition: 

$e^{a}$ peut être calculé quel que soit $a \in \mathbb{R}$: on dit que la fonction exponentielle est définie sur $\mathbb{R}$.

$\ln a$ peut être calculé uniquement pour $a > 0$. On dit donc que la fonction logarithme népérien est définie sur $]0;+\infty[$.


\item elles sont strictement croissantes sur leurs ensembles de définition, c'est-à-dire que:

$a > b > 0 \Leftrightarrow \ln a > \ln b$

et pour la fonction exponentielle:

$a > b \Leftrightarrow e^{a} > e^{b}$

\end{itemize}
