\documentclass[a4paper,12pt]{scrartcl}
\usepackage[utf8x]{inputenc}
\usepackage[T1]{fontenc} % avec T1 comme option  d'encodage c'est ben mieux, surtout pour taper du français.
%\usepackage{lmodern,textcomp} % fortement conseillé pour les pdf. On peut mettre autre chose : kpfonts, fourier,...
\usepackage[french]{babel} %Sans ça les guillemets, amarchpo
\usepackage{amsmath}
\usepackage{multicol}
\usepackage{amssymb}
\usepackage{tkz-tab}
\usepackage{exercice_sheet}

%\trait
%\section*{}
%\exo{}
%\question{}
%\subquestion{}

\date{}


% Title Page
\title{Devoir maison, équations du second degré, CG-1}

\author{\textsc{Mathématiques}}

\begin{document}

\maketitle

%\tableofcontents

\fiche{}

\exo{}

\question{}
\begin{multicols}{2}
\begin{equation*}
f'(x) = 2 
\end{equation*} 

\begin{equation*}
f'(x) = \frac{1}{4}
\end{equation*} 
\begin{equation*}
f'(x) = \frac{-4}{x^2}
\end{equation*}
 
\begin{equation*}
f'(x) = \sqrt{2}
\end{equation*}
 
\begin{equation*}
f'(x) = 0
\end{equation*}
 
\begin{equation*}
f'(x) = \frac{3}{4x^2}
\end{equation*}
\end{multicols}

\question{}

\begin{multicols}{2}
\begin{equation*}
g'(x) = 2x + \frac{1}{3}
\end{equation*}
 
\begin{equation*}
g'(x) = \frac{5}{2}x^2-2
\end{equation*}
 
\begin{equation*}
g'(x) = 6x - \frac{2}{x^2}
\end{equation*}
 
\begin{equation*}
g'(x) = 3x^2 + 1 + \frac{3}{x^2}
\end{equation*}
\end{multicols}


\question{}

\begin{multicols}{2}
\begin{equation*}
h'(x) = \frac{2}{\sqrt{x}} + 6x
\end{equation*}
 
\begin{equation*}
h'(x) = 6x - \frac{4}{x^2} - \frac{5}{2 \sqrt{x}}
\end{equation*}
 
\begin{equation*}
h'(x) = 2x
\end{equation*}
 
\begin{equation*}
h'(x) = \frac{1}{x^2}
\end{equation*}
\end{multicols}


\exo{}
\begin{tikzpicture}
\def\xmi{-1.5}%xmi et xma : intervalle de définition
\def\xma{4}
\pgfmathsetmacro{\lrmargin}{(\xma-\xmi)/10}%de combien la grille dépasse ?
\pgfmathsetmacro{\xlborder}{\xmi-\lrmargin}
\pgfmathsetmacro{\xrborder}{\xma+\lrmargin}

\tkzInit[xmin=\xlborder,xmax=\xrborder,ymin=-3,ymax=6]
\tkzGrid[sub,color=gray, subxstep=.5,subystep=.5]
\tkzAxeXY[very thick]
\tkzGrid

\draw [domain=\xmi:\xma, very thick, color=black!60!green, samples=400] plot(\x,{(\x)^2 - 3*\x }) node[left] {$\mathcal{C}_f$};
\draw [domain=-1.5:0,  thick] plot(\x,{-5*(\x)-1}) ;
\draw [domain=-1:1, thick] plot(\x,{-3*(\x)}) ;
\draw [domain=0:3, thick] plot(\x,{-2.25}) ;
\draw [domain=2.2:3.7, thick] plot(\x,{3*(\x)-9}) ;
\end{tikzpicture}



\begin{tikzpicture}
\def\xmi{-2.2}%xmi et xma : intervalle de définition
\def\xma{2.2}
\pgfmathsetmacro{\lrmargin}{(\xma-\xmi)/10}%de combien la grille dépasse ?
\pgfmathsetmacro{\xlborder}{\xmi-\lrmargin}
\pgfmathsetmacro{\xrborder}{\xma+\lrmargin}

\tkzInit[xmin=\xlborder,xmax=\xrborder,ymin=-3,ymax=6]
\tkzGrid[sub,color=gray, subxstep=.5,subystep=.5]
\tkzAxeXY[very thick]
\tkzGrid

\draw [domain=\xmi:\xma, very thick, color=black!60!green, samples=400] plot(\x,{(\x)^3 - 3*\x + 3}) node[left] {$\mathcal{C}_f$};
\draw [domain=-2.2:-1.5,  thick] plot(\x,{8*(\x)+17}) ;
\draw [domain=\xmi:\xma,  thick] plot(\x,{5}) ;
\draw [domain=-1:1, thick] plot(\x,{-3*(\x)+3}) ;
\draw [domain=\xmi:\xma, thick] plot(\x,{1}) ;
\draw [domain=1.5:2.1, thick] plot(\x,{8*(\x)-11}) ;
\end{tikzpicture}

\exo{}

\question{}
$f'(x) = -\frac{4}{x^2}$

\question{}
$M\left( x_M ; f(x_M) \right)$ soit $M(1 ; 3)$

$N(2 ; 1)$


\question{}
$f'(1) = -4$, $f'(2) = -1$

\question{}

\begin{tikzpicture}
\def\xmi{0.5}%xmi et xma : intervalle de définition
\def\xma{5}
\pgfmathsetmacro{\lrmargin}{(\xma-\xmi)/10}%de combien la grille dépasse ?
\pgfmathsetmacro{\xlborder}{\xmi-\lrmargin}
\pgfmathsetmacro{\xrborder}{\xma+\lrmargin}

\tkzInit[xmin=\xlborder,xmax=\xrborder,ymin=0,ymax=7]
\tkzGrid[sub,color=gray, subxstep=.5,subystep=.5]
\tkzAxeXY[very thick]
\tkzGrid

\draw [domain=\xmi:\xma, very thick, color=black!60!green, samples=400] plot(\x,{4/(\x)-1}) node[left] {$\mathcal{C}_f$};
\draw [domain=0.5:2,  thick] plot(\x,{-4*(\x)+7}) node[left] {$\mathcal{T}_1$};
\draw [domain=0.5:4,  thick] plot(\x,{-1*(\x)+3}) node[left] {$\mathcal{T}_2$};
\end{tikzpicture}


\exo{}


\exo{}

\fiche{}


\fiche{}


\fiche{}


\fiche{}




\end{document}

