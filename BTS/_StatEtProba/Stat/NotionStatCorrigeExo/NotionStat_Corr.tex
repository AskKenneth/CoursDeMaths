\documentclass[a4paper,12pt]{scrartcl}
\usepackage[utf8x]{inputenc}
\usepackage[T1]{fontenc} % avec T1 comme option  d'encodage c'est ben mieux, surtout pour taper du français.
%\usepackage{lmodern,textcomp} % fortement conseillé pour les pdf. On peut mettre autre chose : kpfonts, fourier,...
\usepackage[french]{babel} %Sans ça les guillemets, amarchpo
\usepackage{amsmath}
\usepackage{multicol}
\usepackage{amssymb}
\usepackage{tkz-tab}
\usepackage{exercice_sheet}

%\trait
%\section*{}
%\exo{}
%\question{}
%\subquestion{}

\date{}


% Title Page
\title{Cours \og notion de statistiques \fg{}, corrigé des exercices}

\author{Mathématiques}

\begin{document}

\maketitle

\exo{Relevé journalier}

\question{Compléter le tableau}

On rappelle que la fréquence d'une classe se calcule en divisant l'effectif de cette classe par l'effectif total qui vaut $80 + 320 + 60 + 40 = 500$:

$$f_i = \dfrac{n_i}{N}$$

\begin{center}
\begin{tabular}{@{}|l|l|l|@{}}
\toprule
\textbf{Classe} & \textbf{Effectifs ni} & \textbf{Fréquence fi} \\ \midrule
\textbf{{[}10;20{[}} & 80 & 0,16 \\ \midrule
\textbf{{[}20;30{[}} & 320 & 0,64 \\ \midrule
\textbf{{[}30;40{[}} & 60 & 0,12 \\ \midrule
\textbf{{[}40;50{[}} & 40 & 0,08 \\ \bottomrule
\end{tabular}
\end{center}

\question{Moyenne}

On rappelle la formule de la moyenne:

$$\overline{x} = \dfrac{1}{N} \cdot \sum_{i=1}^{p} n_i \cdot x_i $$.

Ici, en l'absence d'information supplémentaire, on prend pour $x_i$ le milieu de chaque classe: 15, 25, 35 et 45. On a donc ici: 

$$\overline{x} = \dfrac{15 \times 80 + 25 \times 320 + 35 \times 64 + 45 \times 40}{500} = 26.20 \mbox{€}$$ 

\question{Variance, écart-type}

\littlestar{Variance}

On rappelle la formule de la variance de la variable $x$:

$$V_x = \dfrac{1}{N} \sum_{i=1}^{p} n_i \cdot x_i^2 - \overline{x}^2$$

On remarque que cette notation est équivalente à:

$$V_x = \overline{x^2} - \overline{x}^2$$.

Appliqué à cet exemple:

$$V_x = \dfrac{15^2 \times 80 + 25^2 \times 320 + 35^2 \times 60 + 45^2 \times 40}{500} - 26.2^2 = 58.56$$.

Attention, il faut être très vigilant aux erreur d'arrondi, il est très important d'utiliser exclusivement la mémoire de la calculatrice (touche \textsc{Ans}) qui elle, retient 8 à 10 décimales et évite donc les erreurs.

\littlestar{Écart-type}

$\sigma_x = \sqrt{V_x} = 7.65 \mbox{€}$.

\question{Pourcentage de clients dont le montant du retrait est compris dans l'intervalle $[\overline{x}-\sigma_x ; \overline{x}+\sigma_x]$:}

$\overline{x}-\sigma_x = 26.2 - 7.65 = 18.55$

$\overline{x}+\sigma_x = 26.2 + 7.65 = 33.85$

On cherche donc combien de personnes sont dans l'intervalle $[18.55;33.85]$. 

La classe $[20;30[$ est totalement incluse dans cet intervalle. Il faut donc compter son effectif total, 320.

Concernant la classe $[10;20[$, seules les retraits compris entre 18.55 et 20 sont inclus. On ne sait pas comment sont répartis les retraits au sein des classes. On considère donc qu'ils sont uniformément répartis. 

L'intervalle $[18.55;20[$ a une largeur de $20 - 18.55 = 1.45$ et représente donc 14.5\% de l'intervalle. On considère donc qu'elle contiendra 14.5\% de l'effectif.

$80 \times \dfrac{14.5}{100} = 11.6$ que l'on arrondit à 12. 

De la même manière, l'intervalle $[30;33.85[$ représente 38.5\% de la classe $[30;40[$. 

$60 \times \dfrac{38.5}{100} = 23.1$ que l'on arrondit à 23.

L'intervalle $[\overline{x}-\sigma_x ; \overline{x}+\sigma_x]$ contient donc un effectif de $320 + 12 + 23 = 355$.

On calcule le pourcentage de l'effectif total que cela représente: $\dfrac{355}{500} = 0.71 = 71\%$. 

\exo{Temps d'attente}

On commence par compléter le tableau:

\begin{center}
\begin{tabular}{@{}|l|l|l|@{}}
\toprule
\textbf{Classe} & \textbf{Effectifs $n_i$} & \textbf{Fréquence $f_i$} \\ \midrule
\textbf{{[}0 ; 5{[}} & 3 & 0,025 \\ \midrule
\textbf{{[}5 ; 10{[}} & 12 & 0,1 \\ \midrule
\textbf{{[}10 ; 15{[}} & 54 & 0,45 \\ \midrule
\textbf{{[}15 ; 20{[}} & 6 & 0,05 \\ \midrule
\textbf{{[}20 ; 25{[}} & 45 & 0,375 \\ \midrule
\textbf{Total} & \textbf{N =} & \textbf{120} \\ \bottomrule
\end{tabular}
\end{center}

\question{}
La moyenne peut également se calculer à partir des fréquences: 

$$\overline{x} = \sum_{i=1}^{p} f_i \cdot x_i$$

De la même façon qu'à l'exercice précédent, on prend le milieu de chaque classe pour les valeurs de $x_i$.

On a donc ici:

$$\overline{x} = 2.5 \times 0.025 + 7.5 \times 0.1 + \ldots + 22.5 \times 0.375 = 15.75 \mbox{s}$$

\question{Variance, écart-type}

\littlestar{Variance}
de la même manière qu'à l'exercice précédent:

$$V_x = \sum_{i=1}^{p} f_i \cdot x_i^2 - \overline{x}^2$$

Appliqué ici: 

$$V_x = 2.5^2 \times 0.025 + \ldots + 22.5 \times 0.375 = 33.19$$

\littlestar{Écart-type}

$\sigma_x = \sqrt{V_x} = 5.76 \mbox{s}$.

\question{}
$\overline{x} = 15.75$ et $\sigma_x = 5.76$.

On s'intéresse donc à l'intervalle $[10;21.5]$.

On inclut donc entièrement les classes $[10;15[$ et $[15;20[$, ainsi que $1.5/5 = 30\%$ de l'intervalle $[20;25[$.

$45 \times \dfrac{30}{100} = 13.5$, que l'on arrondit à 14. 

L'intervalle $[10;21.5]$ contient donc $54+6+14 = 74$.

$\dfrac{74}{120} = 0.617 \approx 62\%$.

\trait

\end{document}
