\documentclass[a4paper,12pt]{scrartcl}
\usepackage[utf8x]{inputenc}
\usepackage[T1]{fontenc} % avec T1 comme option  d'encodage c'est ben mieux, surtout pour taper du français.
%\usepackage{lmodern,textcomp} % fortement conseillé pour les pdf. On peut mettre autre chose : kpfonts, fourier,...
\usepackage[french]{babel} %Sans ça les guillemets, amarchpo
\usepackage{amsmath}
\usepackage{multicol}
\usepackage{amssymb}
\usepackage{tkz-tab}
\usepackage{exercice_sheet}

%\trait
%\section*{}
%\exo{}
%\question{}
%\subquestion{}


% Title Page
\title{Le vocabulaire des statistiques}

\date{}

\begin{document}

\maketitle

\section*{Notions générales en statistiques}


\littlestar{Population}

Ensemble d'individus (ou unités statistiques) dont on observe plusieurs caractères. Le nombre d'individus est l'effectif $N$ de la population.

\littlestar{Échantillon}

Partie d'une population.

\littlestar{modalités}

Valeurs prises par le caractère pour un individu de la population ou de l'échantillon.

\littlestar{Caractère qualitatif}

Caractère dans les modalités sont seulement repérables et non mesurable (couleur, forme, marque, etc.).

\littlestar{Caractère quantitatif}

Caractère dans les modalités sont mesurables, les mesures étant les valeurs d'une variable statistique (poids, âge, taille, longueur, diamètre, etc.).

\littlestar{Variable discrète ou discontinue}

Variable qui ne peut prendre qu'un nombre limité de valeurs (entières par exemple). À chaque valeur correspond un effectif $n_i$, la série est dite pondérée.

\littlestar{Variable continue}

Variable qui peut prendre n'importe quelle valeur d'un intervalle.

\littlestar{Classe}

Lorsque la variable est continue, ses valeurs sont regroupées dans des intervalles de la forme $[a;b[$ que l'on appelle classes. Le nombre de valeurs dans une classe est l'effectif $n_i$ de cette classe.

\begin{itemize}
\item on appelle amplitude de la classe la différence $b-a$ (la largeur de l'intervalle)
\item on appelle centre de la classe le nombre $\frac{a+b}{2}$ (moyenne arithmétique des bornes de l'intervalle autrement dit, le milieu de l'intervalle)
\end{itemize}

\littlestar{Fréquence $f_i$ d'une valeur ou d'une classe}

C'est le quotient de l'effectif $n_i$ de cette valeur (ou classe) par l'effectif total $N$: 

$$f_i = \dfrac{n_i}{N}$$

C'est un nombre toujours compris entre 0 et 1. Autrement dit: $f_i \in [0;1]$.

\section*{Indicateurs de tendance centrale}

\littlestar{Moyenne, notée $\overline{x}$}

La moyenne représente la somme des valeurs de la série divisée par le nombre de valeurs (l'effectif total).

$$\overline{x} = \dfrac{\mbox{Somme des valeurs}}{\mbox{Nombre de valeurs}} = \dfrac{\mbox{Somme des valeurs}}{\mbox{Effectif total}}$$

\littlestar{Médiane, notée $m$}

C'est la valeur telle que 50\% des valeurs soient inférieures ou égales. 

On l'appelle aussi $Q_2$ (pour comprendre pourquoi on l'appelle ainsi, voir point sur les quartiles).

\littlestar{Le mode}

C'est la valeur d'une série statistique qui a la valeur la plus élevée.

\section*{Indicateurs de dispersion}

\littlestar{Les quartiles}

\begin{itemize}
\item $Q_1$: c'est la valeur de la donnée de la série qui sépare les 25\% inférieurs des données;
\item $Q_3$: c'est la valeur de la donnée de la série qui sépare les 75\% inférieurs des données;
\end{itemize}

\littlestar{Écart interquartile}

C'est la différence $Q_3 - Q_1$ et contient donc 50\% des valeurs.

\littlestar{Étendue}

C'est la différence entre le maximum et le minimum de la série: $x_{max} - x_{min}$.

\littlestar{Écart-type $\sigma$}

Il mesure la dispersion des valeurs autour de la moyenne.

Si $\sigma$ est grand, les valeurs de la série sont dispersées loin de la moyenne.

Si $\sigma$ est petit, les valeurs de la série sont regroupées autour de la moyenne.

\end{document}
