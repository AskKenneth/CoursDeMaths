\documentclass[a4paper,12pt]{scrartcl}
\usepackage[utf8x]{inputenc}
\usepackage[T1]{fontenc} % avec T1 comme option  d'encodage c'est ben mieux, surtout pour taper du français.
%\usepackage{lmodern,textcomp} % fortement conseillé pour les pdf. On peut mettre autre chose : kpfonts, fourier,...
\usepackage[french]{babel} %Sans ça les guillemets, amarchpo
\usepackage{amsmath}
\usepackage{multicol}
\usepackage{amssymb}
\usepackage{tkz-tab}
\usepackage{exercice_sheet}

%\trait
%\section*{}
%\exo{}
%\question{}
%\subquestion{}

\date{}


% Title Page
\title{Corrigé des feuilles d'exercices \og Fiche d'exercices statistiques \fg{}}

\author{Mathématiques}
 
\begin{document}

\maketitle

\exo{}

Série de notes: 6 ; 8 ; 8 ; 9 ; 9 ; 10 ; 11 ; 12 ; 14 ; 17 ; 18 ;18 ; 19.

\question{Calculer la moyenne arrondie au centième}

$$\bar{x} = \dfrac{6 + 8 + 8 + 9 + 9 + 10 + 11 + 12 + 14 + 17 + 18 +18 + 19}{13} = 12.2$$

\question{Déterminer la médiane}

Les notes sont déjà rangées dans l'ordre croissant. Il y en a 13, la médiane est donc la 7ème valeur. La médiane est donc 11. 

\question{Déterminer le premier quartile}

$13 \times 0.25 = 3.25$, on regarde donc la 4ème valeur (on arrondit toujours au-dessus):

$Q_1 = 9$

\question{Déterminer le troisième quartile}

$$13 \times 0.75 = 9.75$$, on regarde donc la 10ème valeur:

$Q_3 = 18$

\exo{}

On trie la série statistique: 1 ; 8 ; 9 ; 12 ; 13 ; 14 ; 15 ; 17 ; 17 ; 20 ; 23 ; 25.

L'effectif est 12.

\question{La médiane}

On fait la moyenne entre la 6ème et la 7ème valeur de la série triée: $Q_2 = \frac{14+15}{2} = 14.5$.

\question{Le premier quartile}

$$12 \times 0.25 = 3$$, on regarde donc la 3ème valeur (on arrondit toujours au-dessus):

$Q_1 = 9$

\question{Le troisième quartile}

$$12 \times 0.75 = 9$$, on regarde donc la 9ème valeur:

$Q_3 = 17$.

\exo{On rappelle la série statistique:}

\begin{tabular}{|l|l|l|l|l|l|l|}
\hline
\textbf{Note}     & 6                      & 8                      & 10                     & 13                     & 14 & 17 \\ \hline
\textbf{Effectif} & \multicolumn{1}{c|}{3} & \multicolumn{1}{c|}{5} & \multicolumn{1}{c|}{6} & \multicolumn{1}{c|}{7} & 5  & 1  \\ \hline
\end{tabular}

\question{}
L'effectif de la classe est de 27 élèves (énoncé).

La moyenne de classe vaut: 

$$\dfrac{6 \times 3 + 8 \times 5 + 10 \times 6 + 13 \times 7 + 14 \times 5 + 17 \times 1}{27} = 10.96$$

Finalement, $\bar{x} \approx 11$.

\question{}
Le nombre d'élève ayant eu une note supérieure ou égale à 10 est $6+7+5+1 = 19$. $\dfrac{19}{27} = 0.704 = 70.4\%$.

\question{}
Sur un effectif de 27, la médiane est la 14ème valeur de la série rangée dans l'ordre croissant. Ici, il s'agit de 10.

$Q_2 = 10$.

\exo{}

\question{}
On remarque que si $y = 0$, 8 est médiane pour cette série.

\question{}
Si $y=4$, on peut exprimer la moyenne comme suit:

$$\bar{x} = \dfrac{7 \times 1 + 7.5 \times 2 + 8 \times 4 + 8.5 \times 3 + 9 \times 1 + x \times 4}{15}$$

Il faut donc résoudre l'équation suivante:

$$8 = \dfrac{7 \times 1 + 7.5 \times 2 + 8 \times 4 + 8.5 \times 3 + 9 \times 1 + x \times 4}{15}$$

$$\Leftrightarrow 120 = 7 \times 1 + 7.5 \times 2 + 8 \times 4 + 8.5 \times 3 + 9 \times 1 + x \times 4$$

$\Leftrightarrow 120 = 88.5 + 4x$

Soit $31.5 = 4x \Leftrightarrow x = 7.875$.

\exo{Classe}

\question{Effectif}
$N = 23$

\question{Moyenne}
$\overline{x} = 10.9$

\question{Moins de 8}

9 élèves ont obtenu 8 ou moins: 

$\frac{9}{23} \approx 0.391 \approx 39\%$

\question{Médiane}

L'effectif $N$ est de 23. Classés dans l'ordre croissant, la médiane correspond à la 12ème valeur. Cela correspond à 11: $Q_2 = 11$.

\exo{Bâtons}

\question{}
Le diagramme donné peut être traduit sous forme d'un tableau similaire à celui donné dans l'exercice précédent:

\begin{center}
\begin{tabular}{|l|c|c|c|c|c|c|c|c|c|c|}\hline
\textbf{Note sur 20} & 8 & 9 & 10 & 11 & 12 & 13 & 14 & 15 & 16 & 17\\\hline
\textbf{Effectif} & 2 & 3 & 1 & 3 & 5 & 4 & 1 & 3 & 2 & 1\\\hline
\end{tabular}
\end{center}

La moyenne des notes est de 12.24.

\question{}
Avec un effectif de 25, il faut regarder le 13ème. Rangés dans l'ordre croissant, le 13ème a 12. On a donc $Q_2 = 12$.

\question{}
Dans cette série, avoir une note strictement supérieure à 13, c'est avoir entre 14 et 17. Cela concerne 7 élèves.

$\frac{7}{25} = 0.28 = 28\%$.

\question{Premier quartile}

$\frac{25}{4} = 6.25$, $Q_1$ est donc le 7ème élément de la série triée dans l'ordre croissant. 

$Q_1 = 11$.

\question{Troisième quartile}

$25 \times \frac{3}{4} = 18.75$, $Q_3$ est donc le 19ème élément de la série triée dans l'ordre croissant. 

$Q_3 = 14$.

\exo{}

\question{}On rappelle que l'étendue est la modalité la plus basse ôtée de la modalité la plus haute.

Ici, $15-9 = 6$.

\question{}

$\overline{x} \approx 10.1$

\question{}
L'effectif est de 13. On regarde donc la 7ème note de la série triée dans l'ordre croissant:

$Q_2 = 9$

\question{Premier quartile}

$\frac{13}{4} = 3.25$, $Q_1$ est donc le 4ème élément de la série triée dans l'ordre croissant. 

$Q_1 = 8$.

\question{Troisième quartile}

$13 \times \frac{3}{4} = 9.75$, $Q_3$ est donc le 10ème élément de la série triée dans l'ordre croissant. 

$Q_3 = 12$.

\exo{}

\question{}
Le maximum de cette série est 54.8s et le minimum est 48.65s. L'étendue est donc $54.8-48.65=6.15$s.

\question{}
La moyenne est 51.34s.

\question{}
L'effectif est $N = 15$. $Q_2$ est donc la modalité de rang $\frac{N+1}{2}$ de la série classée dans l'ordre croissant, soit la huitième valeur.

$Q_2 = 51.8$s.

\question{}
12 coureurs sont dans ce cas. Cela représente $\frac{12}{15} = 0.8 = 80\%$.

\exo{}

\question{}
La taille moyenne est de 174.6cm.

\question{}
Pour calculer la médiane, il faut d'abord trier la série dans l'ordre croissant:

165, 165, 166, 170, 171, 174, 174, 175, 176, 176, 177, 178, 181, 184, 187.

L'effectif est $N = 15$. $Q_2$ est donc la modalité de rang $\frac{N+1}{2}$ de la série classée dans l'ordre croissant, soit la huitième valeur.

$Q_2 = 175$cm.

\question{}
L'étendue est $187-165 = 22$cm.


\exo{}

\question{}
Pour ce faire, il suffit de diviser par 10: 

\begin{center}
\begin{tabular}{|l|l|l|l|l|l|}\hline
0,67 & 0,78 & 0,82 & 1,01 & 0,93 & 0,69\\\hline
0,77 & 0,68 & 0,85 & 0,9 & 1,02 & 1,1\\\hline
\end{tabular}
\end{center}

\question{Médiane}

classée dans l'ordre croissant, la série devient:

6.7, 6.8, 6.9, 7.7, 7.8, 8.2, 8.5, 9, 9.3, 10.1, 10.2, 11.

Son effectif est $N = 12$. On doit donc calculer la moyenne entre la modalité de rang $\frac{N}{2}$ et celle de rang $\frac{N}{2}+1$: $\frac{8.2+8.5}{2} = 8.35$l/100km.

\question{Quartiles}

$12 \times \frac{1}{4} = 3$. Il faut donc regarder le 3ème terme: $Q_1 = 6.9$l/100km.

\question{Quartiles}

$12 \times \frac{3}{4} = 9$. Il faut donc regarder le 9ème terme: $Q_3 = 9.3$l/100km.

\question{Vrai-Faux}

C'est faux, vu que l'amplitude de l'intervalle $\left[ 7 ; 9 \right]$ est strictement inclus dans l'intervalle $\left[ Q_1, Q_3 \right]$\footnote{Sachant que l'intervalle $\left[ Q_1, Q_3 \right]$ contient \emph{par définition} 50\% de l'effectif...}.


\exo{Températures}

\question{Moyennes}

Il s'agit de calculer la moyenne de températures pour al ville A d'une part, et pour la ville B d'autre part.

\begin{itemize}
 \item Ville A: 12.5 degrés;
 \item Ville B: 12.5 degrés.
\end{itemize}

\question{Médianes}

\begin{itemize}
 \item Ville A: $Q_2 = 11$ degrés;
 \item Ville B: $Q_2 = 12.5$ degrés.
\end{itemize}

\question{Étendues}

\begin{itemize}
 \item Ville A: l'étendue est de 18 degrés;
 \item Ville B: l'étendue est de 4 degrés.
\end{itemize}

\question{Quartiles}

\begin{itemize}
 \item Ville A: $Q_1 = 7$  degrés et $Q_3 = 16$ degrés;
 \item Ville B: $Q_1 = 12$ degrés et $Q_3 = 13$ degrés.
\end{itemize}

\question{Comparaison}

Ces deux villes ont les mêmes températures moyennes mais cela ne signifie pas qu'elles ont le même climat: la ville A a une plus forte amplitude thermique entre l'été et l'hiver que la ville B.


\exo{Vrai-Faux}

Commençons par classer la série dans l'ordre croissant:

2,5; 2,5; 4,4; 4,5; 5; 7,5; 7,5; 10,5; 10,5; 10,5; 12; 16.

\question{Étendue}

$16-2.5 = 13.5 \neq 16$. Faux.


\question{Moyenne}

On trouve 7.78, donc faux.


\question{Médiane}

L'effectif est $N=12$. $Q_2$ est donc la demi somme des modalités de rang $\frac{N}{2}$ et $\frac{N}{2}+1$, ici 6 et 7: $\frac{7.5+7.5}{2}=7.5$. 


\question{Premier quartile}

Avec un effectif de $N = 12$, $Q_1$ est le terme de rang 3, soit 4.4. Zut, encore raté!


\question{Troisième quartile}

$Q_3$ est le 9ème terme, soit 10.5. comme quoi quand on veut...


\exo{Poubelles}

\question{Moyenne}

La masse moyenne par mois est $\dfrac{40+25+\ldots +51}{12} = 30$kg.


\question{Médiane}

Classons la série dans l'ordre croissant:

15; 20; 24; 24; 25; 28; 30; 32; 35; 36; 40; 51.

Sur une effectif $N=12$, la médiane est la demi somme du 6ème et du 7ème terme.

$Q_2 = \frac{28+30}{2} = 29$kg.


\question{Quartiles}

$Q_1$ est le 3ème terme. $Q_1 = 24$kg.

$Q_3$ est le 9ème terme. $Q_3 = 35$kg.


\question{}

Cet intervalle est strictement plus grand que $\left[ Q_1, Q_3 \right]$.


\exo{Ampoules}

\question{Pourcentage en-dessous de 1400h}

L'effectif est $N = 6000$ ampoules.

Parmi ces 6000 ampoules, 2010 ont une durée de vie inférieure à 1400h. $\frac{2010}{6000} = \frac{67}{200} = 33.5\%$.

\question{Moyenne}

Dans cet exercice, on a pas des valeurs précises mais des classes ($[1000,1200]$, $[1200,1400]$, etc.). Dans ce cas et faute d'avoir plus d'information, on utilise les \emph{milieux} de chaque classe: pour la classe $[1000,1200]$ on prend 1100, pour $[1200,1400]$ on prend 1300, etc.

La moyenne est donc ici $\overline{x} = \dfrac{1100 \times 550 + 1300 \times 1460 + \ldots + 1900 \times 430}{6000} = 1498$h de durée de vie moyenne.


\exo{}

\partie{}

\question{}
D'après le graphe, on obtient:

\begin{center}
\begin{tabular}{|l|l|l|l|l|}
\hline
\textbf{Durée (min)} & 90 & 100 & 105 & 120\\
\hline
\textbf{Effectifs} & 2 & 6 & 4 & 3\\
\hline
\end{tabular}
\end{center}

\question{Paramètres}

\subquestion{Étendue}

$120-90 = 30$. L'étendue est de 30min.

\subquestion{Médiane}

L'effectif est $N = 15$. $Q_2$ est donc la 8ème valeur de la série dans l'ordre croissant. 

$Q_2 = 100$min.

\subquestion{}

$\overline{x} = \dfrac{90 \times 2 + \ldots + 120 \times 3}{15} = 104$min.

\partie{Laurent}

\question{}
$40$ minutes représentent $\frac{2}{3} \approx 0.667$h. Dans cette durée, il a parcouru 9kms. Sa vitesse moyenne sur cette partie est donc $\dfrac{9}{\frac{2}{3}} = 13.5$km/h.

\question{}
$50$ minutes représentent $\frac{5}{6} \approx 0.833$h. Dans cette durée, il a parcouru 12kms. Sa vitesse moyenne sur cette partie est donc $\dfrac{12}{\frac{5}{6}} = 14.4$km/h.

\question{}
Si on considère le parcours total, la durée est de 90 minutes soit 1.5h. Il a parcouru 21 kms. Sa vitesse moyenne est donc $\dfrac{21}{1.5} = 14$km/h. Belle moyenne!

\exo{Salaires}

La moyenne contrairement à la médiane est sensible aux valeurs extrêmes. On peut déduire de ces informations qu'il y a un petit nombre de personnes aux revenus très élevés en France.

\exo{Vrai-faux}

\question{}
7 étant le premier quartile, c'est 25\% et non $\frac{1}{3}$ que sont en-dessous de 7. \textbf{Faux}.

\question{}
C'est même la définition de la médiane. \textbf{Vrai}.

\question{}
Les quartiles découpant la population en 4 parties d'effectifs égaux, Il y a bien 1/4 de la population entre $Q_1$ et $Q_2$. \textbf{Vrai}.

\question{}
Par définition, \textbf{faux}.

\question{}
Par définition, \textbf{vrai}.

\question{}
Ne pas confondre quartiles et valeurs extrêmes. \textbf{Faux}.

\trait

\end{document}
