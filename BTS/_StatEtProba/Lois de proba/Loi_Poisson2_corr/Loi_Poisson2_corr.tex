\documentclass[a4paper,12pt]{scrartcl}

\usepackage{exercice_sheet}

%\trait
%\section*{}
%\exo{}
%\question{}
%\subquestion{}

\date{}


% Title Page
\title{Loi de Poisson, corrigé des exercices}

\author{\textsc{Mathématiques}}

\begin{document}

\maketitle

\exo{}

\question{}
$P(X=2) = 0.224$

\question{}
$P(X \leqslant 3) = 0.647$

\question{}
$P(X\geqslant 2) = 0.801$

\question{}
$E(X) = \lambda = 3$ et $\sigma(X) = \sqrt{\lambda} = \sqrt{3} \approx 1.732$


\exo{}

\question{}
$P(X = 4) = 0.047$

\question{}
$P(X \leqslant 2) = 0.809$

\question{}
$E(X) = \lambda = 1.5$ et $\sigma(X) = \sqrt{\lambda} = \sqrt{1.5} \approx 1.225$


\exo{}

$X$ suit la loi $\mathcal{P}(2.5)$ avec un $\mathcal{P}$ comme Poisson... 

\begin{itemize}
 \item $P(A) = P(X=0) = 0.082$
 \item $P(B) = P(X=2) = 0.544$
 \item $P(C) = P(X \geqslant 2) = 0.713$
\end{itemize}

\exo{}

\question{}
Le paramètre $\lambda = 5$ de cette loi de Poisson signifie qu'il y a en moyenne 5 personnes qui pointent leur fraise au guichet pour chaque période de 15 minutes. 

\question{}
La période est de 15 minutes. Il s'agit donc de calculer $P(X \geqslant 4)$.

$P(X\geqslant 4) = 0.735$

\question{}
$P(X\geqslant 5) = 0.560$


\exo{}

\question{}
$X$ sui la loi binomiale $\mathcal{B}(120;0.03)$.

\begin{itemize}
 \item $P(X = 2) = 0.177$
 \item $P(X = 5) = 0.140$
 \item $P(X \leqslant 2) = 0.298$
\end{itemize}

\question{}
On rappelle que l'espérance d'une loi binomiale est $np$. Ici, $np = 120 \times 0.03 = 3.6$. 

\question{}

\subquestion{}
Dans certaines conditions (qui sont remplies ici, l'énoncé le sous-entend sans s'épancher sur le sujet) la loi binomiale et la loi de Poisson peuvent être sensiblement égales\footnote{Attention, elles ne sont pas rigoureusement égales, elles sont seulement de valeurs \emph{très} proches, jusqu'à plusieurs chiffres après la virgule}. Pour pouvoir être de valeurs proches, les 2 lois \emph{doivent} avoir la même espérance, c'est une condition nécessaire. Or, le paramètre $\lambda$ (ou $\mu$ selon l'humeur) d'une loi de Poisson \emph{est} sa moyenne. $Y$ suit donc la loi $\mathcal{P}(3.6)$.

\subquestion{}
On peut maintenant calculer les probabilités de la variable discrète $Y$ comme dans les exercices précédents. 

\begin{itemize}
 \item $P(X = 2) = 0.177$
 \item $P(X = 5) = 0.138$
 \item $P(X \leqslant 2) = 0.303$
\end{itemize}

\exo{}

\question{}
Ici, l'exercice est inversé: on connaît une probabilité bien particulière de la variable (ici, la probabilité qu'elle soit égale à 0). 

On peut résoudre cette question en revenant à la formule:

\begin{equation}
 P(X=k) = \frac{e^{-\lambda} \times \lambda^k}{k!}
 \label{Eq:Poiscaille}
\end{equation}

Ainsi, l'équation $P(X = 0) = 0.2$ équivaut à $\dfrac{e^{-\lambda} \times \lambda^0}{0!} = 0.2$ ce qui est une équation d'inconnue $\lambda$.

Or, $0! = 1$ et $\lambda^0 = 1$. 

La formule (\ref{Eq:Poiscaille}) donne donc:

$$e^{-\lambda} = \frac{1}{5}$$

$$\Leftrightarrow e^{\lambda} = 5$$

$$\Leftrightarrow \lambda = \ln 5 \approx 1.609$$

\question{}
Le paramètre d'une loi de Poisson est la moyenne. Donc il n'y a ici aucun calcul à faire: $\lambda = E(X) = 2$.

\question{}
Mais c'est aussi -- dans le cadre de la loi de Poisson et uniquement dans celui-là -- le carré de l'écart-type. On rappelle que $\sigma(X) = \sqrt{\lambda}$. Donc ici, $\lambda = 2^2 = 4$.




\end{document}

