\documentclass[a4paper,12pt]{scrartcl}

\usepackage{exercice_sheet}

%\trait
%\section*{}
%\exo{}
%\question{}
%\subquestion{}

\date{}


% Title Page
\title{Permutations et combinaisons}

\author{\textsc{Mathématiques}}

\begin{document}

\maketitle

%\tableofcontents

\section{Permutation}

\begin{definition}{permutation}
 Une permutation d'un ensemble $E$ est une liste contenant tous les éléments de $E$ écrits dans un certain ordre. 
\end{definition}

Exemple: si $E = \{a; b; c\}$, quelles sont les permutations possibles de cet ensemble? Combien en existe-t-il?

\cadre{3}

Dans un ensemble contenant $n$ éléments, le nombre de permutations s'écrit $n!$, se lit \og{}factorielle $n$\fg{} et vaut: $n! = 1 \times 2 \times 3 \times \ldots \times (n-2) \times (n-1) \times n$.

Par convention, $0! = 1$.

Application: calculer le nombre d'anagrammes possibles du mot \og{}comptable\fg{}.

\cadre{3}

\section{Combinaisons}

\begin{definition}{combinaison}
 Soit un ensemble $E$ contenant $n$ éléments. 
 Une combinaison de $p$ éléments parmi $n$ est le choix de $p$ éléments de $E$.
\end{definition}

Exemple: si $E = \{a;b;c;d\}$, quelles sont les combinaisons de $p = 2$ éléments parmi les $n = 4$ éléments de $E$?

\cadre{3}

Le nombre de combinaisons de $p$ éléments parmi $n$ se note $C_n^p$ ou encore $\binom{n}{p}$ et vaut:

\begin{equation*}
 C_n^p = \binom{n}{p} = \dfrac{n!}{(n-p)! p!}
\end{equation*}


Cas particuliers: 

\begin{enumerate}
 \item $C_n^0 = 1$;
 \item $C_n^n = 1$.
\end{enumerate}

Calculer $C_{4}^2$, $C_{10}^4$ et $C_{10}^0$.

\cadre{2}

Example: le Loto.

Au loto, on tire 6 numéros parmi 49.

Combien existe-t-il de combinaisons? 

\cadre{2}


\end{document}

