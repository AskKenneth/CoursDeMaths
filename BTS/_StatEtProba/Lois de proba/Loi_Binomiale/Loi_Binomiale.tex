\documentclass[a4paper,12pt]{scrartcl}

\usepackage{exercice_sheet}

%\trait
%\section*{}
%\exo{}
%\question{}
%\subquestion{}

\date{}

\renewcommand{\hasnotesprof}{0}

% Title Page
\title{Loi binomiale}

\author{\textsc{Mathématiques -- Probabilités}}

\begin{document}

\maketitle

\tableofcontents

\section*{Introduction}

La loi binomiale sert à comptabiliser le nombre de succès sur un nombre fixé d'expériences aléatoires identiques et indépendantes entre elles.

\section{Propriétés}

\subsection{Épreuve binomiale}

\begin{definition}{épreuve binomiale}
Ou épreuve de \emph{Bernoulli}.

 On appelle épreuve binomiale toute épreuve aléatoire ne pouvant conduire qu'à 2 issues, appelées \emph{succès} ou \emph{échec}.
 
 Étant donné qu'il n'y a que 2 issues possibles, si $p$ et la probabilité de succès, alors $1-p$ est la probabilité d'échec.
\end{definition}

Exemples:

\begin{enumerate}
 \item Un lancer de pièce dont on lit le résultat \emph{pile} ou \emph{face} est une épreuve binomiale car il y a deux résultats possibles. On remarque que l'on peut attribuer l'issue \emph{succès} à pile et \emph{échec} à face ou l'inverse mais ceci est un choix arbitraire.
 \item On choisit une personne majeure au hasard dans la population. On s'intéresse au caractère \og{}est au chômage \fg{}. 
 
 Présenté de la sorte, l'issue \emph{succès} est d'obtenir une personne au chômage, l'issue \emph{échec} tomber sur une personne qui ne l'est pas (actifs, étudiants, arrêts maladie, etc.).
\end{enumerate}

\subsection{Loi binomiale}

\subsubsection{Cadre d'utilisation}\label{par:cadre}

La loi binomiale s'utilise dans un cadre bien précis: on répète $n$ fois une épreuve binomiale et on s'intéresse à la variable $X$ qui compte le nombre de succès.

Exemple: on lance 20 fois une pièce et on considère que \emph{pile} est un succès. On obtient 7 fois \emph{pile}. On a alors dans ce cas:

\begin{enumerate}
 \item $n = 20$ puisque l'on répète l'expérience binomiale 20 fois;
 \item $p = 0.5$ puisque la probabilité de succès dans ce cas est 0.5 (pile ou face);
 \item $X = 7$ car on a obtenu 7 succès.
\end{enumerate}

$n$ et $p$ sont les \textbf{paramètres} de la loi binomiale. 

On écrit alors $\mathcal{B}(n;p)$, ce qui se lit \og{}loi binomiale de paramètre $n$ et $p$\fg{}. Dans cet exemple, on écrit: $\mathcal{B}(20;0.5)$. Enfin, la probabilité de l'issue obtenue ici (7 succès) s'écrit $P(X=7)$. 

Pour justifier qu'une variable aléatoire suit une loi binomiale, il suffit donc de vérifier trois points:

\begin{itemize}
 \item le nombre d'épreuve est fixé à l'avance (dans les exemples précédents, ce nombre d'épreuves s'appelle $n$);
 \item $X$ mesure le nombre de succès d'une suite d'épreuves ne comportant chacune que deux issues;
 \item les épreuves sont \textit{identiques} et \textit{indépendantes}.
\end{itemize}

\subsubsection{Valeurs caractéristiques}
Pour une variable aléatoire $X$ suivant une loi binomiale de paramètres $n$ et $p$, on peut écrire $X = \mathcal{B}(n;p)$. On remarquera que $X$ ne peut prendre que des valeurs \emph{entières}.

\paragraph{Formule générale}
La probabilité que $X$ prenne la valeur $k$ avec $0 \leqslant k \leqslant n$, notée $P(X=k)$ est donnée\footnote{Autrement dit, la probabilité d'obtenir $k$ succès...} par:

\begin{equation}
 P(X=k) = C_n^k p^k (1-p)^{n-k}
\end{equation}

\paragraph{Espérance et écart-type}

L'espérance et l'écart-type pour une variable $X$ suivant une loi binomiale $\mathcal{B}(n;p)$ se calculent ainsi:

\begin{itemize}
 \item $E(X) = np$
 \item $\sigma_X = \sqrt{np(1-p)}$
\end{itemize}




Dans l'exemple cité dans le paragraphe \ref{par:cadre}, on a donc: 

$$P(X=7) = C^7_{20} \times 0.5^7 \times 0.5^{20-7} = 0.0739$$ 

Une probabilité pouvant s'exprimer en termes de pourcentages, on peut écrire $P(X=7) = 7.39$ \% autrement dit lors de cette expérience, on a environ $7.39$ \% de chances d'obtenir sept fois pile sur les 20 lancers.

L'espérance de la variable $X$ est $E(X) = 20 \times 0.5 = 10$. Ceci signifie que sur 20 lancers et si je répète l'expérience consistant à lancer une pièce 20 fois, $X$ aura une valeur moyenne de 10.

Son écart-type est $\sigma_X = \sqrt{20 \times 0.5 \times 0.5} = \sqrt{5} \approx 2.24$. L'écart-type se comprend comme en statistiques. C'est un paramètre de dispersion de la variable $X$: plus $\sigma_X$ est grand plus $X$ variera d'une expérience\footnote{Les 20 lancers} à l'autre.

\subsubsection{Un exemple; un contrexemple}

On considère une urne dans laquelle se trouve 10 balles rouge et 30 balles vertes.

On prélève suivant deux procédures 5 balles dans cette urne et on compte le nombre $X$ de balles rouges obtenues.

\paragraph{Tirage sans remise:} les 5 balles sont prélevées l'une après l'autre sans remise;

\paragraph{Tirage avec remise:} on tire une balle, on note sa couleur, on la remet dans l'urne puis on recommence jusqu'à la cinquième balle. 

Dans lequel de ces deux cas la variable $X$ suit-elle une loi binomiale?

\cadre{3}

\section{Applications}

\section*{Exercices}

\exo[1]{Encore une petite pièce}

On jette 5 fois une pièce de monnaie. On note $X$ la variable aléatoire donnant le nombre de pile obtenu.

\question{}Quelles valeurs peut prendre $X$?

\question{}Justifier que X suit une loi de probabilité binomiale dont on déterminera les paramètres. 

\question{}Calculer la probabilité d'obtenir 4 piles. 

\question{}Calculer la probabilité d'obtenir au plus 2 piles.


\exo[1]{Mazout}
Après une marée noire en Bretagne, l'orgasme de protection des oiseaux de mer a évalué à 20 000 la population de sternes au large du Finistère. 500 d'entre eux ont été bagués. Un an après, on capture 100 sternes dans cette zone.

\question{}Calculer la probabilité de ne pas avoir d'oiseau bagué ?

\question{}D'avoir au moins 2 oiseaux bagués ?


\exo[3]{Assurance}
Une grande mutuelle d'assurance envisage d'éventuels changements de tarifs. Pour cela, elle a étudié le risque d'accident automobile des ses assurés en fonction de leur ancienneté de leur permis. Parmi ses assurés, il y a 20 \% de jeunes ayant leur permis depuis moins de 5 ans et le risque d'accident au cours de l'année de ces jeunes est de 0.4.

Le risque d'accident des assurés ayant leur permis depuis plus de 5 ans est de 0.125.

\question{}Si on choisit au hasard 10 jeunes conducteurs, quelle est la probabilité d'en voir au moins un ayant un accident dans l'année?

\question{}Même question avec 10 assurés ayant leur permis depuis plus de 5 ans.

\question{}Dans cette question, on s'intéresse à l'ensemble des assurés: les jeunes ($J$) et les autres ($\overline{J}$). 

La probabilité $p$ qu'un assuré pris au hasard ait un accident ($A$) est: 

$$p = P(A) = P(A \cap J) + P\left(A \cap \overline{J}\right)$$

\subquestion{}En utilisant les probabilités conditionnelles, calculer $p$.

\subquestion{}En prenant 10 assurés au hasard, quelle est la probabilité d'en voir au moins un ayant un accident dans l'année.
 




\end{document}

