\documentclass[a4paper,12pt]{scrartcl}
\usepackage[utf8x]{inputenc}
\usepackage[T1]{fontenc} % avec T1 comme option  d'encodage c'est ben mieux, surtout pour taper du français.
%\usepackage{lmodern,textcomp} % fortement conseillé pour les pdf. On peut mettre autre chose : kpfonts, fourier,...
\usepackage[french]{babel} %Sans ça les guillemets, amarchpo
\usepackage{amsmath}
\usepackage{multicol}
\usepackage{amssymb}
\usepackage{tkz-tab}
\usepackage{exercice_sheet}

%\trait
%\section*{}
%\exo{}
%\question{}
%\subquestion{}

\date{}


% Title Page
\title{Probabilités conditionnelles}

\author{\textsc{Mathématiques}}

\begin{document}

\maketitle

\tableofcontents

\section{Étude}

\subsection{Lancer de dés}

On considère comme expérience aléatoire le lancer d’un dé à six faces non pipé à l’issue de laquelle on envisage les événements suivants:

\begin{itemize}
\item $A =$ \og J’obtiens un résultat pair  \fg{}
\item $B =$ \og j’obtiens un 4 \fg{}
\end{itemize}

Détailler les événements élémentaires les ensembles suivants : 	

\begin{itemize}
\item l'univers des possibles: $\Omega$
\item $A$
\item $B$
\item $A \cap B$
\end{itemize}

\cadre{3}

Calculer $P(A)$, $P(B)$ et $P(A \cap B)$.

\cadre{3}

\subsection{Lancer de dés (bis)}

On considère comme expérience aléatoire le lancer d’un dé à six faces non pipé à l’issue de laquelle on envisage les événements suivants :

\begin{itemize}
\item $A = $ \og J'obtiens un multiple de 3 \fg{} 
\item $B = $ \og J'obtiens un nombre pair \fg{} 
\end{itemize}

Détailler les événements élémentaires les ensembles suivants : 	

\begin{itemize}
\item l'univers des possibles: $\Omega$
\item $A$
\item $B$
\item $A \cap B$
\end{itemize}

\cadre{3}

Calculer $P(A)$, $P(B)$ et $P(A \cap B)$.

\cadre{3}

Calculons directement la probabilité de l’événement $B$ sachant que $A$ est réalisé $P(B/A)$. 

Si $A$ est réalisé, l'univers des possibles se réduit à $\Omega' = $

\cadre{2}

Donc on en déduit que: $P(B/A)$ 

\cadre{2}

Vérifier que l’on a $P(B/A) = \dfrac{P(A \cap B)}{P(A)}$

\cadre{3}

\textbf{Remarque:} dans ce cas $P(B/A) = P(B)$. Les événements $B$ et $A$ sont indépendants : savoir que l’événement $A$ est réalisé ou pas n’apporte aucune information utile sur la probabilité de réalisation de l’événement $B$. Dans ce cas on peut vérifier que $P(A \cap B) = P(A) \times P(B)$:

\cadre{3}

\section{Définitions et propriétés}

\subsection{Probabilités conditionnelles}

\begin{definition}

Soit $A$ un événement de probabilité non nulle. La probabilité conditionnelle de $B$ relative à $A$ ou probabilité de $B$ sachant que $A$ est:

\Answer{$P_A(B) = P(B/A) = \dfrac{P(A \cap B)}{P(A)}$}
 
\end{definition}

\textbf{Propriété :} $P(A \cap B) = P(A) \times P(B/A) = P(B) \times P(A/B)$. 

\subsection{Événements indépendants}

Deux événements $A$ et $B$ sont indépendants si et seulement si l’une des conditions suivantes est réalisée: 

\begin{itemize}
\item $P(A/B = P(A)$
\item $P(B/A = P(B)$
\item $P(A \cap B) = P(A) \times P(B)$
\end{itemize}

\section{Exemples d'application}

\subsection{Application directe des formules}

Soient $A$ et $B$ deux événements tels que $P(A) = 0.5$, $P(B) = 0.4$ et $P(A \cap B) = 0.1$.

\begin{itemize}
\item Les événements sont-ils incompatibles? 
\item Les événements sont-ils indépendants?
\end{itemize}

\end{document}

