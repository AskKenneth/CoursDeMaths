\documentclass[a4paper,12pt]{scrartcl}

\usepackage{exercice_sheet}

%\trait
%\section*{}
%\exo{}
%\question{}
%\subquestion{}

\date{}

\renewcommand{\hasnotesprof}{0}


% Title Page
\title{Interprétation des consignes}

\author{\textsc{Mathématiques, Français}}

\begin{document}

\maketitle

\paragraph{\og{}Résoudre dans ($\mathbb{R}, [0;+\infty[$, etc.)\fg{}}
\begin{itemize}
 \item Résoudre l'équation en ne tenant compte que des solutions appartenant à l'ensemble cité ($\mathbb{R}, [0;+\infty[$, etc.). 
\end{itemize}


\paragraph{\og{}Résoudre graphiquement\fg{} ou toute consigne contenant le mot \og{}graphiquement\fg{}}
\begin{itemize}
 \item Lecture graphique sans raisonnement à développer sur la copie.
\end{itemize}


\paragraph{\og{}Montrer que...\fg{}}
\begin{itemize}
 \item Établir le raisonnement menant à...
 
 Ne signifie pas nécessairement que le raisonnement doit être long ou complexe.
\end{itemize}


\paragraph{\og{}En déduire...\fg{}}
\begin{itemize}
 \item \og{}Montrer que\fg{} à partir du résultat de la question précédente.
\end{itemize}


\paragraph{\og{}Justifier que...\fg{}}
\begin{itemize}
 \item Énoncer, souvent de façon succincte, les raisons qui font que. Peut aussi avoir une signification proche de \og{}Montrer que...\fg{}.
\end{itemize}


\paragraph{\og{}Exprimer $y$ en fonction de $x$\fg{}}
\begin{itemize}
 \item Écrire l'expression $y=\ldots$ avec une expression à base de $x$ sur les points de suspension.
\end{itemize}


\end{document}

