\documentclass[a4paper,12pt]{scrartcl}
\usepackage[utf8x]{inputenc}
\usepackage[T1]{fontenc} % avec T1 comme option  d'encodage c'est ben mieux, surtout pour taper du français.
%\usepackage{lmodern,textcomp} % fortement conseillé pour les pdf. On peut mettre autre chose : kpfonts, fourier,...
\usepackage[french]{babel} %Sans ça les guillemets, amarchpo
\usepackage{amsmath}
\usepackage{multicol}
\usepackage{amssymb}
\usepackage{tkz-tab}
\usepackage{exercice_sheet}


%\trait
%\section*{}
%\exo{}
%\question{}
%\subquestion{}

\title{Suites numériques}

\date{}

%\author{}

\begin{document}

\maketitle

\tableofcontents

\section{Généralités}

\subsection{Définition}

Une suite est une famille de nombres (ses termes) indexés par un entier naturel.

\subsection{Notation}

On considère la suite $\left( u_n \right)_{0 \leqslant n \leqslant 4}$ des 5 nombres suivants : 3 ; 7 ; 5 ; -2 ; 4.

Dans cette notation, $n$ est un entier naturel compris entre 0 et 4 inclus.

\begin{itemize}
\item le premier terme de la suite se note $u_0$ et vaut : ...
\item le deuxième terme de la suite se note $u_1$ et vaut : ...
\item le troisième terme de la suite se note $u_2$ et vaut : ...
\item etc.
\end{itemize}

À noter que $u_n$ désigne \og le terme de rang $n$ \fg{} qui est un \textbf{nombre} et que $u_n$ avec les parenthèses désigne \og la suite $(u_n)$ \fg{}.

\subsection{Définir une suite}

\subsubsection{Définition d'une suite par son \emph{terme général}}

Il s'agit d'exprimer chaque terme de la suite $\left( u_n \right)$ en fonction de $n$. 

À noter que cette façon de définir une suite est similaire à la définition d'une fonction à partir de $x$ : $f(x) = 2x^{2}$.

Soit la suite $\left( u_n \right)$ définie par $u_n = 2n^{2}$ pour $n \in \mathbb{N}$.

Calculer $u_0$, $u_1$, $u_{10}$ et le quatorzième terme de la suite.

\cadre{3}

\subsubsection{Définition d'une suite par \emph{récurrence}}

Définir une suite \emph{par récurrence}, c'est exprimer chaque terme de la suite en fonction d'un ou plusieurs termes précédents. Pour que la suite soit complètement définie, il faut également connaître le premier terme.

Avec une telle définition, pour calculer le $n$\textsuperscript{ème} terme, il faut calculer tous les termes précédents.

Exemple: soit la suite $\left( u_n \right)$ définie par : 
$\begin{cases}
u_0 = 2\\
u_{n+1} = 4 u_n - 3
\end{cases}$

Calculer $u_1$, $u_3$ et $u_{10}$.

\cadre{3}

Exemple\footnote{Cette suite s'appelle la suite de \emph{Fibonacci}} soit la suite $\left( u_n \right)$ définie par : 
$\begin{cases}
u_0 = 1\\
u_1 = 1\\
u_{n+2} = u_{n+1} + u_{n}
\end{cases}$

Calculer $u_2$, $u_3$ et $u_{8}$.

\cadre{3}

\section{Des suites particulières}

Nous allons voir dans cette partie deux types de suites qui sont souvent étudiés. Les suites \emph{arithmétiques} et les suites \emph{géométriques}.

\subsection{Suites arithmétiques}

\subsubsection{Définition}

Une suite arithmétique de raison $r$  est une suite de réels tels que chacun d'eux soit égal à celui qui le précède augmenté de $r$ (où $r$ est un réel différent de 0). Cette suite est donc définie par la donnée du premier terme $u_0$ et de la relation de récurrence : $u_{n+1} = u_n + r$.

\begin{itemize}
\item La suite 5 ; 8 ; 11 ; 14 ; 17 est une suite arithmétique car pour passer d'un terme au suivant, on additionne 3. La raison $r$ est donc $r = 3$. Le premier terme est 5.
\item La suite des nombre pairs est une suite arithmétique de premier terme $u_0 = ...$ et de raison $r = ...$.
\end{itemize}



Les nombres 15 ; 11 ; 7 ; 3 ; -1 sont-ils les 5 premiers termes d'une suite arithmétique ? Si oui, indiquer le premier terme et la raison.

\cadre{2}

Les nombres 3 ; 5 ; 7 ; 10 ; 12 sont-ils les 5 premiers termes d'une suite arithmétique ?

\cadre{2}

La suite $(u_n)$ définie pour $n \in \mathbb{N}$ par $u_n = 3n+4$ est-elle une suite arithmétique? Si oui, indiquer le 1er terme et la raison.	

\cadre{3}

\subsubsection{Expression du terme de rang $n$}

Considérons la suite arithmétique de raison $r$ : $u_0$ ; $u_1$ ; $u_2$ ; ... ; $u_n$.

Exprimer chaque terme en fonction de $u_0$ et $r$.

\begin{itemize}
\item $u_1 = u_0 + r$
\item $u_2 = u_1 + r = ...$
\item $u_3 = u_2 + r = ...$
\item etc.
\end{itemize}

D'où la relation générale : $u_n = n \times r + u_0$.

\exemple Soit la suite de premier terme $u_0 = 5$ et de raison $r = 3$.

\begin{itemize}
\item Calculer $u_9$
\item Calculer le quinzième terme.
\end{itemize}

\cadre{3}

\subsubsection{Calcul de la somme des $n+1$ premiers termes}

Soit $S_n$ la somme des $n+1$ premiers termes d'une suite arithmétique de raison $r$ : $S_n = u_0 + u_1 + ... + u_{n-1} + u_n$.

On peut\footnote{Ça veut dire qu'on le fera pas en cours et qu'on admet le résultat} montrer que:

\Answer{$S_n = (n+1) \times \dfrac{u_0 + u_n}{2}$}

$n+1$ est le nombre de termes additionné, $u_0$ est le premier terme et $u_n$ le dernier.

De manière plus générale, il se peut que le premier terme additionné ne soit pas $u_0$: on peut en effet additionner du rang $p$ au rang $n$. Ainsi, la somme des termes se calcule comme suit:

\Answer{$S = (n-p+1) \times \dfrac{u_p + u_n}{2}$}

Que l'on peut également écrire: 

$S = \mbox{nb. de termes additionnés} \times \dfrac{\mbox{premier terme} + \mbox{dernier terme}}{2}$

\exemple Calculer la somme des 15 premiers termes d'une suite arithmétique de raison -2 et de premier terme 24:

\cadre{3}

\exemple $(u_n)$ étant une suite arithmétique, écrire la formule permettant de calculer $S = u_7 + u_8 + ... + u_{14} + u_{15}$.

\cadre{3}


\subsection{Suites géométriques}

\subsubsection{Définition}

Une suite géométrique de raison $q$ est une suite de réels tels que chacun d'eux soit égal à celui qui le précède multiplié par $q$ ($q$ est un réel différent de 0 et de 1). Cette suite est donc définie par la donnée du premier terme $u_0$ et de  la relation de récurrence : 

\Answer{$u_{n+1}= q \times u_n$}

\exemple La suite des puissances de deux : 1 ; 2 ; 4 ; 8 ; 16 ; 32 ; 64 ; 128 est une suite géométrique de premier terme $u_0 = ...$ et de raison  $q = ...$

\exemple la suite de nombres -8 ; 16 ; -32 ; 64 ; -128 ; 256   est une suite géométrique de premier terme $u_0 = ...$ et de raison  $q = ...$

\exemple 8 ; 4 ; 2 ; 1 ; 0.5 est une suite géométrique de premier terme $u_0 = ...$ et de raison  $q = ...$

\vspace{1em}

\textbf{Pour prouver qu'une suite est une suite géométrique il suffit de montrer que le quotient $\frac{u_{n+1}}{u_n}$  est constant quel que soit $n$. Ce quotient constant est la raison de la suite géométrique.}

\exemple Les nombres 5 ; 1 ; 0,2 ; 0,04 sont-ils les 4 premiers termes consécutifs d'une suite géométrique ? Si oui, indiquer le 1er terme et la raison.

\cadre{2}

\exemple Les nombres 4 ; 5 ; 8 ; 10 sont-ils les termes consécutifs d'une suite géométrique ? Si oui, indiquer le 1er terme et la raison.

\cadre{2}

\exemple La suite $(u_n)$ définie pour $n \in \mathbb{N}$ par : $u_n = e^{2n}$ est-elle une suite géométrique ? Si oui, indiquer le premier terme $u_0$ et la raison.

\cadre{2}

\subsubsection{Expression  du terme de rang $n$}

Considérons la suite géométrique de raison $q$ : $u_0$ ;$u_1$ ; $u_2$ ; ... ; $u_n$

Exprimer chaque terme en  fonction de $u_0$ et $q$.
\begin{itemize}
\item $u_1 = u_0 × q$
\item $u_2 = u_1 × q =... $
\item $u_3 =...$
\item etc.
\end{itemize}

D'où la relation générale : 

\Answer{$u_n = u_0 \times q^n$}

\exemple Calculer $u_4$ et le 8ème terme d'une suite géométrique de raison 2 et de premier terme -5.

\cadre{3}

Soit une suite géométrique de raison 1.04 et de premier terme $u_0 > 0$. À partir de quelle valeur de $n$, $u_n$ dépasse-t-il le double de $u_0$ ?

\cadre{3}

\subsubsection{Somme de termes consécutifs}

Soit $S$, la somme des termes de la suite du rang $p$ au rang $n$ avec $p < n$. On peut montrer que:

\Answer{$S = u_p \dfrac{1-q^{n-p+1}}{1-q}$}

Où $n-p+1$ est le nombre de termes additionnés, $u_p$ le premier terme additionné et $u_n$ le dernier terme additionné.

Que l'on peut également écrire: $S = \mbox{premier terme} \times \dfrac{1-q^{\mbox{nb. de termes additionnés}}}{1-q}$

\exemple Calculer la somme des 10 premiers termes d'une suite géométrique de raison 2 et de premier terme 3.

\cadre{3}

\exemple $(u_n)$ étant une suite géométrique, écrire la formule qui permet de calculer $S = u_7 + u_8 + ... + u_{14} + u_{15}$.

\cadre{3}

\subsubsection{Limite d'une suite géométrique}

Pour calculer la limite d'une suite géométrique de raison $q$ on utilise les résultats suivants : 

\begin{itemize}
\item $q > 1$. La limite est $+\infty$ si $u_0 > 0$, $-\infty$ si $u_0 < 0$;
\item $-1 < q < 1$. La limite est 0;
\item $q < -1$. La suite n'a pas de limite.
\end{itemize}

\exemple Soit $u_n$ la suite géométrique de raison $q = 3$ et de premier terme $u_0 = -2$. Donner la limite de $u_n$, $\lim u_n$. 

\cadre{2}

\exemple Soit $u_n$ la suite géométrique de raison $q = 1/2$ et de premier terme $u_0 = 1$. Donner la limite de $u_n$, $\lim u_n$. 

\paragraph{Récapitulatif sur les limites de suites:}

soit la suite géométrique $(u_n)$ de raison $q$ et de premier terme $u_0$. On peut alors déduire la limite de la suite (si cette limite existe) comme suit:

\begin{center}
\begin{tabular}{|l|l|l|}\hline
 & $u_0 > 0$ & $u_0 < 0$ \\ \hline 
$q>1$ & $\lim u_n = +\infty$ & $\lim u_n = -\infty$ \\ \hline 
$-1 < q < 1$ & $\lim u_n = 0$ & $\lim u_n = 0$ \\ \hline 
$q < -1$ & pas de limite & pas de limite \\ \hline 
\end{tabular}
\end{center}

\subsection*{Récapitulatif sur les suites géométriques et arithmétiques}

%\begin{table}[h]
\begin{center}
\begin{tabular}{|l|l|l|}\hline
 & \textbf{Arithmétique}  &    \textbf{Géométrique}     \\\hline
\textbf{Terme général}  & $u_n = u_0 + r \times n$  & $u_n = u_0 \times q^n$ \\\hline
\textbf{Définition par récurrence} & 

$
\left\lbrace
\begin{array}{l }
   u_0  \\
   u_{n+1} = u_n + r \\
\end{array}
\right.
$       
            
            &

$
\left\lbrace
\begin{array}{l }
   u_0  \\
   u_{n+1} = u_n \times q \\
\end{array}
\right.
$ 
            
                       \\\hline
\textbf{\og{}Montrer que $u_n$ est... \fg{}} & $u_{n+1} - u_n = ...$  & $\dfrac{u_{n+1}}{u_n} = ...$ \\\hline
\textbf{Somme de $n$ termes consécutifs} & $S = (n-p+1) \dfrac{u_p+u_n}{2}$ & $S = u_p \dfrac{1-q^{n-p+1}}{1-q}$\\ \hline

\end{tabular}
\end{center}
%\end{table}

\section{Étude d'une suite définie par une relation de récurrence du type :  $u_{n+1} = a {u_n + b}$.}

Il s'agit là d'un exercice type qui revient en examen quasiment à coup sûr lorsqu'il y a un exercice sur les suites.

On considère la suite $(u_n)$ définie par :  $u_0  = 2$ et pour tout entier naturel $n$, $u_{n+1} = 2 u_n + 3$.

Calculer $u_1$, $u_2$. La suite $(u_n)$ est-elle géométrique ? Justifier.

\cadre{4}

Pour tout entier naturel $n$, on pose $v_n = u_n+3$.

En exprimant $v_{n+1}$ en fonction de $u_{n+1}$ puis en fonction de $u_{n}$, déterminer l'expression de $v_{n+1}$ en fonction de $v_{n}$. En déduire la nature de la suite $(v_n)$.

\cadre{6}

Exprimer $v_n$ en fonction de $n$ puis $u_n$ en fonction de $n$.

\cadre{6}

Déterminer la limite de la suite quand $n$ tend vers $+\infty$. 

% \section{Algorithmique}
% 
% \subsection{Définition}
% 
% Un algorithme\footnote{Du nom du mathématicien perse Al-Khwârizmî.} est une suite finie d'instructions opératoires sans ambiguïté ayant pour but de résoudre un problème.
% 
% Des casse-têtes tels que le Rubik's Cube peuvent être résolus à l'aide d'algorithmes (il en existe beaucoup, comme le couche par couche, sandwich, etc.). 
% 
% \subsection{}



\section*{Exercices}

\exo[1]{Suites arithmétiques}

\question{}
Calculer le 36ème terme de la suite arithmétique 5; 8; 11; 14; ...

\question{}
Quelle est la raison d'une suite arithmétique de 1er terme $k_0 =7$ , de 15ème terme $k_{14} = 77$. Calculer la somme de ses 15 premiers termes.

\question{}
Écrire la suite arithmétique de 8 termes connaissant le premier $p_0 = 35$ et le dernier $p_7 = 112$. 

\question{}
Calculer la somme de 20 termes consécutifs d'une suite arithmétique, le premier étant 5 et le dernier 57. Quelle est sa raison?

\question{}
Montrer que la suite $(u_n)$ définie par la relation $u_n = 3n+17$ est une suite arithmétique. En donner la raison et le premier terme.

\exo[1]{Suites géométriques}

\question{}
On considère une suite géométrique de premier terme $u_0 = 2$ et de raison 3. Calculer $u_7$, $u_{17}$.

\question{}
Calculer la somme des 10 premiers termes d'une suite géométrique, le premier étant 100 et la raison 2,5.

\question{}
La somme des $n+1$ premiers termes d'une suite géométrique de premier terme 10000 et de raison $q = 1.5$, est $S_n = 131 875$. Calculer $n$. 


\probleme[2]{}
Une entreprise achète une machine au prix de 9 000 euros. Elle estime que cette machine se déprécie de 20\% par an. On note $V_0$ la valeur initiale de la machine. Les nombres $V_1$, $V_2$, $V_3$ et $V_4$ représentent les valeurs de la machine au bout d'un an, de deux ans, de trois ans, de quatre ans.

\question{}
Calculer $V_1$ et $V_2$.

\question{}
Montrer que les nombres $V_0$, $V_1$ et $V_2$ forment une suite géométrique dont on précisera la raison.

\question{}
On admet que $V_n = 9000 \times 0,8^n$. En utilisant cette formule, calculer $V_3$ et $V_4$.

\question{}
Au bout de combien d'années la valeur de la machine devient-elle inférieure à la moitié de sa valeur initiale ?


\probleme[1]{}
La fraction de lumière transmise par un dioptre air-verre est de 96\%. Calculer la quantité de lumière transmise par un téléobjectif comportant 5 lentilles.


\probleme[2]{}

On considère la suite $(u_n)$ définie par $u_0 = 900$ et pour tout entier naturel $n$, $u_{n+1} = 0.6 u_n + 	200$.

\partie{}

\question{}
Calculer $u_1$ et $u_2$.

\question{}
On considère la suite $(v_n)$ définie  pour tout entier naturel $n$, par $v_n = u_n - 500$.

\subquestion{}
Montrer que la suite $(v_n)$ est une suite géométrique dont on donnera le premier terme et la raison.

\subquestion{}     
Exprimer $v_n$ en fonction de $n$.

\subquestion{}
En déduire que $u_n = 400 \times0.6^n + 500$.

\subquestion{}
Déterminer la limite de la suite $(u_n)$.

\partie{}
Deux sociétés A et B se partagent, dans un pays, le marché des télécommunications. Les clients souscrivent, le 1er janvier, soit auprès de A, soit de B, un contrat d'un an au terme duquel ils sont libres à nouveau de choisir entre A et  B. L'an 2006, la société A détient 90\% du marché et la société B, qui vient de se lancer, 10\%.
On estime que, chaque année 20\% de la clientèle de A change pour B, et de même 20\% de la clientèle de B change pour A.
On considère une population représentative de 1000 clients de l'année 2006. Ainsi 900 sont clients de la société A et 100 sont clients de la société B. 
On veut étudier l'évolution de cette population les années suivantes.

\question{La société A}

\subquestion{}
Vérifier que la société A compte 740 clients en 2007. Calculer le nombre de clients en 2008.

\subquestion{}
On note le nombre de clients de A l'année $2006 + n$.
Établir que $a_{n+1} = 0.8 a_n + 0.2 (1000-a_n)$. En déduire que : $a_{n+1} = 0.6 a_n + 200$.

\question{}
En utilisant le résultat de la première partie, que peut-on prévoir pour l'évolution du marché des télécommunications dans ce pays ?

\probleme[3]{}
Le 1er janvier 2006, une grande entreprise compte 1 500 employés. Une étude montre que lors de chaque année à venir, 10\% de l'effectif du 1er janvier partira à la retraite au cours de l'année. Pour ajuster ses effectifs à ses besoins l'entreprise embauche 100 jeunes dans l'année. 

Pour tout entier naturel $n$, on appelle, le nombre d'employés de l'entreprise le 1er janvier de l'année $( 2006 + n)$.

\question{La suite $(u_n)$.}

\subquestion{}
Calculer $u_0$, $u_1$ et $u_2$. La suite de terme général $u_n$ est-elle arithmétique, géométrique ? Justifier les réponses.

\subquestion{}
Expliquer pourquoi on a, pour tout entier naturel $n$, $u_{n+1} = 0.9 u_n + 100$. 

\question{}
Pour tout entier naturel $n$, on pose : $v_n = u_n - 1000$. 

\subquestion{}
En exprimant $v_{n+1}$ en fonction de $u_{n+1}$, puis en fonction de $u_{n}$, déterminer l'expression de $v_{n+1}$ en fonction de $v_{n}$.

\subquestion{}
Exprimer $v_{n}$ en fonction de $n$.

En déduire que, pour tout entier naturel $n$, $u_n = 500 \times 0.9^n + 1000$.

\subquestion{[Optique seulement]}

Déterminer la limite de la suite $(u_n)$ quand $n$ tend vers plus l'infini.

\question{}
Au 1er janvier 2006, l'entreprise compte un sur effectif de 300 employés. À partir de quelle année, le contexte restant le même, l'entreprise ne sera-t-elle plus en sur effectif ?

\probleme[3]{Des petits trous}
Une entreprise réalise un forage.

\begin{itemize}
\item le premier mètre de forage a pour coût 200 €.
\item le second mètre de forage a pour coût 220 €.
\item le troisième mètre de forage a pour coût 240 €
\end{itemize}

Et ainsi de suite : le coût de chaque mètre supplémentaire augmente de 20 € par rapport au coût du mètre précédent.

\partie{Calculs de coûts}

\question{}
Donner le coût du 4ème mètre de forage.

\question{}
Calculer le coût total d'un forage de 4 mètres.

\partie{Étude d'une suite}

On note : 

\begin{itemize}
\item $u_0$ le nombre correspondant au coût du 1er mètre de forage ; $u_0 = 200$ ;
\item $u_1$ le nombre correspondant au coût du 2e mètre de forage ; $u_1 = 220$ ;
\item $u_2$ le nombre correspondant au coût du 3e mètre de forage ; $u_2 = 240$ ; et ainsi de suite...
\item $u_n$ le nombre correspondant au coût du (n+1)\textsuperscript{ème} mètre de forage 
\end{itemize}

\question{}
Indiquer si la suite $(u_n)$ est une suite arithmétique ou géométrique. Le cas échéant, en donner la raison.

\question{}
Exprimer $u_n$ en fonction de $n$.

\question{}
Calculer $u_{10}$.

\partie{Exploitation}

\question{}
Calculer le coût total d'un forage dont la profondeur est de 11 mètres

\question{}
Montrer que le coût total d'un forage de $n$ mètres ($n \in \mathbb{N}$) est $c_n = 190n + 10n^2$

\question{}
Quelle profondeur peut-on atteindre si l'on dispose d'un budget de  6500 € ?

\probleme[2]{Nouvel atelier}


Dans une entreprise, on installe un nouvel atelier. Pendant la période de mise en route la production de $n$ième jour ($n$ entier non nul) est donnée par $u_n = 80 - 27 e^{-0.1n}$ unités.

\question{}
Montrer que la production est croissante c'est-à-dire que $u_{n+1} \ge u_n$.

\question{}
Au bout de combien de  jours la production dépassera-t-elle les 72 unités?

\question{}
On pose $v_n = e^{-0.1n}$ où $n$ est un entier naturel non nul.

\subquestion{}
Montrer que la suite $(v_n)$ est une suite géométrique dont on donnera la raison.

\subquestion{}
Calculer la somme : $S = v_1 + v_2 + \ldots + v_{12}$. 

\question{}
À la suite d'une avarie, l'atelier est arrêté après 12 jours de fonctionnement. En utilisant le résultat obtenu à la question précédente, donner la production totale obtenue pendant la période de fonctionnement (valeur arrondie à l'unité)

\end{document}
