\documentclass[a4paper,12pt]{scrartcl}
\usepackage[utf8x]{inputenc}
\usepackage[T1]{fontenc} % avec T1 comme option  d'encodage c'est ben mieux, surtout pour taper du français.
%\usepackage{lmodern,textcomp} % fortement conseillé pour les pdf. On peut mettre autre chose : kpfonts, fourier,...
\usepackage[french]{babel} %Sans ça les guillemets, amarchpo
\usepackage{amsmath}
\usepackage{multicol}
\usepackage{amssymb}
\usepackage{tkz-tab}
\usepackage{exercice_sheet}

%\trait
%\section*{}
%\exo{}
%\question{}
%\subquestion{}

\date{}


% Title Page
\title{Cours \og Fonction exponentielle et logarithme \fg{}, corrigé des exercices}

\author{Mathématiques}

\begin{document}

\maketitle


\exo{}

$f\left(0.35 \right) \approx -219.89$ ;
$g\left( \frac{2}{3} \right) \approx -2.39$ ; 
$h\left(\sqrt{2} \right) \approx -0.27$ ;
$i\left(5.2 \right) \approx 6.72$.

\exo{En l'absence de précision, on peut arrondir au millième...}

$$\ln \left( \frac{5}{3} \right) \approx 0.511$$

$$\frac{\ln 5}{\ln 3} \approx 1.465$$

$$\ln 5 - \ln 3 \approx 0.511$$

On remarque que $\ln \left( \frac{5}{3} \right) = \ln 5 - \ln 3$. C'est normal et découle d'une propriété du logarithme. 

\exo{Résoudre les équations:}

\question{}
$2.2^x = 3 \Leftrightarrow \ln(2.2^x) = \ln(3) \Leftrightarrow x \ln(2.2) = \ln(3) \Leftrightarrow x = \dfrac{\ln(3)}{\ln(2.2)} \approx 1.393$

$S = \left\lbrace \dfrac{\ln(3)}{\ln(2.2)} \right\rbrace$

\question{}
$2^x - 1 = 8 \Leftrightarrow 2^x = 9 \Leftrightarrow x = \dfrac{\ln 9}{\ln 2}$

$S = \left\lbrace \dfrac{\ln 9}{\ln 2} \right\rbrace$

\question{}
$\left( \dfrac{3}{4} \right)^x - 2 = 4 \Leftrightarrow \left( \dfrac{3}{4} \right)^x = 6 \Leftrightarrow x = \dfrac{\ln\left( 6 \right)}{\ln(\frac{3}{4})} \Leftrightarrow \frac{\ln 3 + \ln 2}{\ln 3 - 2\ln 2}$

$S = \left\lbrace \dfrac{\ln\left( 6 \right)}{\ln(\frac{3}{4})} \right\rbrace$

\question{}
$2 \times 3^x = 5 \Leftrightarrow 3^x = \dfrac{5}{2} \Leftrightarrow x = \dfrac{ \ln \frac{5}{2} }{\ln 3}$

$S = \left\lbrace \dfrac{ \ln \frac{5}{2} }{\ln 3} \right\rbrace$

\question{}
$5 \times 1.6^x + 3 = 2 \Leftrightarrow 1.6^x = - \dfrac{1}{5}$ or, $1.6^x$ est positif pour tout $x$, donc $S = \emptyset$.

\question{}
$3^{2x+1} = 4 \Leftrightarrow (2x+1) \ln(3) = \ln(4) \Leftrightarrow 2x+1 = \dfrac{\ln 4}{\ln 3} \Leftrightarrow 2x = \dfrac{\ln 4}{\ln 3} - 1 = \dfrac{\ln 4 - \ln 3}{\ln 3}$

On a donc $x = \dfrac{\ln 4 - \ln 3}{2 \ln 3}$

$S = \left\lbrace \dfrac{\ln 4 - \ln 3}{2 \ln 3} \right\rbrace$

\question{}
$2^{3x} - 1 = 4 \Leftrightarrow 2^{3x} = 5 \Leftrightarrow 3x = \dfrac{\ln 5}{\ln 2}$ d'où $x = \dfrac{\ln 5}{3 \ln 2}$

$S = \left\lbrace \dfrac{\ln 5}{3 \ln 2} \right\rbrace$

\question{}
$3^{x \ln 2} - 4 = 5 \Leftrightarrow 3^{x \ln 2} = 9 \Leftrightarrow x \ln 2 \ln 3 = \ln 9 \Leftrightarrow x = \dfrac{\ln 9}{\ln 2 \ln 3}$

Or, $\ln 9 = 2 \ln 3$, donc $x = \dfrac{2}{\ln 2}$

$S = \left\lbrace \dfrac{2}{\ln 2} \right\rbrace$

\question{}
$e^x = 4 \Leftrightarrow \ln e^x = \ln 4 \Leftrightarrow x = \ln 4$

$S = \left\lbrace \ln 4 \right\rbrace$

\question{}
$4 e^x - 2 = 2 \Leftrightarrow e^x = 1 \Leftrightarrow x = 0$

$S = \left\lbrace 0 \right\rbrace$

\question{}
$2 e^{3 \ln x} = 5 \Leftrightarrow e^{3 \ln x} = \dfrac{5}{2} \Leftrightarrow 3 \ln x = \ln \left( \dfrac{5}{2} \right)$

Soit: $\ln x = \dfrac{ \ln \left( \dfrac{5}{2} \right)}{3}$
D'où: $x = e^{\frac{ \ln \left( \frac{5}{2} \right)}{3}} = e^{\ln \left( \frac{5}{2} \right) \times \frac{1}{3}} = \left( \dfrac{5}{2} \right)^\frac{1}{3}$, que l'on peut aussi écrire $\sqrt[3]{\dfrac{5}{2}}$

$S = \left\lbrace \left( \dfrac{5}{2} \right)^\frac{1}{3} \right\rbrace$

\question{}
$3 e^{2x-1} + 4 = 16 \Leftrightarrow e^{2x-1} = 4 \Leftrightarrow 2x-1 = \ln 4 \Leftrightarrow x = \dfrac{\ln 4 + 1}{2}$

$S = \left\lbrace \dfrac{\ln 4 + 1}{2} \right\rbrace$

\question{}
$4 \times (2e^x - 1) = 2 \Leftrightarrow 2e^x - 1 = \dfrac{1}{2} \Leftrightarrow e^x = \dfrac{3}{4}$. Soit $x = \ln \left( \dfrac{3}{4} \right)$

$S = \left\lbrace \ln \left( \dfrac{3}{4} \right) \right\rbrace$

\question{}
$2e^x - 1 = -3 \Leftrightarrow e^x = -1$. Or, $e^x > 0 \forall x \in \mathbb{R}$\footnote{$\forall$ se lit (et se comprend) \og quel que soit \fg{}.}. Il n'y a donc pas de solution:

$S = \left\lbrace \emptyset \right\rbrace$

\question{}
$2^x = 3^{x+1} \Leftrightarrow x \ln 2 = (x+1) \ln 3$. On développe le membre de droite: $x \ln 2 = x \ln 3 + \ln 3 \Leftrightarrow x (\ln 2 - \ln 3) = \ln 3$. On a donc: $x = \dfrac{\ln 3}{\ln 2 - \ln 3}$

$S = \left\lbrace \dfrac{\ln 3}{\ln 2 - \ln 3} \right\rbrace$

\exo{}

\question{}
$\ln x = 2 \Leftrightarrow e^{\ln x} = e^2 \Leftrightarrow x = e^2$

$S = \left\lbrace e^2 \right\rbrace$

\question{}
$\ln x = -2 \Leftrightarrow e^{\ln x} = e^{-2} \Leftrightarrow x = e^{-2}$

$S = \left\lbrace e^{-2} \right\rbrace$

\question{}
$\ln 2x = 1 \Leftrightarrow 2x = e$ Soit $x = \dfrac{e}{2}$

$S = \left\lbrace \dfrac{e}{2} \right\rbrace$

\question{}
$2 \ln x = 1 \Leftrightarrow \ln x = \dfrac{1}{2} \Leftrightarrow x = e^{\frac{1}{2}}$

$S = \left\lbrace e^{\frac{1}{2}} \right\rbrace$

\question{}
$2 \ln 2x = 1 \Leftrightarrow \ln 2x = \dfrac{1}{2} \Leftrightarrow 2x = e^{\frac{1}{2}}$ Soit: $x = \dfrac{e^{\frac{1}{2}}}{2} = \dfrac{\sqrt{e}}{2}$

$S = \left\lbrace \dfrac{e^{\frac{1}{2}}}{2} \right\rbrace$

\question{}
$\ln(x+2) = 1 \Leftrightarrow x+2 = e \Leftrightarrow x = e-2$

$S = \left\lbrace e-2 \right\rbrace$

\question{}
$\ln(2x +1) = -2 \Leftrightarrow 2x+1 = e^{-2} \Leftrightarrow x = \dfrac{e^{-2} - 1}{2}$

$S = \left\lbrace \dfrac{e^{-2} - 1}{2} \right\rbrace$

\question{}
$8 \ln (5x-4) = -4 \Leftrightarrow \ln (5x-4) = -\frac{1}{2} \Leftrightarrow 5x-4 e^{-\frac{1}{2}} \Leftrightarrow x = \dfrac{e^{-\frac{1}{2}}+4}{5}$

$\mathcal{S} = \left\{\dfrac{e^{-\frac{1}{2}}+4}{5}\right\}$

\question{}
$\ln 2x = \ln(3x-1) \Leftrightarrow 2x = 3x-1 \Leftrightarrow x=1$

$\mathcal{S} = \left\{1\right\}$

\exo{Inéquations}

\question{}
$4 \times 5^x > 2 \Leftrightarrow 5^x > \dfrac{1}{2} \Leftrightarrow x \ln 5 > \ln \left( \dfrac{1}{2} \right) \Leftrightarrow x > \dfrac{\ln \left( \dfrac{1}{2} \right)}{\ln 5} = -\dfrac{\ln 2}{\ln 5}$

$S = \left] -\dfrac{\ln 2}{\ln 5} ; +\infty \right[$

\question{}
$3^x + 1 < 3 \Leftrightarrow 3^x < 2 \Leftrightarrow x \ln 3 < \ln 2 \Leftrightarrow x < \dfrac{\ln 2}{\ln 3}$

$S = \left] -\infty ; \dfrac{\ln 2}{\ln 3} \right[$

\question{}
$0.2^x > 4 \Leftrightarrow x \ln 0.2 > \ln 4 \Leftrightarrow x < \dfrac{\ln 4}{\ln 0.2}$. Ici, l'inégalité change de sens car $\ln 0.2 < 0$.

$S = \left] - \infty ; \dfrac{\ln 4}{\ln 0.2} \right[$

\question{}
$ 0.5^x < 1 \Leftrightarrow x \ln 0.5 < \ln 1 \Leftrightarrow x \ln 0.5 < 0 \Leftrightarrow x > 0$. Ici également, l'inégalité change de sens car $\ln 0.5 < 0$.

$S = \left] 0 ; +\infty \right[$

\question{}
$3 \times \ln x > 2 \Leftrightarrow \ln x > \dfrac{2}{3} \Leftrightarrow x > e^{\frac{2}{3}} \Leftrightarrow $

$S = \left] e^{\frac{2}{3}} ; +\infty \right[$

\exo{Mise sous forme exponentielle}
À noter que ce type d'expression peut revenir assez souvent car elles sont des applications dans de nombreux domaines, comme l'activité d'un matériau nucléaire, l'élimination d'une molécule par le corps et bien d'autres...

On peut écrire $\ln y = -0.5x + 3$, ce qui équivaut à $e^{\ln y} = e^{-0.5x + 3}$ soit $y = e^{-0.5x + 3}$.

On a donc (propriété de l'exponentielle): $y = e^{-0.5x} \times e^{3} \approx 20 \times e^{-0.5x}$.

On a bien mis $y$ sous la forme $k \cdot e^{- \lambda x}$, avec $k = 20$ et $\lambda = 0.5$.

\exo{}

$z = \ln(y-3) = 2x+1$. 

On a $y - 3 = e^{2x+1} \Leftrightarrow y = e^{2x+1} +3 = e \times e^{2x} + 3$. On a donc $y = A \cdot e^{Bx} + C$ avec $A = 2.7$, $B = 2$ et $C = 3$.

\exo{}

$z = \ln \left( \dfrac{y-2}{x-5} \right)$ et $z = -3x + 2$. 

On a donc $\ln \left( \dfrac{y-2}{x-5} \right) = -3x + 2$.

On cherche $y$, il s'agit donc d'une résolution d'une équation à une inconnue.

L'équation ci-dessus équivaut à:

$\dfrac{y-2}{x-5} = e^{-3x + 2} \Leftrightarrow y-2 = (x-5)\cdot e^{-3x + 2}$.

On a donc $y = (x-5)\cdot e^{-3x + 2}+2 = (x-5)\cdot e^{-3x}\cdot e^2+2 \approx 7.4\cdot(x-5)\cdot e^{-3x} + 2$.

On a donc bien $y = A\cdot(x-5)\cdot e^{Bx} + C$ avec $A = 7.4$, $B = -3$ et $C = 2$.

\exo{}

$z = e^y = 5x$

$e^y = 5x \Leftrightarrow y = \ln 5x \Leftrightarrow y = \ln 5 + \ln x$. On a donc $y = \ln x + k$ avec $k = \ln 5$.

\exo{}

$z = e^{5y} = 3x^2$. 

D'où: $5y = \ln(3x^2) \Leftrightarrow 5y = 2 \ln x + \ln 3 \Leftrightarrow y = \dfrac{2}{5} \ln x + \dfrac{\ln 3}{5}$. On a donc bien $y = A \cdot \ln x + B$ avec $A = \dfrac{2}{5}$ et $B \approx 0.22$.

\setcounter{exos}{1}

\exo{Équations}

\question{}
$2^x = 9 \Leftrightarrow x \ln 2 = \ln 9 \Leftrightarrow x = \dfrac{\ln 9}{\ln 2}$

$S = \left\lbrace \dfrac{\ln 9}{\ln 2} \right\rbrace$

\question{}
$5 \times 3^x = 8 \Leftrightarrow 3^x = \dfrac{8}{5} \Leftrightarrow x \ln 3 = \ln \dfrac{8}{5} \Leftrightarrow x = \dfrac{\ln 8 - \ln 5}{\ln 3}$

$S = \left\lbrace \dfrac{\ln 8 - \ln 5}{\ln 3} \right\rbrace$

\question{}
$2 \times 5^x + 6 = 8 \Leftrightarrow 5^x = 1$ soit $x = 0$

$S = \left\lbrace 0 \right\rbrace$

\question{}
$4 e^x - 5 = 7 \Leftrightarrow e^x = 3$ soit $x = \ln 3$

$S = \left\lbrace \ln 3 \right\rbrace$

\question{}
$5 \times e^x = 15 \times 2^x \Leftrightarrow e^x = 3 \times 2^x$

On prend le logarithme des deux côtés de l'équation: $x = \ln 3 + x \ln 2 \Leftrightarrow x - x \ln 2 = \ln 3$ soit $x \times (1-\ln 2) = \ln 3$. On a donc finalement $x = \dfrac{\ln 3}{1-\ln 2}$.

$S = \left\lbrace \dfrac{\ln 3}{1-\ln 2} \right\rbrace$

\question{}
$2 \times e^{2x-1} + 3 = 7 \Leftrightarrow e^{2x-1} = 2 \Leftrightarrow 2x-1 = \ln 2 \Leftrightarrow x = \dfrac{\ln 2 + 1}{2}$

$S = \left\lbrace \dfrac{\ln 2 + 1}{2} \right\rbrace$

\exo{Encore des équations}

\question{}
$\ln x = -4 \Leftrightarrow x = e^{-4}$

$S = \left\lbrace e^{-4} \right\rbrace$

\question{}
$3 \ln x + 4  = 5 \Leftrightarrow \ln x = \dfrac{1}{3} \Leftrightarrow x = e^{\frac{1}{3}}$

$S = \left\lbrace e^{\frac{1}{3}} \right\rbrace$

\question{}
$3 \ln(2x-5)+5 = 4 \Leftrightarrow \ln(2x-5) = -\dfrac{1}{3} \Leftrightarrow 2x-5 = e^{-\frac{1}{3}} \Leftrightarrow x = \dfrac{e^{-\frac{1}{3}} + 5}{2}$. On remarque que $e^{-\frac{1}{3}} = \dfrac{1}{e^{\frac{1}{3}}} = \dfrac{1}{\sqrt[3]{e}}$. Mais la première notation peut parfaitement être laissée en l'état.

$S = \left\lbrace \dfrac{e^{-\frac{1}{3}} + 5}{2} \right\rbrace$

\exo{Inéquations}

\question{}
$4^x > 5 \Leftrightarrow x \ln 4 > \ln 5 \Leftrightarrow x > \dfrac{\ln 5}{\ln 4}$. Le sens de l'inégalité est inchangé car $\ln 4$ est positif.

$S = \left] \dfrac{\ln 5}{\ln 4} ; +\infty \right[$

\question{}
$\left( \dfrac{1}{3} \right)^x \geqslant 9 \Leftrightarrow x \ln \dfrac{1}{3} \geqslant \ln 9$. On remarque que $9 = 3^2$, donc que $\ln 9 = \ln 3^2 = 2 \ln 3$. De même, $\ln \dfrac{1}{3} = -\ln 3$. L'inéquation devient donc: $-x \ln 3 \geqslant 2 \ln 3$. $\ln 3$ est strictement positif, on peut donc simplifier l'inéquation par $\ln 3$ sans qu'elle ne change de sens. 

On a donc $-x \geqslant 2$. On multiplie des deux côtés par $-1$, l'inégalité change donc de sens: $x \leqslant -2$

$S = \left] -\infty ; -2 \right]$

\exo{}
$y = \ln C$ et $y= -0.5x +3.2$

\question{}
On a donc $\ln C = -0.5x +3.2$. En prenant l'exponentielle de chaque membre, on obtient: $C = e^y = e^{-0.5x +3.2} = e^{-0.5x} \times e^{3.2} \approx 25 \times e^{-0.5x}$.

On a donc bien $C$ qui peut s'écrire de la forme $C = \alpha e^{-\beta x}$ avec $\alpha = 25$ et $\beta = 0.5$.

\question{}
Pour $x = 5$, $C = 25 \times e^{-0.5 \times 5} \approx 2.05$

\exo{}
$y = \ln(h+3) $ et $y = -x +4$.

\question{}
On a l'équation $\ln(h+3) = -x +4$. 

Elle est équivalente à $h+3 = e^{-x +4} \Leftrightarrow h = e^{-x +4} - 3 = e^{-x} \times e^4 - 3$. 

On a donc $h \approx 55 \times e^{-x} - 3$. On a bien écrit $h$ sous la forme $h = \alpha e^{\beta x} + \gamma$, avec $\alpha = 55$, $\beta = -1$ et $\gamma = -3$.

\question{}
Pour $x = 6$, $h = 55 \times e^{-6} - 3 = -2.86$.

\exo{Simplifier les écritures:}
On rappelle pour cet exercice que $\ln e = 1$.

\question{}
$A = \ln e^4 = 4$ car l'exponentielle et le logarithme népérien sont des fonctions réciproques.

\question{}
$B = \ln \dfrac{1}{e} = -\ln e = -1$ 

\question{}
$C = \ln e^3 - \ln e^2 = 3 \ln e - 2 \ln e = 3-2 = 1$ 

\question{}
$D = \ln \dfrac{1}{e^2} = -\ln e^2 = -2 \ln e = -2$

\question{}
$E = \ln(e + 1) + \ln(e - 1) - \ln(e^2- 1) = \ln \left( (e + 1)(e - 1) \right) - \ln(e^2- 1)$. Il y a une identité remarquable:

$E = \ln(e^2- 1) - \ln(e^2- 1)$

On obtient donc $E = 0$.

\exo{Équation}
$\ln x = \dfrac{\ln(3x-2)}{2} = \dfrac{1}{2} \ln(3x-2)$.

\littlestar{Ensemble de définition}
pour que les membres de cette équation soient calculables, on doit avoir: $x > 0$ d'une part et $3x - 2 > 0 \Leftrightarrow x > \frac{2}{3}$ d'autre part. On doit donc avoir:

$x \in \left] 0 ; +\infty \right[ \cap \left] \dfrac{2}{3} ; +\infty \right[$, donc $x \in \left] \dfrac{2}{3} ; +\infty \right[$. 

On peut aussi écrire $x > \frac{2}{3}$.

On a donc: $2\ln x = \ln(3x-2)$ soit $\ln x^2 = \ln(3x-2)$. Or, si les logarithmes népériens de deux nombres sont égaux, les deux nombres sont égaux. L'équation équivaut donc, pour $x \in \left] \dfrac{2}{3} ; +\infty \right[$ à:

$x^2 = 3x-2 \Leftrightarrow x^2 - 3x + 2 = 0$. On reconnaît ici une équation du 2\textsuperscript{nd} degré:

$x^2 - 3x + 2 = 0$ dont on sait retrouver les solutions.

On rappelle que $\Delta = b^2 - 4ac$. Donc ici:

$\Delta = 9 - 4 \times 1 \times 2 = 1$. $\Delta > 0$, il y a donc 2 solutions. $\sqrt{\Delta} = 1$:

$x_{1,2} = \dfrac{-b \pm \sqrt{\Delta}}{2a}$. Ici:

$$x_1 = \dfrac{3-1}{2} = 1$$

et 

$$x_2 = \dfrac{3+1}{2} = 2$$

Les 2 solutions trouvées sont supérieures à $\dfrac{2}{3}$. Elles doivent donc être retenues toutes les deux.

$S = \left\lbrace 1 ; 2 \right\rbrace$

\exo{}
Soit $f$ définie sur $\mathbb{R}$ par\footnote{Pour les curieux, cette fonction s'appelle un \emph{cosinus hyperbolique}.} $f(x) = \dfrac{e^x + e^{-x}}{2}$. 

\question{}
Une exponentielle est strictement positive sur $\mathbb{R}$, soit $e^x > 0$ et $e^{-x} > 0$ pour tout $x \in \mathbb{R}$. Donc $e^x + e^{-x} > 0$. Diviser par 2 ne change pas le signe, donc $\dfrac{e^x + e^{-x}}{2} > 0$ soit $f(x) > 0$ pour tout $x \in \mathbb{R}$.

\question{}
On prend le membre de droite de l'égalité et on le développe. Dans un premier temps, on développe l'identité remarquable\footnote{On rappelle que $\left(e^x\right)^2 = \left(e^2\right)^x = e^{2x}$}.

$e^{-x}(e^x-1)^2 = e^{-x}(e^{2x} - 2e^x + 1) = e^{2x-x} - 2e^{x-x} + e^{-x}$

L'expression vaut donc bien: $e^x - 2 + e^{-x} = e^x + e^{-x} - 2$. 

On remarque que $(e^x-1)^2$ est un carré, donc $(e^x-1)^2 \geqslant 0, \forall x \in \mathbb{R}$. On sait également que $e^{-x} > 0, \forall x \in \mathbb{R}$. Donc $e^{-x}(e^x-1)^2 \geqslant 0, \forall x \in \mathbb{R}$.

Grâce à la question précédente, on déduit que: 

$e^x + e^{-x} - 2 \geqslant 0$

$\Leftrightarrow e^x + e^{-x} \geqslant 2$ 

$\Leftrightarrow \dfrac{e^x + e^{-x}}{2} \geqslant 1$

Soit $f(x) \geqslant 1$.

\trait

\end{document}
