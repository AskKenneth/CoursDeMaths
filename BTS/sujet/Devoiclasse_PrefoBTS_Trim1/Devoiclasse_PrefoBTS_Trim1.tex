\documentclass[a4paper,12pt]{scrartcl}
\usepackage[utf8x]{inputenc}
\usepackage[T1]{fontenc} % avec T1 comme option  d'encodage c'est ben mieux, surtout pour taper du français.
%\usepackage{lmodern,textcomp} % fortement conseillé pour les pdf. On peut mettre autre chose : kpfonts, fourier,...
\usepackage[french]{babel} %Sans ça les guillemets, amarchpo
\usepackage{amsmath}
\usepackage{multicol}
\usepackage{amssymb}
\usepackage{tkz-tab}
\usepackage{exercice_sheet}



%\trait
%\section*{}
%\exo{}
%\question{}
%\subquestion{}

\date{}


% Title Page
\title{Devoir en classe, préformation BTS}

\author{\rotatebox{10}{\textsc{Mathématiques}} \\ 60 minutes}

\begin{document}

\maketitle

{\Large Nom:} 
\hspace{60mm}
{\Large Prénom:}
\vspace{6mm}

Note : la calculatrice est autorisée. Donc sauf mention contraire, tout résultat donné sans la moindre étape de calcul ne sera pas pris en compte.

On rappelle que résoudre une équation, c'est donner un ensemble. Par exemple: $\mathcal{S} = \left\lbrace 3 \right\rbrace$ ou encore $\mathcal{S} = \left\lbrace \dfrac{1}{2} ; 4 \right\rbrace$, etc.

\exo{Équations du 1\textsuperscript{er} degré à une inconnue}

Résoudre en donnant les valeurs \emph{exactes}:

\question{} $5x+2 = 12$

\lignes{2}

\question{} $24x+12 = 6$

\lignes{2}

\question{} $7x+4 = 9$

\lignes{2}


\exo{Développements}

Développer les expressions suivantes:

\question{} $(x+1)(x-2)$

\lignes{2}

\question{} $(x+4)^{2}$

\lignes{2}

\question{} $(2x-1)(2x+1)$

\lignes{2}

\question{} $2(3x+1)^{2}$

\lignes{2}

\exo{Factorisations}

Factoriser les expressions suivantes (sans passer par le calcul de $\Delta$):

\question{} $a^{2} - a$

\lignes{2}

\question{} $x^{2} + 2x + 1$

\lignes{2}

\question{} $5x^{2} + 10x + 5$

\lignes{2}

\question{} $x^{3} + x^{2}$

\lignes{2}

\exo{Racines carrées}

Simplifier\footnote{écrire avec des nombres les plus petits possible sous les radicaux $\sqrt{\mbox{ }}$, voire pas de radical lorsque c'est possible.} au maximum les écritures suivantes.

\question{} $\sqrt{300}$	

\lignes{2}

\question{} $\sqrt{675}$

\lignes{2}

\question{} $\sqrt{432}$

\lignes{2}

\question{} $\sqrt{\dfrac{16}{4}}$

\lignes{2}

\question{}  $\dfrac{\sqrt{a^{5}} + \sqrt{a^{3}}}{\sqrt{a}}$ avec $a \geqslant 0$.

\lignes{2}


\exo{Puissances}

Réduire les écritures suivantes.

\question{} $5^{12} \times 5^{9}$

\lignes{2}

\question{} $\dfrac{8^{16}}{8^{2}}$

\lignes{2}

\question{} $\dfrac{1}{12^{20}}$

\lignes{2}

\question{} $x^{a} x^{b}$

\lignes{2}

\exo{Fractions}

Écrire sous forme d'une seule fraction après avoir mis au même dénominateur, réduire puis simplifier lorsque c'est possible.

\question{} $\dfrac{1}{12} - \dfrac{1}{16}$

\lignes{2}

\question{} $\dfrac{1}{f} - \dfrac{1}{g}$

\lignes{2}

\question{} $\dfrac{1}{x-1} + 4$

\lignes{3}

\question{} $\dfrac{1}{x+3} + \dfrac{1}{x+1}$

\lignes{3}

\question{} $\dfrac{2x+1}{x+1} - \dfrac{x}{2x+2}$

\lignes{3}

\exo{Équations}

Note : il y a ici des équations du second degré. Ce sont des cas particuliers que l'on peut résoudre sans passer par le calcul de $\Delta$, qu'il ne faut pas utiliser ici.

\question{} $(x+1)^{2} = 0$

\lignes{2}

\question{} $(x-2)(x+3) = 0$

\lignes{2}

\question{} $4x^{2} + 16x + 16 = 0$

\lignes{2}

\question{} Résoudre, après avoir trouvé les valeurs interdites: $\dfrac{x+1}{x-3} = \dfrac{x+6}{x+1}$

\lignes{4}

\trait

\begin{center}
Fin.
\end{center}

\end{document}

