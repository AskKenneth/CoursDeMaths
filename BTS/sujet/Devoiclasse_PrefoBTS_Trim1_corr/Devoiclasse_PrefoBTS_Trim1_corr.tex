\documentclass[a4paper,12pt]{scrartcl}
\usepackage[utf8x]{inputenc}
\usepackage[T1]{fontenc} % avec T1 comme option  d'encodage c'est ben mieux, surtout pour taper du français.
%\usepackage{lmodern,textcomp} % fortement conseillé pour les pdf. On peut mettre autre chose : kpfonts, fourier,...
\usepackage[french]{babel} %Sans ça les guillemets, amarchpo
\usepackage{amsmath}
\usepackage{multicol}
\usepackage{amssymb}
\usepackage{tkz-tab}
\usepackage{exercice_sheet}



%\trait
%\section*{}
%\exo{}
%\question{}
%\subquestion{}

\date{}


% Title Page
\title{Devoir en classe, préformation BTS}

\author{\rotatebox{10}{\textsc{Mathématiques}} \\ 60 minutes}

\begin{document}

\maketitle

{\Large Nom:} 
\hspace{60mm}
{\Large Prénom:}
\vspace{6mm}

Note : la calculatrice est autorisée. Donc sauf mention contraire, tout résultat donné sans la moindre étape de calcul ne sera pas pris en compte.

On rappelle que résoudre une équation, c'est donner un ensemble. Par exemple: $\mathcal{S} = \left\lbrace 3 \right\rbrace$ ou encore $\mathcal{S} = \left\lbrace \dfrac{1}{2} ; 4 \right\rbrace$, etc.

\exo{Équations du 1\textsuperscript{er} degré à une inconnue}

Résoudre en donnant les valeurs \emph{exactes}:

\question{} $5x+2 = 12$

\Answer{$\mathcal{S} = \left\lbrace 5 \right\rbrace$}

\question{} $24x+12 = 6$

\Answer{$\mathcal{S} = \left\lbrace -\dfrac{1}{4} \right\rbrace$}

\question{} $7x+4 = 9$

\Answer{$\mathcal{S} = \left\lbrace \dfrac{7}{5} \right\rbrace$}

\exo{Développements}

Développer les expressions suivantes:

\question{} $(x+1)(x-2)$

\Answer{$x^{2} - x - 2$}

\question{} $(x+4)^{2}$

On reconnaît l'identité remarquable $(a+b)^2 = a^{2} + 2ab + b^{2}$ :

\Answer{$x^{2} + 8x + 16$}

\question{} $(2x-1)(2x+1)$

On reconnaît l'identité remarquable $(a+b)(a-b) = a^{2} - b^{2}$ :

\Answer{$4x^2 - 1$}

\question{} $2(3x+1)^{2}$

\Answer{$18x^2 + 12x + 3$}

\exo{Factorisations}

Factoriser les expressions suivantes (sans passer par le calcul de $\Delta$):

\question{} $a^{2} - a$

le facteur commun est $a$.

\Answer{$a(a-1)$}

\question{} $x^{2} + 2x + 1$

Identité remarquable:

\Answer{$(x+1)^{2}$}

\question{} $5x^{2} + 10x + 5$

On factorise par 5 

$5(x^2 + 2x + 1)$

Puis on reconnaît l'identité remarquable (la même que ci-dessus):

\Answer{$5(x+1)^{2}$}

\question{} $x^{3} + x^{2}$

Le facteur commun est $x^2$:

\Answer{$x^2(x+1)$}

\exo{Racines carrées}

Simplifier\footnote{écrire avec des nombres les plus petits possible sous les radicaux $\sqrt{\mbox{ }}$, voire pas de radical lorsque c'est possible.} au maximum les écritures suivantes.

\question{} $\sqrt{300}$	

\Answer{$10 \sqrt{3}$}

\question{} $\sqrt{675}$

\Answer{$15\sqrt{3}$}

\question{} $\sqrt{432}$

\Answer{$12\sqrt{3}$}

\question{} $\sqrt{\dfrac{16}{4}}$

\Answer{$2$}

\question{}  $\dfrac{\sqrt{a^{5}} + \sqrt{a^{3}}}{\sqrt{a}}$ avec $a \geqslant 0$.

\answer{$a^2 + a$}, ou encore :

\Answer{$a(a+1)$}


\exo{Puissances}

Réduire les écritures suivantes.

\question{} $5^{12} \times 5^{9}$

\Answer{$5^{21}$}

\question{} $\dfrac{8^{16}}{8^{2}}$

\Answer{$8^{14}$}

\question{} $\dfrac{1}{12^{20}}$

\Answer{$12^{-20}$}

\question{} $x^{a} x^{b}$

\Answer{$x^{a+b}$}

\exo{Fractions}

Écrire sous forme d'une seule fraction après avoir mis au même dénominateur, réduire puis simplifier lorsque c'est possible.

\question{} $\dfrac{1}{12} - \dfrac{1}{16}$

$= \dfrac{4}{48} - \dfrac{3}{48}$

\Answer{$\dfrac{1}{48}$}

\question{} $\dfrac{1}{f} - \dfrac{1}{g}$

\Answer{$\dfrac{g-f}{fg}$}

\question{} $\dfrac{1}{x-1} + 4$

\Answer{$\dfrac{4x-3}{x-1}$}

\question{} $\dfrac{1}{x+3} + \dfrac{1}{x+1}$

$$\dfrac{(x+1)+(x+3)}{(x+3)(x+1)}$$

\Answer{$\dfrac{2x+4}{(x+3)(x+1)}$}

\question{} $\dfrac{2x+1}{x+1} - \dfrac{x}{2x+2}$

Le plus petit dénominateur commun est $2x+2$. On \og amplifie \fg{}  donc la première fraction par 2:
$\dfrac{4x+2}{2x+2} - \dfrac{x}{2x+2}$

\Answer{$\dfrac{3x+2}{2x+2}$}

\exo{Équations}

Note : il y a ici des équations du second degré. Ce sont des cas particuliers que l'on peut résoudre sans passer par le calcul de $\Delta$, qu'il ne faut pas utiliser ici.

\question{} $(x+1)^{2} = 0$

$\Leftrightarrow x+1 = 0$

$\Leftrightarrow x = -1$

\Answer{$\mathcal{S} = \left\lbrace -1 \right\rbrace$}

\question{} $(x-2)(x+3) = 0$



\[
\left\{ 
\begin{array}{c}
x-2 = 0 \\ 
\mbox{ou} \\ 
x+3 = 0
\end{array}
\right. 
\]

\Answer{$\mathcal{S} = \left\lbrace -3 ; 2 \right\rbrace$}

\question{} $4x^{2} + 16x + 16 = 0$

$\Leftrightarrow x^{2} + 4x + 4 = 0$ (division par 4 des deux côtés)

$\Leftrightarrow (x+2)^{2} = 0$ (factorisation)

$\Leftrightarrow x+2 = 0$

\Answer{$\mathcal{S} = \left\lbrace -2 \right\rbrace$}

\question{} Résoudre, après avoir trouvé les valeurs interdites: $\dfrac{x+1}{x-3} = \dfrac{x+6}{x+1}$

\littlestar{Valeurs interdites}

$x-3 = 0 \Leftrightarrow x = 3$

$x+1 = 0 \Leftrightarrow x = -1$

Les valeurs interdites sont donc $-1$ et $3$.

\littlestar{Produit en croix}

$\dfrac{x+1}{x-3} = \dfrac{x+6}{x+1} \Leftrightarrow (x+1)^{2} = (x-3)(x+6)$ pour $x \notin \left\lbrace -1 ; 3 \right\rbrace$

$\Leftrightarrow x^{2} + 2x + 1 = x^{2} + 3x - 18$

$\Leftrightarrow x = 19$

\Answer{$\mathcal{S} = \left\lbrace 19 \right\rbrace$}

\trait

\begin{center}
Fin.
\end{center}

\end{document}

