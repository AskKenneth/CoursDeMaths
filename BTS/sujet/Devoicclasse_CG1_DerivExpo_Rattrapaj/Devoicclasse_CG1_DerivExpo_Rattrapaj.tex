\documentclass[a4paper,12pt]{scrartcl}
\usepackage[utf8x]{inputenc}
\usepackage[T1]{fontenc} % avec T1 comme option  d'encodage c'est ben mieux, surtout pour taper du français.
%\usepackage{lmodern,textcomp} % fortement conseillé pour les pdf. On peut mettre autre chose : kpfonts, fourier,...
\usepackage[french]{babel} %Sans ça les guillemets, amarchpo
\usepackage{amsmath}
\usepackage{multicol}
\usepackage{amssymb}
\usepackage{tkz-tab}
\usepackage{exercice_sheet}

%\trait
%\section*{}
%\exo{}
%\question{}
%\subquestion{}

\date{}


% Title Page
\title{Devoir maison, CG-4}

\author{\rotatebox{0}{\textsc{Mathématiques}}}

\begin{document}

\maketitle

\section*{Exponentielle et logarithme}

\exo{Résoudre les équations suivantes:}

\question{}
$x^{2} = 0$

\question{}
$x^{2} - 2 = 0$

\question{}
$e^{2x} = 1$

\question{}
$\dfrac{1}{2^{x}} = 8$

\question{}
$2x^2 -12 +10= 0$

\exo{Résoudre les inéquations suivantes:}

\question{}
$e^{-x} < 3$

\question{}
$\ln (3x+1) > 4$

\section*{Dérivation}

\exo{Calculer les dérivées des fonctions suivantes. Les simplifier lorsque c'est possible:}

\question{}
$f_1(x) = 2x^3 + 1$

\question{}
$f_2(x) = e^{x+2}$

\question{}
$f_3(x) = 3x^2 + 2x - 1$

\question{}
$f_4(x) = \ln(2x^3 + 1)$

\exo{}
Soit la fonction $f$ telle que $f(x) = e^{-2x - 1}$. 

\question{Montrer que $f'(x) = -2e^{-2x - 1}$.}

\question{Donner le signe de $f'(x)$ sur $\mathbb{R}$.}

\question{En déduire les variations de $f$ sur $\mathbb{R}$.}


\trait

\begin{center}
Fin.
\end{center}

\end{document}
