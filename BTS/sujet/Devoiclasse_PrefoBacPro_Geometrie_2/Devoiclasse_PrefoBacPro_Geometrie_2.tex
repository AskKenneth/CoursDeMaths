\documentclass[a4paper,12pt]{scrartcl}
\usepackage[utf8x]{inputenc}
\usepackage[T1]{fontenc} % avec T1 comme option  d'encodage c'est ben mieux, surtout pour taper du français.
%\usepackage{lmodern,textcomp} % fortement conseillé pour les pdf. On peut mettre autre chose : kpfonts, fourier,...
\usepackage[french]{babel} %Sans ça les guillemets, amarchpo
\usepackage{amsmath}
\usepackage{multicol}
\usepackage{amssymb}
\usepackage{tkz-tab}
\usepackage{exercice_sheet} 



%\trait
%\section*{}
%\exo{}
%\question{}
%\subquestion{}

\date{}


% Title Page
\title{Devoir en classe, préformation Bac Pro Optique}

\author{\rotatebox{10}{\textsc{Géométrie}} \\ 40 minutes}

\begin{document}

\maketitle

{\Large Nom:} 
\hspace{60mm}
{\Large Prénom:}
\vspace{6mm}

Note : la calculatrice est autorisée. Donc sauf mention contraire, tout résultat donné sans la moindre étape de calcul ne sera pas pris en compte...

\exo{Rectangles}
On considère un rectangle de 9cm de large et 14cm de long.

\question{Donner le périmètre $\mathcal{P}_1$ et l'aire $\mathcal{A}_1$ de ce rectangle.}
\lignes{1}

\question{Ce rectangle subit un agrandissement de facteur $k = 2$. Quelles sont ses dimensions?} 

\lignes{2}

\question{Donner le périmètre $\mathcal{P}_2$ et l'aire $\mathcal{A}_2$ de ce nouveau rectangle.} 

\lignes{2}

\exo{Triangles}

Le triangle $ABC$ a pour base $[AB]$, de longueur 16cm et pour hauteur $[HC]$, de longueur 5cm. 

\lignes{2}

\exo{Tracer \underline{au compas} la parallèle à la droite passant par le point A, en laissant les traits de construction.}

\begin{center}
\includegraphics[width=0.8\linewidth]{pics/DroitePoint.pdf}
\end{center}

\exo{Tracer \underline{au compas} la perpendiculaire à la droite passant par le point A, en laissant les traits de construction.}

\begin{center}
\includegraphics[width=0.8\linewidth]{pics/DroitePoint.pdf}
\end{center}



%\trait

%\begin{center}
%Fin.
%\end{center}

\end{document}

