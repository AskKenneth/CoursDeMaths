\documentclass[a4paper,12pt]{scrartcl}
\usepackage[utf8x]{inputenc}
\usepackage[T1]{fontenc} % avec T1 comme option  d'encodage c'est ben mieux, surtout pour taper du français.
%\usepackage{lmodern,textcomp} % fortement conseillé pour les pdf. On peut mettre autre chose : kpfonts, fourier,...
\usepackage[french]{babel} %Sans ça les guillemets, amarchpo
\usepackage{amsmath}
\usepackage{multicol}
\usepackage{amssymb}
\usepackage{tkz-tab}
\usepackage{exercice_sheet}

%\trait
%\section*{}
%\exo{}
%\question{}
%\subquestion{}

\date{}

\newcommand{\classe}{OL2}
\newcommand{\typedevoir}{Devoir en classe} 
\newcommand{\duree}{60 minutes}

\renewcommand{\hascorrection}{0}


% Title Page
\title{\typedevoir{} -- \classe{}\writecorrword{}} 

\author{Durée : \duree{}}

\begin{document} 

\maketitle 

Note: les questions \textbf{ne} sont \textbf{pas} par ordre croissant de difficulté. 

Calculatrice autorisée, smartphone ou tout objet connecté interdit. Sujet à rendre avec la copie, ne restez pas bloqué$\cdot$e sur une question trop longtemps au détriment du reste. 

%Un exercice avec son corrigé à inclure dans un fichier principal.
\exo{Statistiques à une variable}

Une entreprise fabrique des conserves alimentaires dont l'étiquette annonce une masse de 250 grammes.
Les masses obtenues pour un échantillon de 500 conserves prises au hasard sont données dans le tableau suivant: 

\begin{table}[h]
\centering
\caption{Masses des 500 conserves.}
\label{table_conserves}
\begin{tabular}{|l|l|l|l|l|l|}
\hline
\textbf{Masse (en g) $x_{i}$}        & {[}235 ; 240{[} & {[}240 ; 245{[} & {[}245 ; 250{[} & {[}250 ; 255{[} & {[}255 ; 260{[} \\ \hline
\textbf{Nb. de conserves} & 33              & 67              & 217             & 132             & 51              \\ \hline
\end{tabular}
\end{table}

\question{}
À l'aide de la calculatrice, calculer, en utilisant les milieux des classes, la masse moyenne $\overline{x}$ ainsi que l'écart type $\sigma_x$ des conserves de cet échantillon.

On fournira les valeurs en grammes arrondies au dixième.

\correction{
    $\overline{x} = \dfrac{237.5 \times 33 + \ldots + 257.5 \times 51}{500} = 248.5$g.

    $\sigma_x = 5.1$g.
}

\question{}
Donner les quartiles $Q_1$ et $Q_3$ et calculer l'équart interquartile.

\correction{
    $Q_1 = 247.5$g.

    $Q_3 = 252.5$g.

    L'écart interquartile vaut $Q_3 - Q_1 = 5$g.
}

\question{}
Calculer le pourcentage des conserves alimentaires ayant une masse comprise entre 240g et 255 grammes. 

\correction{
    Il y a 416 conserves qui dont la masse est comprise entre 240g et 255g. Cela représente $\dfrac{416}{500} = 0.832 = 83.2\%$. 
}


%Un exercice avec son corrigé à inclure dans un fichier principal.
\exo{Statistiques à une variable}

Une entreprise fabrique des conserves alimentaires dont l'étiquette annonce une masse de 250 grammes.
Les masses obtenues pour un échantillon de 500 conserves prises au hasard sont données dans le tableau suivant: 

\begin{table}[h]
\centering
\caption{Masses des 500 conserves.}
\label{table_conserves}
\begin{tabular}{|l|l|l|l|l|l|}
\hline
\textbf{Masse (en g) $x_{i}$}        & {[}235 ; 240{[} & {[}240 ; 245{[} & {[}245 ; 250{[} & {[}250 ; 255{[} & {[}255 ; 260{[} \\ \hline
\textbf{Nb. de conserves} & 33              & 67              & 217             & 132             & 51              \\ \hline
\end{tabular}
\end{table}

\question{}
À l'aide de la calculatrice, calculer, en utilisant les milieux des classes, la masse moyenne $\overline{x}$ ainsi que l'écart type $\sigma_x$ des conserves de cet échantillon.

On fournira les valeurs en grammes arrondies au dixième.

\correction{
    $\overline{x} = \dfrac{237.5 \times 33 + \ldots + 257.5 \times 51}{500} = 248.5$g.

    $\sigma_x = 5.1$g.
}

\question{}
Donner les quartiles $Q_1$ et $Q_3$ et calculer l'équart interquartile.

\correction{
    $Q_1 = 247.5$g.

    $Q_3 = 252.5$g.

    L'écart interquartile vaut $Q_3 - Q_1 = 5$g.
}

\question{}
Calculer le pourcentage des conserves alimentaires ayant une masse comprise entre 240g et 255 grammes. 

\correction{
    Il y a 416 conserves qui dont la masse est comprise entre 240g et 255g. Cela représente $\dfrac{416}{500} = 0.832 = 83.2\%$. 
}
 

%Un exercice avec son corrigé à inclure dans un fichier principal.
\exo{}

On considère la suite $(u_n)$ définie pour tout entier naturel $n$ par:

$$\left\lbrace
\begin{array}{ll}
   u_0 = 5 \\
   u_{n+1} = 2 u_n + 1 \\
\end{array}
\right.$$

% $u_0 = 13$ et pour tout entier naturel $n$, $u_{n+1} = 3 u_n - 14$.

\question{}
Calculer $u_1$, $u_2$ et $u_3$.

\correction{
\begin{itemize}
 \item $u_0 = 5$
 \item $u_1 = 2u_0 + 1 = 11$
 \item $u_2 = 2u_1 + 1 = 23$
 \item $u_3 = 2u_2 + 1 = 47$
\end{itemize}
}

\question{}
On considère la suite $(v_n)$ définie pour tout entier naturel $n$, par $v_n = u_n + 1$.


\subquestion{}
Après avoir exprimé $v_{n+1}$ en fonction de $v_n$, montrer que la suite $(v_n)$ est une suite géométrique dont on donnera le premier terme et la raison.

\correction{%Corrigé à faire
Calculons $\frac{v_{n+1}}{v_n}$.

$\forall n \in \mathbb{N}, v_n = u_n + 1$, d'où $v_{n+1} = u_{n+1} + 1$

Donc $\frac{v_{n+1}}{v_n} = \frac{u_{n+1} + 1}{v_n}$

Or d'après l'énoncé, $u_{n+1} = 2 u_n + 1$

D'où $\frac{v_{n+1}}{v_n} = \frac{2 u_n 1+1+}{v_n} = \frac{2 u_n + 2}{v_n}$

On déduit de l'énoncé que $u_n = v_n -1$. 

Donc $\frac{v_{n+1}}{v_n} = \frac{2 (v_n - 1) +2}{v_n}$

Finalement:

\Answer{$\dfrac{v_{n+1}}{v_n} = 2$}

$(v_n)$ est donc géométrique de raison 2 et de premier terme $v_0 = u_0 +1 = 6$.
}

\subquestion{}
En déduire l'expression de $v_n$ en fonction de $n$.

\correction{
On en déduit son terme général: $v_n = 6 \times 2^n$
}

\subquestion{}
En déduire que $u_n = 6 \times 2^n - 1$.

\correction{
$u_n = v_n -1 = 6 \times 2^n - 1$
}

\subquestion{}
À partir de quelle valeur de $n$ a-t-on $u_n > 200$?

\correction{
$u_n > 200 \Leftrightarrow 6 \times 2^{n}-1 >200$

soit: $\Leftrightarrow 2^{n} > 33.5$

$\Leftrightarrow \ln(2^{n}) > \ln 33.5$

$\Leftrightarrow n \ln 2 > \ln 33.5$

Or, $\ln 2$ est strictement positif, donc l'inéquation équivaut à:

$n > \dfrac{\ln 33.5}{\ln 2} \approx 5.07$

On a donc $u_n > 200$ pour $n \geqslant 6$ 
}

% \subquestion{}
% Déterminer $\lim u_n$ et $\lim v_n$.

% \correction{
% $(v_n)$ est géométrique de raison $q = 3 > 1$ et de premier terme $v_0 = 6 > 0$. 
% }

% \begin{equation*}
% \lim_{n \to +\infty} v_n = + \infty 
% \end{equation*}

% De plus, $v_n > u_n$ quel que soit $n \in \mathbb{N}$, d'où: 

% \begin{equation*}
 % \lim_{n \to +\infty} u_n = + \infty 
% \end{equation*}
% }

% \question{}
% Soient les suites $U_n = \frac{1}{u_n}$ et $V_n = \frac{1}{v_n}$. Déterminer $\lim U_n$ et $\lim V_n$.

% \correction{
 % D'après les résultats précédents, $\lim U_n$ est de la forme $\frac{1}{+\infty}$ d'où $\lim U_n = 0^+$.

 % D'après les résultats précédents, $\lim V_n$ est de la forme $\frac{1}{+\infty}$ d'où $\lim V_n = 0^+$.
% }



 
\end{document}
