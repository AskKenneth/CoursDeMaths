%Un exercice avec son corrigé à inclure dans un fichier principal.

\exo{Dans chaque cas, résoudre l'équation $P(x) = 0$ et factoriser $P(x)$ lorsque c'est possible.}

\question{}
$P(x) = -3x^2	+2x 	+5$

\correction{
$P$ est un polynôme du second degré. On calcule donc $\Delta = b^2 - 4ac$:

$\Delta = 2^2 - 4 \times (-3) \times 5 = 64 > 0$ et $\sqrt{\Delta} = 8$

L'équation a donc 2 solutions: $x_{1,2} = \dfrac{-b \pm \sqrt{\Delta}}{2a}$

$x_1 = \frac{-2-8}{-6} = \frac{5}{3}$

$x_2 = \frac{-2+8}{-6} = -1$

On a donc: \Answer{$S = \left\lbrace -1 ; \dfrac{5}{3} \right\rbrace$}
}

\question{}
$P(x) = 3x^2	-1x	-2$

\correction{
$P$ est un polynôme du second degré. On calcule donc $\Delta = b^2 - 4ac$:

$\Delta = (-1)^2 - 4 \times 3 \times (-2) = 25 > 0$ et $\sqrt{\Delta} = 5$

L'équation a donc 2 solutions:  $x_{1,2} = \dfrac{-b \pm \sqrt{\Delta}}{2a}$

$x_1 = \frac{1+5}{6} = 1$

$x_2 = \frac{1-5}{6} = -\frac{2}{3}$

On a donc: \Answer{$S = \left\lbrace -\dfrac{2}{3} ; 1 \right\rbrace$}
}

