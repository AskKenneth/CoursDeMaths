\exo{Résoudre l'équation $Q(x) = 0$}

\question{}
$Q(x) = \frac{2}{x+2} - x + 4$

\correction{
Valeurs interdites: $x+2 = 0 \Leftrightarrow x=-2$

Mise au même dénominateur: 

$Q(x) = \dfrac{-x^2+2x+10}{x+2}$

Pour $x \neq -2$, $Q(x) = 0 \Leftrightarrow -x^2+2x+10 = 0$.

Il s'agit d'une équation polynomiale du 2nd degré:

$\Delta = 44 > 0$ donc 2 solutions.

$\sqrt{\Delta} = 2 \sqrt{11}$.

$x_1 = \frac{-2 - 2 \sqrt{11}}{-2} = 1 + \sqrt{11}$

$x_2 = \frac{-2 + 2 \sqrt{11}}{-2} = 1 - \sqrt{11}$

On a donc \Answer{$\mathcal{S} = \left\lbrace 1 - \sqrt{11} ; 1 + \sqrt{11} \right\rbrace$}
}

\question{}
$Q(x) = \frac{1}{x-2} + \frac{2}{x-3} - 5$

\correction{
Valeurs interdites: $x-2 = 0 \Leftrightarrow x=2$ et $x-3 = 0 \Leftrightarrow x=3$.

Mise au même dénominateur: 

$\dfrac{-5 x^{2}+28 x-37}{(x-2)(x-3)}$

Pour $x \in \mathbb{R} - \{2;3\}$, $Q(x) = 0 \Leftrightarrow -5 x^{2}+28 x-37 = 0$.

Il s'agit d'une équation polynomiale du 2nd degré:


$\Delta = 44 > 0$ donc 2 solutions.

$\sqrt{\Delta} = 2 \sqrt{11}$.

$x_1 = \frac{-28 - 2 \sqrt{11}}{-10} = \frac{14 + \sqrt{11}}{5}$

$x_2 = \frac{-28 + 2 \sqrt{11}}{-10} = \frac{14 - \sqrt{11}}{5}$

On a donc \Answer{$\mathcal{S} = \left\lbrace \frac{14 - \sqrt{11}}{5} ; \frac{14 + \sqrt{11}}{5} \right\rbrace$}
}
