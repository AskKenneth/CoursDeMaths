%Un exercice avec son corrigé à inclure dans un fichier principal.

\exo{Résoudre dans $\mathbb{R}$ les équations suivantes}

\question{}
$x^2 + 4 = 8x$

\correction{
L'équation équivaut à $x^2 - 8x + 4 = 0$. Il s'agit d'une équation polynomiale du 2nd degré.

$\Delta = 48 > 0$. L'équation a donc 2 solutions.

$\sqrt{\Delta} = 4\sqrt{3}$

$x_1 = \dfrac{8 + 4\sqrt{3}}{2} = 4 + 2\sqrt{3}$
 
$x_2 = \dfrac{8 - 4\sqrt{3}}{2} = 4 - 2\sqrt{3}$

\Answer{$\mathcal{S} = \left\lbrace 4 - 2\sqrt{3} ; 4 + 2\sqrt{3} \right\rbrace$}
}

\question{}
$\dfrac{x+1}{x-1} = 1$

\correction{
Il y a une écriture fractionnaire où $x$ apparaît au dénominateur. On cherche donc la ou les valeurs interdites:

$x-1=0 \Leftrightarrow x = 1$

1 est l'unique valeur interdite. On résout donc sur $\mathbb{R} - \{1\}$

Pour $x \neq 1$, l'équation équivaut à $x+1 = x-1$

$\Leftrightarrow 0=2$.

On a donc: \Answer{$\mathcal{S} = \emptyset $}
}

\question{}
$\dfrac{x^2 + 4x + 2}{x^2 + 6x + 5} = 0$

\correction{
L'écriture est fractionnaire où $x$ apparaît au dénominateur. On cherche donc la ou les valeurs interdites:

\begin{equation}
x^2 + 6x + 5 = 0
\label{eq:gnieh}
\end{equation}

Il s'agit d'une équation du second degré: 

$\Delta = 6^2 - 4 \times 5 = 16 > 0$ donc l'équation a 2 solutions.

$\sqrt{\Delta} = 4$

$x_1 = \dfrac{-6 + 4}{2} = -1$

$x_2 = \dfrac{-6 - 4}{2} = -5$

L'ensemble des solutions de l'équation \ref{eq:gnieh} est $\{-5;-1\}$.

Il s'agit aussi de l'ensemble des valeurs interdites de l'équation de départ, on résout donc sur $\mathbb{R} - \{-5;-1\}$. 

$\dfrac{x^2 + 4x + 2}{x^2 + 6x + 5} = 0 \Leftrightarrow x^2 + 4x + 2 = 0$

Il s'agit d'une équation du second degré: 

$\Delta = 4^2 - 4 \times 2 = 8 > 0$ donc l'équation a 2 solutions.

$\sqrt{\Delta} = 2 \sqrt{2}$

$x_1 = \dfrac{-4 - 2 \sqrt{2}}{2} = -2 - \sqrt{2}$

$x_2 = \dfrac{-4 + 2 \sqrt{2}}{2} = -2 + \sqrt{2}$

Ces deux solutions sont bien dans l'ensemble de définition. Les solutions de l'équation sont donc: 

\Answer{$\mathcal{S} = \{ -2 - \sqrt{2} ; -2 + \sqrt{2} \}$}

}

\question{}
$\dfrac{1}{x-1} = \dfrac{1}{3}$

\correction{
On recherche les valeurs interdites: $x-1 = 0 \Leftrightarrow x = 1$. 1 est donc l'unique valeur interdite. On résout sur $\mathbb{R} - \{1\}$. 

Produit en croix: L'équation équivaut à $x-1 = 3$ soit $x = 4$. La solution trouvée est bien dans $\mathbb{R} - \{1\}$.

D'où \Answer{$\mathcal{S} = \{4\}$}

}
