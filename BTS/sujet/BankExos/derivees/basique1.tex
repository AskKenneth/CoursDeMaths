\exo{Calculer les dérivées des fonctions suivantes. Les simplifier lorsque cela est possible:}

\question{}
$f_1(x) = 5x^2 + 2$

\correction{
$f'_1(x) = 10x$
}

\question{}
$f_2(x) = e^{5x^2 + 2}$

\correction{
$f'_2(x) = 10xe^{x^2 + 2}$
}

\question{}
$f_3(x) = \ln\left( \dfrac{1}{x^2 + 1} \right)$

\correction{
$f_3(x) = - \ln\left( x^2 + 1 \right)$, d'où:

$f'_3(x) = - \dfrac{2x}{x^2 + 1}$
}

\question{}
$f_4(x) = x \sqrt{x}$

\correction{
$f'_4(x) = \sqrt{x} + \dfrac{x}{2 \sqrt{x}}$
}

% \question{}
% $f_5(x) = e^{x} \ln x$

%\correction{
%$f'_5(x) = e^{x} \ln x + e^{x} \dfrac{1}{x} =  e^x \left( \ln x + \dfrac{1}{x} \left)$
%}

% \question{}
% $f_6(x) = \sqrt{x^{3} + 1}$

%\correction{
%$f'_6(x) = \dfrac{3x^2}{x^3 + 1}$
%}
% 
% \question{}
% $f_7(x) = \dfrac{3x + 1}{x^{2} + x + 1}$

%\correction{
%$f'_7(x) = \frac{(3 (x^{2}+x+1)-(3\cdot x+1) (2\cdot x+1))}{\left(x^{2}+x+1\right)^{2}}$
%}
% 
% \exo{Résoudre les équations suivantes:}
% 
% \question{}
% $x^{2} + 1 = 0$
% 
% \question{}
% $x^{2} - 1 = 0$
% 
% \question{}
% $e^{4x} = 2$
% 
% \question{}
% $e^{\frac{x}{3}} = 4$
% 
% \question{}
% $\left( \dfrac{1}{4} \right)^{x} = 8$
