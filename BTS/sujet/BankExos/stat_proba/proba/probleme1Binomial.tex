\probleme{Stylos}

Tous les résultats seront arrondis à $10^{-2}$ près.

Une entreprise produit en grande quantité des stylos. La probabilité qu'un stylo présente un défaut est égale à 0.1.

\question{} On prélève dans cette production, successivement et avec remise huit stylos.

On note X la variable aléatoire qui compte le nombre de stylos présentant un défaut parmi les huit stylos prélevés.

\subquestion{} On admet que X suit une loi binomiale. Donner les paramètres de cette loi.

\subquestion{} Calculer la probabilité des événements suivants:

\begin{itemize}
\item A: « il n'y a aucun stylo avec un défaut »;

\item B: « il y a au moins un stylo avec un défaut »;

\item C: « il y a exactement deux stylos avec un défaut ».
\end{itemize}

\question{} En vue d'améliorer la qualité du produit vendu, on décide de mettre en place un contrôle qui accepte tous les stylos sans défaut et 20\% des stylos avec défaut.

On prend au hasard un stylo dans la production. On note D l'événement « le stylo présente un défaut », et E l'événement « le stylo est accepté ».

\subquestion{} Construire un arbre traduisant les données de l'énoncé.

\subquestion{} Calculer la probabilité qu'un stylo soit accepté au contrôle.

\subquestion{} Justifier que la probabilité qu'un stylo ait un défaut sachant qu'il a été accepté au contrôle est égale à 0,022 à $10^{-3}$ près.

\question{} Après le contrôle, on prélève, successivement et avec remise, huit stylos parmi les stylos acceptés.

Calculer la probabilité qu'il n'y ait aucun stylo avec un défaut dans ce prélèvement de huit stylos.

Comparer ce résultat avec la probabilité de l'événement $A$ calculée à la question 1)b).

Quel commentaire peut-on faire?