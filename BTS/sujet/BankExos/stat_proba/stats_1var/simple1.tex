%Un exercice avec son corrigé à inclure dans un fichier principal.
\exo{Statistiques à une variable}

Une entreprise fabrique des conserves alimentaires dont l'étiquette annonce une masse de 250 grammes.
Les masses obtenues pour un échantillon de 500 conserves prises au hasard sont données dans le tableau suivant: 

\begin{table}[h]
\centering
\caption{Masses des 500 conserves.}
\label{table_conserves}
\begin{tabular}{|l|l|l|l|l|l|}
\hline
\textbf{Masse (en g) $x_{i}$}        & {[}235 ; 240{[} & {[}240 ; 245{[} & {[}245 ; 250{[} & {[}250 ; 255{[} & {[}255 ; 260{[} \\ \hline
\textbf{Nb. de conserves} & 33              & 67              & 217             & 132             & 51              \\ \hline
\end{tabular}
\end{table}

\question{}
À l'aide de la calculatrice, calculer, en utilisant les milieux des classes, la masse moyenne $\overline{x}$ ainsi que l'écart type $\sigma_x$ des conserves de cet échantillon.

On fournira les valeurs en grammes arrondies au dixième.

\correction{
    $\overline{x} = \dfrac{237.5 \times 33 + \ldots + 257.5 \times 51}{500} = 248.5$g.

    $\sigma_x = 5.1$g.
}

\question{}
Donner les quartiles $Q_1$ et $Q_3$ et calculer l'équart interquartile.

\correction{
    $Q_1 = 247.5$g.

    $Q_3 = 252.5$g.

    L'écart interquartile vaut $Q_3 - Q_1 = 5$g.
}

\question{}
Calculer le pourcentage des conserves alimentaires ayant une masse comprise entre 240g et 255 grammes. 

\correction{
    Il y a 416 conserves qui dont la masse est comprise entre 240g et 255g. Cela représente $\dfrac{416}{500} = 0.832 = 83.2\%$. 
}
