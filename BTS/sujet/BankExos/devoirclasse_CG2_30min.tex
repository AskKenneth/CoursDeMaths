\documentclass[a4paper,12pt]{scrartcl}
\usepackage[utf8x]{inputenc}
\usepackage[T1]{fontenc} % avec T1 comme option  d'encodage c'est ben mieux, surtout pour taper du français.
%\usepackage{lmodern,textcomp} % fortement conseillé pour les pdf. On peut mettre autre chose : kpfonts, fourier,...
\usepackage[french]{babel} %Sans ça les guillemets, amarchpo
\usepackage{amsmath}
\usepackage{multicol}
\usepackage{amssymb}
\usepackage{tkz-tab}
\usepackage{exercice_sheet}

%\trait
%\section*{}
%\exo{}
%\question{}
%\subquestion{}

\date{}

\newcommand{\classe}{CG1}
\newcommand{\typedevoir}{Devoir en classe} 
\newcommand{\duree}{60 minutes}
  
\renewcommand{\hascorrection}{1} 


% Title Page
\title{\typedevoir{} -- \classe{}\writecorrword{}} 

\author{Durée : \duree{}}

\begin{document}

\maketitle 

Note: les questions \textbf{ne} sont \textbf{pas} par ordre croissant de difficulté. 

Calculatrice inutile mais autorisée, smartphone interdit. Sujet à rendre avec la copie, ne restez pas bloqué$\cdot$e sur une question trop longtemps au détriment du reste.

%Un exercice avec son corrigé à inclure dans un fichier principal.

\exo{Résoudre dans $\mathbb{R}$ les équations suivantes}

\question{}
$x^2 + 4 = 8x$

\correction{
L'équation équivaut à $x^2 - 8x + 4 = 0$. Il s'agit d'une équation polynomiale du 2nd degré.

$\Delta = 48 > 0$. L'équation a donc 2 solutions.

$\sqrt{\Delta} = 4\sqrt{3}$

$x_1 = \dfrac{8 + 4\sqrt{3}}{2} = 4 + 2\sqrt{3}$
 
$x_2 = \dfrac{8 - 4\sqrt{3}}{2} = 4 - 2\sqrt{3}$

\Answer{$\mathcal{S} = \left\lbrace 4 - 2\sqrt{3} ; 4 + 2\sqrt{3} \right\rbrace$}
}

\question{}
$\dfrac{x+1}{x-1} = 1$

\correction{
Il y a une écriture fractionnaire où $x$ apparaît au dénominateur. On cherche donc la ou les valeurs interdites:

$x-1=0 \Leftrightarrow x = 1$

1 est l'unique valeur interdite. On résout donc sur $\mathbb{R} - \{1\}$

Pour $x \neq 1$, l'équation équivaut à $x+1 = x-1$

$\Leftrightarrow 0=2$.

On a donc: \Answer{$\mathcal{S} = \emptyset $}
}

\question{}
$\dfrac{x^2 + 4x + 2}{x^2 + 6x + 5} = 0$

\correction{
L'écriture est fractionnaire où $x$ apparaît au dénominateur. On cherche donc la ou les valeurs interdites:

\begin{equation}
x^2 + 6x + 5 = 0
\label{eq:gnieh}
\end{equation}

Il s'agit d'une équation du second degré: 

$\Delta = 6^2 - 4 \times 5 = 16 > 0$ donc l'équation a 2 solutions.

$\sqrt{\Delta} = 4$

$x_1 = \dfrac{-6 + 4}{2} = -1$

$x_2 = \dfrac{-6 - 4}{2} = -5$

L'ensemble des solutions de l'équation \ref{eq:gnieh} est $\{-5;-1\}$.

Il s'agit aussi de l'ensemble des valeurs interdites de l'équation de départ, on résout donc sur $\mathbb{R} - \{-5;-1\}$. 

$\dfrac{x^2 + 4x + 2}{x^2 + 6x + 5} = 0 \Leftrightarrow x^2 + 4x + 2 = 0$

Il s'agit d'une équation du second degré: 

$\Delta = 4^2 - 4 \times 2 = 8 > 0$ donc l'équation a 2 solutions.

$\sqrt{\Delta} = 2 \sqrt{2}$

$x_1 = \dfrac{-4 - 2 \sqrt{2}}{2} = -2 - \sqrt{2}$

$x_2 = \dfrac{-4 + 2 \sqrt{2}}{2} = -2 + \sqrt{2}$

Ces deux solutions sont bien dans l'ensemble de définition. Les solutions de l'équation sont donc: 

\Answer{$\mathcal{S} = \{ -2 - \sqrt{2} ; -2 + \sqrt{2} \}$}

}

\question{}
$\dfrac{1}{x-1} = \dfrac{1}{3}$

\correction{
On recherche les valeurs interdites: $x-1 = 0 \Leftrightarrow x = 1$. 1 est donc l'unique valeur interdite. On résout sur $\mathbb{R} - \{1\}$. 

Produit en croix: L'équation équivaut à $x-1 = 3$ soit $x = 4$. La solution trouvée est bien dans $\mathbb{R} - \{1\}$.

D'où \Answer{$\mathcal{S} = \{4\}$}

}


%Un exercice avec son corrigé à inclure dans un fichier principal.

\exo{Donner les ensembles de définition des fonctions suivantes}

\question{}
$f_1(x) = \dfrac{1}{x^2-1}$

\correction{
L'expression de $f_1(x)$ contient une écriture fractionnaire avec $x$ au dénominateur. Il faut donc résoudre l'équation $x^2-1$ pour trouver les potentielles valeurs interdites. 

$x^2 - 1 = 0$

$\Leftrightarrow x^2 = 1$

$\Leftrightarrow x = 1$ ou $x = -1$

L'ensemble de définition est donc: 

\Answer{$D_{f_1} = \mathbb{R} - \{-1 ; 1\}$}
}

\question{}
$f_2(x) = \sqrt{\dfrac{2x-1}{x+4}}$

\correction{
Pour que $f_2(x)$ soit défini, il faut que $\dfrac{2x-1}{x+4} \geqslant 0$

Il faut donc dresser un tableau de signe de $\dfrac{2x-1}{x+4}$. Pour ce faire, il faut d'abord étudier le signe du numérateur et du dénominateur:

\begin{itemize}
 \item $2x-1 \geqslant 0 \Leftrightarrow x \geqslant \frac{1}{2}$
 \item $x+4 \geqslant 0 \Leftrightarrow x \geqslant -4$
\end{itemize}

On peut maintenant dresser le tableau de signe:

\begin{center}
\begin{tikzpicture}
   \tkzTabInit{$x$ / 1 , $2x-1$ / 1 , $x+4$ / 1, $\dfrac{2x-1}{x+4}$ / 1}{$-\infty$, $-4$, $\frac{1}{2}$ , $\infty$}
   \tkzTabLine{, -,  , -,  z,+}
   \tkzTabLine{, -, z, +,   ,+}
   \tkzTabLine{, +, d, -,  z,+}
\end{tikzpicture}
\end{center}

D'où: $\dfrac{2x-1}{x+4} \geqslant 0 \Leftrightarrow x \in ]-\infty ; -4 [ \cup \left[\frac{1}{2} ; +\infty \right]$

On a donc

\Answer{$D_{f_2} = ]-\infty ; -4 [ \cup \left[\dfrac{1}{2} ; +\infty \right[$}
}


%Un exercice avec son corrigé à inclure dans un fichier principal.

\exo{Mise en équation, résolution}

Dans 20 ans, Iseult aura deux fois l'âge qu'elle avait il y a 5 ans.

Quel âge a-t-elle?

\correction{
Appelons $i$ son âge actuel. On a alors:

$(i+20) = 2(i-5)$

$\Leftrightarrow i+20 = 2i-10$

$S = \{30\}$

Iseult a donc:

\Answer{30 ans}
}


%Un exercice avec son corrigé à inclure dans un fichier principal.

\exo{Composées de fonctions}

Soient les fonctions $f$, $g$ et $h$ définies sur $\mathbb{R}$ telles que:

\begin{itemize}
 \item $f: x \longmapsto 2x+1$
 \item $g: x \longmapsto \frac{1}{2}x - \frac{1}{2}$
 \item $h: x \longmapsto x^3$
\end{itemize}

Donner l'expression en fonction de $x$ des composées suivantes:

\question{} $f \circ g$

\correction{
$x \longmapsto \frac{1}{2}x - \frac{1}{2} \longmapsto 2\left(\frac{1}{2}x - \frac{1}{2} \right)+1 = x$

\Answer{$f \circ g(x) = x$}
}

\question{} $f \circ h$

\correction{
$x \longmapsto x^3 \longmapsto 2x^3+1$

\Answer{$f \circ h(x) = 2x^3+1$}
}

\question{} $h \circ f$

\correction{
$x \longmapsto 2x+1 \longmapsto (2x+1)^3$

\Answer{$f \circ h(x) = (2x+1)^3$}

}


%Un exercice avec son corrigé à inclure dans un fichier principal.

\exo{} 
Soit une fonction $f$ définie sur $[-2;3]$ et de courbe représentative $\mathcal{C}_f$ (ci-dessous).
Résoudre par lecture graphique les équations suivantes. 

\question{}
$f(x) = -2$

\correction{
$\mathcal{S} = \left\{ -2 \right\}$
}

\question{}
$f(x) = 1$

\correction{
$\mathcal{S} = \left\{ 0;2 \right\}$
}

\question{}
$f(x) = 3$

\correction{
$\mathcal{S} = \left\{ 3 \right\}$
} 


\begin{center}
    \simpleplot{-2.1}{3.1}{\x}{5/24*(\x)^3 - 3/8*(\x)^2 - 1/12*(\x)+1}{$\mathcal{C}_f$}{1.2}
\end{center}
%\begin{center}
%\simpleplot{-4}{1}{\x}{(\x)^3+4*(\x)^2+2*(\x)-3}{$\mathcal{C}_f$}{1.5}
%\end{center}


\end{document}
