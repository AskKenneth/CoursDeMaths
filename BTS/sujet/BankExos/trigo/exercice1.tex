\exo{Trigonométrie, écritures algébriques}

\question{}
Soit $B = (\cos x + \sin x)^2 + (\cos x - \sin x)^2$. Montrer que $B = 2$.

\correction{
$(\cos x + \sin x)^2 = \cos^2 x + 2 \cos x \sin x + \sin^2 x$

et

$(\cos x - \sin x)^2 = \cos^2 x - 2 \cos x \sin x + \sin^2 x$ or, $\cos^2 x + \sin^2 x = 1$

D'où $B = 2 \cos^2 x + 2 \sin^2 x = 2(\cos^2 x + \sin^2 x) = 2$. 

}

\question{}
Résoudre l'équation suivante (résultat en radians): $2 \sin x +1 = 0$.

\correction{
Quand rien n'est précisé, c'est qu'on résout l'équation dans $\mathbb{R}$. 

$2 \sin x +1 = 0 \Leftrightarrow \sin x = -\frac{1}{2}$

$\arcsin \left( -\dfrac{1}{2} \right) = -\dfrac{\pi}{6}$

Par symétrie, $\pi - \left( - \dfrac{\pi}{6} \right) = \dfrac{7 \pi}{6}$ est également solution.

Donc $S = \left\lbrace -\dfrac{\pi}{6} + 2k\pi ; \dfrac{7\pi}{6} + 2k\pi\right\rbrace$ avec $k \in \mathbb{Z}$\footnote{Ok, celle-là elle est un peu costaud, mais pour voir ce qui se passe, un petit dessin de cercle trigo est suffisant...}.
}

\question{}
$\alpha \in [\pi ; 2\pi]$ tel que $\cos \alpha = \frac{1}{\sqrt{3}}$. Calculer la valeur de $\sin \alpha$.

\correction{
$\cos^2 \alpha = \dfrac{1}{3}$.

Or, $\cos^2 \alpha + \sin^2 \alpha = 1, \forall \alpha \in \mathbb{R}$.

D'où $\sin^2 \alpha = 1- \cos^2 \alpha = \dfrac{2}{3}$

Soit: \Answer{$\sin \alpha = \pm \sqrt{\dfrac{2}{3}}$}
}
