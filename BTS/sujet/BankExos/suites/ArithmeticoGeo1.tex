%Un exercice avec son corrigé à inclure dans un fichier principal.
\exo{}

On considère la suite $(u_n)$ définie pour tout entier naturel $n$ par:

$$\left\lbrace
\begin{array}{ll}
   u_0 = 13 \\
   u_{n+1} = 3 u_n - 14 \\
\end{array}
\right.$$

% $u_0 = 13$ et pour tout entier naturel $n$, $u_{n+1} = 3 u_n - 14$.

\question{}
Calculer $u_1$, $u_2$ et $u_3$.

\correction{
\begin{itemize}
 \item $u_0 = 13$
 \item $u_1 = 3u_0 - 14 = 25$
 \item $u_2 = 3u_1 - 14 = 61$
 \item $u_3 = 3u_2 - 14 = 169$
\end{itemize}
}

\question{}
On considère la suite $(v_n)$ définie pour tout entier naturel $n$, par $v_n = u_n - 7$.


\subquestion{}
Montrer que la suite $(v_n)$ est une suite géométrique dont on donnera le premier terme et la raison.

\correction{
Calculons $\frac{v_{n+1}}{v_n}$.

$\forall n \in \mathbb{N}, v_n + u_n - 7$, d'où $v_{n+1} = u_{n+1} - 7$

Donc $\frac{v_{n+1}}{v_n} = \frac{u_{n+1} - 7}{v_n}$

Or d'après l'énoncé, $u_{n+1} = 3 u_n - 14$

D'où $\frac{v_{n+1}}{v_n} = \frac{3 u_n - 14 - 7}{v_n} = \frac{3 u_n - 21}{v_n}$

On déduit de l'énoncé que $u_n = v_n + 7$. 

Donc $\frac{v_{n+1}}{v_n} = \frac{3 (v_n + 7) - 21}{v_n}$

Finalement:

\Answer{$\dfrac{v_{n+1}}{v_n} = 3$}

$(v_n)$ est donc géométrique de raison 3 et de premier terme $v_0 = u_0 - 7 = 6$.
}

\subquestion{}
En déduire l'expression de $v_n$ en fonction de $n$.

\correction{
On en déduit son terme général: $v_n = 6 \times 3^n$
}

\subquestion{}
En déduire que $u_n = 6 \times 3^n + 7$.

\correction{
$u_n = v_n + 7 = 6 \times 3^n + 7$
}

\subquestion{}
Déterminer $\lim u_n$ et $\lim v_n$.

\correction{
$(v_n)$ est géométrique de raison $q = 3 > 1$ et de premier terme $v_0 = 6 > 0$. 

\begin{equation*}
\lim_{n \to +\infty} v_n = + \infty 
\end{equation*}

De plus, $v_n > u_n$ quel que soit $n \in \mathbb{N}$, d'où: 

\begin{equation*}
 \lim_{n \to +\infty} u_n = + \infty 
\end{equation*}
}

\question{}
Soient les suites $U_n = \frac{1}{u_n}$ et $V_n = \frac{1}{v_n}$. Déterminer $\lim U_n$ et $\lim V_n$.

\correction{
 D'après les résultats précédents, $\lim U_n$ est de la forme $\frac{1}{+\infty}$ d'où $\lim U_n = 0^+$.

 D'après les résultats précédents, $\lim V_n$ est de la forme $\frac{1}{+\infty}$ d'où $\lim V_n = 0^+$.
}


