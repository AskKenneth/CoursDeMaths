\documentclass[a4paper,12pt]{scrartcl}
\usepackage[utf8x]{inputenc}
\usepackage[T1]{fontenc} % avec T1 comme option  d'encodage c'est ben mieux, surtout pour taper du français.
%\usepackage{lmodern,textcomp} % fortement conseillé pour les pdf. On peut mettre autre chose : kpfonts, fourier,...
\usepackage[french]{babel} %Sans ça les guillemets, amarchpo
\usepackage{amsmath}
\usepackage{multicol}
\usepackage{amssymb}
\usepackage{tkz-tab}
\usepackage{exercice_sheet}



%\trait
%\section*{}
%\exo{}
%\question{}
%\subquestion{}

\date{}


% Title Page
\title{Devoir en classe, préformation BTS}

\author{\rotatebox{180}{\textsc{Mathématiques}} \\ 60 minutes}

\begin{document}

\newcommand{\fewlines}{3}
\newcommand{\midlines}{4}
\newcommand{\manylines}{6}

\maketitle

%{\Large Nom:} 
%\hspace{60mm}
%{\Large Prénom:}
%\vspace{4mm}

Note : la calculatrice est autorisée. Donc sauf mention contraire, tout résultat donné sans la moindre étape de calcul ne sera pas pris en compte.

On rappelle que résoudre une équation, c'est donner l'\emph{ensemble} des solutions. Par exemple: $\mathcal{S} = \left\lbrace 3 \right\rbrace$, $\mathcal{S} = \left[\dfrac{-4 \pi}{7} ; +\infty \right[$ ou encore $\mathcal{S} = \left\lbrace \dfrac{1}{2} ; 4 \right\rbrace$, etc.

 \exo{Équations du 2\textsuperscript{nd} degré à une inconnue}

Résoudre les équations suivantes en donnant les valeurs \emph{exactes}:

\question{} $3x^{2} - 4x -4 = 0$

\lignes{\fewlines}

\question{} $7x^{2} -8x -8 = 0$

\lignes{\fewlines}

\question{} $-3x^{2} + 8x = 5$

\lignes{\fewlines}

\question{} $-4x^{2} + 3x - 4 = 0$

\lignes{\fewlines}

\question{} Factoriser $7x^{2} -8x -8$

\lignes{\fewlines}


\exo{Dérivation}

Donner la dérivée des fonctions suivantes:

\question{} $f_1(x) = -3x^{2} + 8x - 5$

\lignes{\fewlines}

\question{} $f_2(x) = 8x^3$

\lignes{\fewlines}

\question{} $f_3(x) = 10^{12}$

\lignes{\fewlines}



\exo{Systèmes de deux équations à deux inconnues linéaires}

Résoudre les systèmes suivants:

\question{} 

\[
\left \{
\begin{array}{c @{=} c}
    2x - 3y & 7 \\
    x  + 5y & -3 \\
\end{array}
\right.
\]

\lignes{\midlines}

\question{} 

\[
\left \{
\begin{array}{c @{=} c}
    3x + 4y & 32 \\
    7x + 6y & 58 \\
\end{array}
\right.
\]

\lignes{\midlines}

\question{} 

Jean-Marie a acheté 5 torchons et 3 serviettes. Il a payé 23.90€.

Marie-Pierre a acheté 4 torchons et 2 serviettes. Elle a payé 17.60€.

Quel est le prix d'un torchon ? Quel est le prix d'une serviette ?

\lignes{\manylines}

\exo{Système de deux équations à deux inconnues non-linéaires}

Résoudre le système suivant:

\[
\left \{
\begin{array}{c @{=} c}
    x + 3y & 0 \\
    4x^2 + 5y^2 & 41 \\
\end{array}
\right.
\]

\lignes{\manylines}

\begin{center}
Fin.
\end{center}

\section*{Formulaires}

\documentclass[a4paper,12pt]{scrartcl}
\usepackage[utf8x]{inputenc}
\usepackage[T1]{fontenc} % avec T1 comme option  d'encodage c'est ben mieux, surtout pour taper du français.
%\usepackage{lmodern,textcomp} % fortement conseillé pour les pdf. On peut mettre autre chose : kpfonts, fourier,...
\usepackage[french]{babel} %Sans ça les guillemets, amarchpo
\usepackage{amsmath}
\usepackage{multicol}
\usepackage{amssymb}
\usepackage{tkz-tab}
\usepackage{exercice_sheet}

%\trait
%\section*{}
%\exo{}
%\question{}
%\subquestion{}

\date{}


% Title Page
\title{Formulaire, polynômes du second degré à une inconnue}

\author{\textsc{Mathématiques}}

\begin{document}

\maketitle




On considère une fonction polynomiale du second degré $P$ telle que $P(x) = ax^2 + bx + c$ où $a$, $b$ et $c$ sont des nombres réels. 

\section*{Calcul du discriminant}

$\Delta = b^2 - 4ac$

\section*{Solutions de l'équation $P(x) = 0$}

\begin{itemize}
\item $\Delta > 0$

L'équation a 2 solutions : 

$\mathcal{S} = \left\lbrace \dfrac{-b-\sqrt{\Delta}}{2a} ; \dfrac{-b+\sqrt{\Delta}}{2a} \right\rbrace$

\item $\Delta = 0$

L'équation a une unique solution :

$\mathcal{S} = \left\lbrace \dfrac{-b}{2a}\right\rbrace$

\item $\Delta < 0$

L'équation n'a pas de solution :

$\mathcal{S} = \emptyset$

\end{itemize}

\section*{Factorisation de $P(x)$}

\begin{itemize}
\item $\Delta > 0$

L'équation $P(x) = 0$ a 2 solutions que l'on appelle $x_1$ et $x_2$: 

On peut donc écrire la forme factorisée de $P(x)$: $P(x) = a(x - x_1)(x - x_2)$

\item $\Delta = 0$

L'équation $P(x) = 0$ a une unique solution que l'on appelle $x_1$:

On peut donc écrire la forme factorisée de $P(x)$: $P(x) = a(x - x_1)^2$

\item $\Delta < 0$

L'équation $P(x) = 0$ n'a pas de solution, on ne peut donc pas factoriser $P(x)$.

\end{itemize}


\trait

\begin{center}
Fin.
\end{center}

\end{document}


\section*{Dérivation}

\subsection*{Dérivées de fonctions usuelles}

\begin{center}

\begin{tabular}{|c|c|}
\hline 
\textbf{fonction $f$} & \textbf{fonction dérivée $f'$} \\ 
\hline 
$k$, $k \in \mathbb{R}$ & $0$ \\ 
\hline 
$x$ & $1$ \\ 
\hline 
$x^2$ & $2x$ \\ 
\hline 
$x^{\alpha}$, $\alpha \in \mathbb{Z}$ & $\alpha x^{\alpha - 1}$ \\ 
\hline 
$e^x$ & $e^x$ \\ 
\hline 
$\ln x$ & $\dfrac{1}{x}$ \\ 
\hline 
$\sqrt{x}$ & $\dfrac{1}{2\sqrt{x}}$ \\ 
\hline 
\end{tabular} 
\end{center}

\subsection*{Dérivées de produits, quotients, etc.}

\begin{center}
\begin{tabular}{|c|c|}
\hline 
\textbf{fonction $f$} & \textbf{fonction dérivée $f'$} \\ 
\hline 
$u+v$ & $u'+v'$ \\ 
\hline 
$ku$ & $ku'$ \\ 
\hline 
$u \times v$ & $u'v+uv'$ \\ 
\hline 
$\dfrac{u}{v}$ & $\dfrac{u'v-uv'}{v^2}$ \\ 
\hline 
$\dfrac{1}{u}$ & $-\dfrac{u'}{u^2}$ \\ 
\hline 
\end{tabular} 
\end{center}

% \subsection*{Dérivées de fonctions composées}
% 
% 
% \begin{center}
% \begin{tabular}{|c|c|}
% \hline 
% \textbf{fonction $f$} & \textbf{fonction dérivée $f'$} \\ 
% \hline 
% $v \circ u$ ou $v(u(x))$ & $u' \times v' \circ u $ ou $u'(x) \times v'(u(x))$ \\ 
% \hline 
% $e^u$ & $u'e^u$ \\ 
% \hline 
% $\ln u$ & $\dfrac{u'}{u}$ \\ 
% \hline 
% $u^\alpha$ & $\alpha u' u^{\alpha - 1}$ \\ 
% \hline 
% \end{tabular}
% \end{center}


\trait

\end{document}

