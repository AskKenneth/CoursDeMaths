\documentclass[a4paper,12pt]{scrartcl}
\usepackage[utf8x]{inputenc}
\usepackage[T1]{fontenc} % avec T1 comme option  d'encodage c'est ben mieux, surtout pour taper du français.
%\usepackage{lmodern,textcomp} % fortement conseillé pour les pdf. On peut mettre autre chose : kpfonts, fourier,...
\usepackage[french]{babel} %Sans ça les guillemets, amarchpo
\usepackage{amsmath}
\usepackage{multicol}
\usepackage{amssymb}
\usepackage{tkz-tab}
\usepackage{exercice_sheet}

%\trait
%\section*{}
%\exo{}
%\question{}
%\subquestion{}

\date{}


% Title Page
\title{Devoir maison 1}

\author{Mathématiques}

\begin{document}

\maketitle

\exo{Racines carrées}

Simplifier les racines suivantes.
\begin{multicols}{2}

\question{$\sqrt{300}$}

\question{$\sqrt{180}$}

\question{$\sqrt{192}$}

\question{$\sqrt{3528}$}
\end{multicols}

\exo{Factorisations}

Factoriser puis réduire les expressions suivantes.

\begin{multicols}{2}
\question{}
$x^2 + 2x$

\question{}
$4x^2 + 8x + 4$

\question{}
$9x^2 - 144$

\question{}
$12x^2 - 24x + 12$
\end{multicols}

\exo{Polynômes du 2\textsuperscript{nd} degré}

Pour chacun de ces polynômes, trouver les racines dans $\mathbb{R}$, les factoriser et donner le tableau de signe.

\begin{multicols}{2}
\question{}
$5x^2 + x - 6$

\question{}
$-6x^2 + x + 7$

\question{}
$2x^2 + x - 3$

\question{}
$-3x^2 -5x +2$
\end{multicols}

\exo{Systèmes de 2 équations à 2 inconnues}

\begin{multicols}{2}
\question{}
$$
\begin{cases} 
a+b &= 0 \\
a-b &= 6
\end{cases}
$$

\question{}
$$
\begin{cases} 
2f+3g &= 8 \\
-f+2g &= 3
\end{cases}
$$

\question{}
$$
\begin{cases} 
3y+z &= 9 \\
\dfrac{z}{2} + 2y &= 8
\end{cases}
$$

\question{}
Dans une étable peuplée uniquement de poules et de vaches, il y a 38 têtes et 120 pattes. Combien y'a-t-il de poules, combien y'a-t-il de vaches?
\end{multicols}

\clearpage
\exo{Statistiques}

On a mesuré les tailles en cm des personnes dans un groupe de 16 personnes. Voici les résultats:

140, 141, 148, 153, 156, 157, 158, 164, 171, 181, 182, 190, 191, 195, 196, 199

Indiquer:

\question{L'étendue}

\question{La moyenne}

\question{La médiane}

\question{Le premier et le troisième quartile}

\question{L'étendue}


\trait

\begin{center}
Fin.
\end{center}

\end{document}
