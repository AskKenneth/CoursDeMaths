\documentclass[a4paper,12pt]{scrartcl}
\usepackage[utf8x]{inputenc}
\usepackage[T1]{fontenc} % avec T1 comme option  d'encodage c'est ben mieux, surtout pour taper du français.
%\usepackage{lmodern,textcomp} % fortement conseillé pour les pdf. On peut mettre autre chose : kpfonts, fourier,...
\usepackage[french]{babel} %Sans ça les guillemets, amarchpo
\usepackage{amsmath}
\usepackage{multicol}
\usepackage{amssymb}
\usepackage{tkz-tab}
\usepackage{exercice_sheet}

%\trait
%\section*{}
%\exo{}
%\question{}
%\subquestion{}

\date{}


% Title Page
\title{Devoir maison, équations du second degré, CG-1}

\author{\textsc{Mathématiques}}

%mettre la correction ou pas
\renewcommand{\hascorrection}{1}

\begin{document}

\maketitle

\exo{Dans chaque cas, résoudre l'équation $P(x) = 0$ et factoriser $P(x)$ lorsque c'est possible.}

\question{}
$P(x) = -3x^2	+2x 	+5$

\correction{
$P$ est un polynôme du second degré. On calcule donc $\Delta = b^2 - 4ac$:

$\Delta = 2^2 - 4 \times (-3) \times 5 = 64 > 0$ et $\sqrt{\Delta} = 8$

L'équation a donc 2 solutions: $x_{1,2} = \dfrac{-b \pm \sqrt{\Delta}}{2a}$

$x_1 = \frac{-2-8}{-6} = \frac{5}{3}$

$x_2 = \frac{-2+8}{-6} = -1$

On a donc: \Answer{$S = \left\lbrace -1 ; \dfrac{5}{3} \right\rbrace$}
}

\question{}
$P(x) = 3x^2	-1x	-2$

\correction{
$P$ est un polynôme du second degré. On calcule donc $\Delta = b^2 - 4ac$:

$\Delta = (-1)^2 - 4 \times 3 \times (-2) = 25 > 0$ et $\sqrt{\Delta} = 5$

L'équation a donc 2 solutions:  $x_{1,2} = \dfrac{-b \pm \sqrt{\Delta}}{2a}$

$x_1 = \frac{1+5}{6} = 1$

$x_2 = \frac{1-5}{6} = -\frac{2}{3}$

On a donc: \Answer{$S = \left\lbrace -\dfrac{2}{3} ; 1 \right\rbrace$}
}

\question{}
$P(x) = 5x^2	-8x	+6$

\correction{
$P$ est un polynôme du second degré. On calcule donc $\Delta = b^2 - 4ac$:

$\Delta = (-8)^2 - 4 \times 5 \times 6 = 56 < 0$ 

L'équation n'a donc pas de solution: \Answer{$S = \emptyset $}
}

\question{}
$P(x) = 2x^2	-4x	+2$

\correction{
$P$ est un polynôme du second degré. On calcule donc $\Delta = b^2 - 4ac$:

$\Delta = (-4)^2 - 4 \times 2 \times 2 = 25 > 0$ 

L'équation a donc 1 solution:  $x_{1} = \dfrac{-b }{2a}$

$x_1 = \frac{4}{4} = 1$

On a donc: \Answer{$S = \left\lbrace 1 \right\rbrace$}
}


\exo{Lecture graphique (graphe page \pageref{fonction})}


\begin{figure}
\begin{center}
\begin{tikzpicture}
\tkzInit[xmin=-2.3,xmax=4.3,ymin=-3,ymax=9]
\tkzGrid[sub,color=gray, subxstep=.5,subystep=.5]
\tkzAxeXY[very thick]
\tkzGrid

\draw [domain=-2.3:4.3, very thick,] plot(\x,{(\x)^2 - 2*\x - 1});

\correction{\draw (-2,7) node {$f(x) = 7$};
\draw (4,7) node {};
\draw (-2,0) node {};
\draw (4,0) node {};

\draw [very thick, dashed, red] (-2,7) -- (4,7);
\draw [very thick, dashed, red] (-2,7) -- (-2,0);
\draw [very thick, dashed, red] (4,7) -- (4,0);

\draw (-2,-2) node {$f(x) = -2$};
\draw [very thick, dashed, green](-2,-2) -- (4,-2);
\draw [very thick, dashed, green](1,-2) -- (1,0);


\draw (-2,-2.5) node {$f(x) = -\frac{5}{2}$};
\draw [very thick, dashed, blue](-2,-2.5) -- (4,-2.5);}

\end{tikzpicture}
\end{center}
\caption{Courbe $\mathcal{C}_f$, représentative de la fonction $f:x\longmapsto x^2-2x-1$}
\label{fonction}
\end{figure}



\question{Résoudre graphiquement (on pourra rendre l'énoncé avec les traits de construction):}

\subquestion{}
$f(x) = 7$

\correction{

}

\subquestion{}
$f(x) = -2$

\correction{

}

\subquestion{}
$f(x) = -\frac{5}{2}$

\correction{

}

\question{Retrouver les résultats de la question précédente par le calcul}

\correction{

\begin{itemize}
 \item $f(x) = 7 \Leftrightarrow x^2 - 2x - 8 = 0$. $\Delta = 36 > 0$

L'équation a donc 2 solutions:  $x_{1,2} = \dfrac{-b \pm \sqrt{\Delta}}{2a}$

$x_1 = \frac{2+6}{2} = 4$

$x_2 = \frac{2-6}{2} = -2$

On a donc: \Answer{$S = \left\lbrace -2 ; 4 \right\rbrace$}

\item $f(x) = -2 \Leftrightarrow x^2 - 2x + 1 = 0$. $\Delta = 0$

L'équation a donc 1 solution:  $x_{1} = \dfrac{-b }{2a}$

$x_1 = \frac{2}{2} = 1$

On a donc: \Answer{$S = \left\lbrace 1 \right\rbrace$}

    \item $f(x) = -\frac{5}{2} \Leftrightarrow x^2 - 2x + \frac{3}{2} = 0$. $\Delta = -2 < 0$

L'équation n'a donc pas de solution: \Answer{$S = \emptyset $}
\end{itemize}


}


\exo{Mise en équations}

Le produit de l'âge de Robert dans 10 ans par celui qu'il avait il y a 10 ans est égal à 44.

Quel est l'âge de Robert ?

\correction{
Appelons $r$ l'âge de Robert. 

Dans 10 ans, Robert aura $r+10$ ans.

Il y a 10 ans, Robert avait $r-10$ ans.

On a donc l'équation $(r+10)(r-10) = 44$.

$\Leftrightarrow r^2 - 100 = 44$

$\Leftrightarrow r^2 = 144$

$\Leftrightarrow r = 
\left\lbrace
\begin{array}{ll}
   x = 12 \\
   x = -12 \\
\end{array}
\right.$

L'âge de Robert ne pouvant être négatif, on en déduit qu'il a 12 ans. 
}


\end{document}
