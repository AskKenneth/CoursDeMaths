\documentclass[a4paper,12pt]{scrartcl}
\usepackage[utf8x]{inputenc}
\usepackage[T1]{fontenc} % avec T1 comme option  d'encodage c'est ben mieux, surtout pour taper du français.
%\usepackage{lmodern,textcomp} % fortement conseillé pour les pdf. On peut mettre autre chose : kpfonts, fourier,...
\usepackage[french]{babel} %Sans ça les guillemets, amarchpo
\usepackage{amsmath}
\usepackage{multicol}
\usepackage{amssymb}
\usepackage{tkz-tab}
\usepackage{exercice_sheet}

%\trait
%\section*{}
%\exo{}
%\question{}
%\subquestion{}

\date{}


% Title Page
\title{Devoir maison 1, OL-2}

\author{\textsc{Mathématiques}}

\begin{document}

\maketitle

\exo{}

\question{}
24 ; 18 ; 12 ; 6 sont-ils les premiers termes d'une suite arithmétique? Si oui, en donner le premier terme et la raison.

\question{}
$2$ ; $-3$ ; $\frac{9}{2}$ ; $-\frac{27}{4}$ sont-ils les premiers termes d'une suite géométrique? Si oui, en donner le premier terme et la raison.

\question{}
Soit la suite géométrique $(u_n)$ de premier terme $u_0 = 7$ et de raison $q = -2$. 

\subquestion{}
Calculer $u_1$, $u_4$ et $u_7$.

\subquestion{}
Calculer $S_{4,12} = u_4 + u_5 + \ldots + u_{12}$.

\exo{}

On considère la suite $(u_n)$ définie pour tout entier naturel $n$ par:

$$\left\lbrace
\begin{array}{ll}
   u_0 = 13 \\
   u_{n+1} = 3 u_n - 14 \\
\end{array}
\right.$$

% $u_0 = 13$ et pour tout entier naturel $n$, $u_{n+1} = 3 u_n - 14$.

\question{}
Calculer $u_1$, $u_2$ et $u_3$.

\question{}
On considère la suite $(v_n)$ définie pour tout entier naturel $n$, par $v_n = u_n - 7$.

\subquestion{}
Montrer que la suite $(v_n)$ est une suite géométrique dont on donnera le premier terme et la raison.

\subquestion{}     
En déduire l'expression de $v_n$ en fonction de $n$.

\subquestion{}
En déduire que $u_n = 6 \times 3^n + 7$.

\subquestion{}
Déterminer $\lim u_n$ et $\lim v_n$.

\question{}
Soient les suites $U_n = \frac{1}{u_n}$ et $V_n = \frac{1}{v_n}$. Déterminer $\lim U_n$ et $\lim V_n$.


\trait

\begin{center}
Fin.
\end{center}

\end{document}
