\documentclass[a4paper,12pt]{scrartcl}
\usepackage[utf8x]{inputenc}
\usepackage[T1]{fontenc} % avec T1 comme option  d'encodage c'est ben mieux, surtout pour taper du français.
%\usepackage{lmodern,textcomp} % fortement conseillé pour les pdf. On peut mettre autre chose : kpfonts, fourier,...
\usepackage[french]{babel} %Sans ça les guillemets, amarchpo
\usepackage{amsmath}
\usepackage{multicol}
\usepackage{amssymb}
\usepackage{tkz-tab}
\usepackage{exercice_sheet}



%\trait
%\section*{}
%\exo{}
%\question{}
%\subquestion{}

\date{}


% Title Page
\title{BTS blanc, opticien lunetier}

\author{\textsc{Mathématiques}}

\begin{document}

\maketitle

\exo{Probabilités}
\textbf{Les deux parties de cet exercice peuvent être traitées de façon indépendantes.}

Une entreprise fabrique des verres ophtalmiques à partir de verres semi-finis.

\partie{Probabilités conditionnelles}

Ce fabriquant possède un stock de verres semi-finis provenant de deux fournisseurs différents, désignés par \emph{fournisseur 1} et \emph{fournisseur 2}.

On admet que 60 \% des verres semi-finis proviennent du fournisseur 1, le reste venant du fournisseur 2.

On admet également que 2 \% des verres semi-finis du fournisseur 1 sont défectueux et que 1 \% des verres semi-finis du fournisseur 2 sont défectueux.

On prélève au hasard un verre semi-fini dans ce stock.

On considère les événements suivants:

\begin{itemize}
\item A: \og le verre semi-fini prélevé provient du fournisseur 1 \fg{};
\item B: \og le verre semi-fini prélevé provient du fournisseur 2 \fg{};
\item D: \og le verre semi-fini prélevé est défectueux \fg{}.
\end{itemize}

Pour répondre, on pourra réaliser un arbre pondéré sur la copie. 

\question{}
Calculer la probabilité $P(B \cap D)$.

\question{}
Montrer que la probabilité que le verre semi-fini prélevé soit défectueux est égale à 0.016.

\question{}
Calculer la probabilité conditionnelle $P_D(B)$.

On rappelle que $P_D(B)$ est la probabilité de l'événement $B$ sachant que l'événement $D$ est réalisé.

\partie{Loi binomiale, loi de Poisson et loi normale}

Sauf indication contraire, les résultats seront arrondis à $10^{-3}$. 

On prélève au hasard $n$ verres semi-finis dans un stock pour vérification. La probabilité qu'un verre semi-fini prélevé au hasard dans ce stock soit défectueux est égale à 0.016. Le stock est assez important pour assimiler un prélèvement de $n$ verres à un tirage avec remise.

On considère la variable aléatoire $X$ qui, à tout prélèvement de $n$ verres semi-finis dans ce stock associe le nombre de verres semi-finis défectueux.

\question{}
Justifier que la variable aléatoire $X$ suit une loi binomiale dont on donnera les paramètres. 

\question{}
Dans cette question, $n=250$.

\subquestion{}
Calculer l'espérance mathématique $E(X)$. Interpréter le résultat.

\subquestion{}
Calculer la probabilité qu'aucun verre ne soit défectueux.

\subquestion{}
En déduire la probabilité qu'au moins 1 verre soit défectueux.

\subquestion{}
On admet que la loi de la variable aléatoire $X$ peut être approchée par une loi de Poisson. Donner le paramètre $\lambda$ de cette loi de Poisson.

\subquestion{}
On désigne par $Y$ une variable aléatoire suivant la loi de Poisson de paramètre obtenu à la question précédente. Calculer $P(Y \geqslant 1)$. 

\question{}
Dans cette question, $n = 1000$.

On admet que la loi de la variable aléatoire $X$ peut être approchée par la loi normale de moyenne 16 et d'écart-type 3.97: $\mathcal{N}(16;3.97)$.

Justifier ces paramètres par le calcul, on s'aidera de la loi binomiale $\mathcal{B}(1000;0.016)$.

\exo{Suites et études de fonction}

La chaîne de magasin Optitan commercialise des lunettes solaires. La direction se propose de déterminer le prix de vente unitaire d'un de ses modèles pour réaliser la meilleure recette possible.

La direction raisonne à partir des deux hypothèses suivantes:

\begin{itemize}
\item pour un prix de base de 50€, il y a 10 000 acheteurs en une année;
\item toute augmentation de 20€ entraîne une diminution de 20\% du nombre de clients.
\end{itemize}

\partie{Modèle discret}

\question{}
La suite $(p_n)$, avec $n$ entier naturel compris entre 0 et 4, des différents prix testés, en euros, par l'entreprise est une suite arithmétique de premier terme $p_0 = 50$ et de raison $r = 20$.

Déterminer $p_1$, $p_2$, $p_3$ et $p_4$.

\question{}
Pour $n$ entier naturel compris entre 0 et 4, on désigne par $c_n$ le nombre de clients acheteurs potentiels, lorsque le prix unitaire est égal à $p_n$.

\subquestion{}
Montrer que $(c_n)$ est une suite géométrique dont on précisera la raison et le premier terme.

\subquestion{}
Exprimer, pour tout entier naturel $n$ compris entre 0 et 4, $c_n$ en fonction de $n$.

\question{}
Pour tout entier naturel $n$ compris entre 0 et 4, on désigne par $r_n$ la recette correspondant au prix unitaire $p_n$.

\subquestion{}
Reproduire et compléter le tableau suivant:

\begin{center}
\begin{tabular}{|l|l|l|l|l|l|}
\hline
$n$   & 0       & 1  & 2 & 3 & 4   \\ \hline
$p_n$ & 50      & 70 &   &   & 130 \\ \hline
$c_n$ & 10 000  &    &   &   &     \\ \hline
$r_n$ & 500 000 &    &   &   &     \\ \hline
\end{tabular}
\end{center}

\subquestion{}
D'après le tableau précédent, quel prix $p_n$ permet à Optitan de réaliser la meilleure recette?

\partie{Modèle continu}

On considère la fonction $R$ de la variable réelle $x$ défniie sur l'intervalle $\left[0 ; 4\right]$ par:

$$R(x) = (5+2x)e^{-0.2x}$$

\question{}
Étude de la fonction $R$.

\subquestion{}
Calculer $R'(x)$, pour tout réel $x$ de l'intervalle $[0;4]$. 

\subquestion{}
Étudier le signe de $R'(x)$ sur l'intervalle $[0;4]$. 

En déduire le tableau de variation de $R$, dans lequel figureront les valeurs exactes de $R(0)$, $R(4)$ et $R(x_0)$, où $x_0$ est la valeur de $x$ pour laquelle la fonction $R$ admet un maximum. 

\subquestion{}
Donner les valeurs approchées arrondies à $10^{-2}$ près de $R(2.5)$ et de $R(4)$.

\question{}
On admet que, lorsque le prix de vente unitaire du modèle de lunettes solaires considéré au début de l'exercice est $(50+20x)$ €, avec $0 \leqslant x \leqslant 4$, la recette correspondante est $R(x)$, en centaines de milliers d'euros.

En utilisant les résultats de la question 1:

\subquestion{}
Déterminer le prix unitaire en euros pour lequel la recette est maximale. %La réponse sera arrondie à l'euro le plus proche.

\subquestion{}
Deux prix permettent une recette de 600 000 €. Expliquer pourquoi l'un est favorable à l'acheteur et et l'autre au vendeur.


\exo{Statistiques à deux variables}
La société INFOLOG a mis au point un nouveau logiciel de gestion destiné aux PME. Cette société a mené une enquête dans une région auprès de 300 entreprises équipées d'ordinateurs aptes à recevoir ce logiciel, ceci afin de déterminer à quel prix chacune de ces entreprises accepterait d'acquérir un exemplaire de ce nouveau logiciel. Elle a obtenu les résultats suivants:

\begin{center}
\begin{tabular}{|c|c|c|c|c|c|}
\hline 
$x$: prix proposé (\emph{centaines} d'euros) & 30 & 25 & 20 & 15 & 10 \\
\hline 
$y$: nb. d'entreprises disposées à acheter & 90 & 120 & 170 & 200 & 260\\
\hline 
\end{tabular} 
\end{center}

\question{}
Représenter graphiquement sur votre copie le nuage de points de la série $(x_i ; y_i$) dans un repère orthogonal (unités graphiques : 1 cm pour 200 euros en abscisses et 5 cm pour 100 entreprises en ordonnées). Placer le point moyen $G$ après avoir déterminé ses coordonnées. %créer graphe

\question{}
Ajustement affine.

\emph{Aucun détail de calculs n'est demandé dans la question 2.}

\subquestion{}
Donner, à $10^{-3}$ près, le coefficient de corrélation linéaire $r$ de $y$ en $x$.

\subquestion{}
Déterminer, par la méthode des moindres carrés, l'équation de la droite $\mathcal{D}$ d'ajustement affine de $y$ en $x$ sous la forme $y = ax + b$.


Tracer $\mathcal{D}$ sur le graphique.

\question{}
En utilisant l'ajustement précédent, préciser pour quel prix de vente la société INFOLOG peut espérer que les 300 entreprises acceptent d'acquérir le logiciel.

\appendix

\section*{Formulaire}

\subsection*{Probabilités}

$P(A \cup B) = P(A) + P(B) - P(A \cap B)$

$P_B(A) = \dfrac{P(A \cap B)}{P(B)}$

\littlestar{Loi de Poisson de paramètre $\lambda$} $P(X=k) = \dfrac{\lambda^{k} \times e^{-\lambda}}{k!}$

$E(X) = \lambda$

$\sigma(X) = \sqrt{\lambda}$

\littlestar{Loi binomiale de paramètres $n$ et $p$} $P(X=k) = C_{n}^{k} \cdot p^{k} \cdot (1-p)^{n-k}$ avec $C_{n}^{k} = \binom{n}{k} = \dfrac{n!}{k!(n-k)!}$ et $n! = 1 \times 2 \times 3 \times 4 \times \ldots \times n$.

$E(X) = np$ 

$\sigma(X) = \sqrt{np(1-p)}$

\subsection*{Dérivation}

\begin{multicols}{2}
\subsubsection*{Dérivées usuelles}

$\left(e^x\right)' = e^x$

$\left(\ln x\right)' = \dfrac{1}{x}$

$\left(\dfrac{1}{x}\right)' = -\dfrac{1}{x^2}$

$\left( x^\alpha \right)' = \alpha x^{\alpha - 1}$

\subsubsection*{Opérations sur les dérivées}

$\left(u+v\right)' = u'+v'$

$\left(ku\right)' = ku'$

$\left(uv\right)' = u'v + uv'$

$\left(\dfrac{u}{v}\right)' = \dfrac{u'v-uv'}{v^{2}}$

$\left(u^{\alpha}\right)' = \alpha u' u^{\alpha-1}$

$(\ln u)' = \dfrac{u'}{u}$

$\left(e^{u}\right)' = u' e^{u}$
\end{multicols}

\trait

\begin{center}
Fin.
\end{center}

\end{document}
