\documentclass[a4paper,12pt]{scrartcl}
\usepackage[utf8x]{inputenc}
\usepackage[T1]{fontenc} % avec T1 comme option  d'encodage c'est ben mieux, surtout pour taper du français.
%\usepackage{lmodern,textcomp} % fortement conseillé pour les pdf. On peut mettre autre chose : kpfonts, fourier,...
\usepackage[french]{babel} %Sans ça les guillemets, amarchpo
\usepackage{amsmath}
\usepackage{multicol}
\usepackage{amssymb}
\usepackage{tkz-tab}
\usepackage{exercice_sheet}

%\trait
%\section*{}
%\exo{}
%\question{}
%\subquestion{}

\date{}


% Title Page
\title{Devoir maison 1, OL-2}

\author{\textsc{Mathématiques}}

%mettre la correction ou pas
\renewcommand{\hascorrection}{1}

\begin{document}

\maketitle

\exo{}

\question{}
24 ; 18 ; 12 ; 6 sont-ils les premiers termes d'une suite arithmétique? Si oui, en donner le premier terme et la raison.

\correction{
On calcule, pour les termes dont on dispose, $u_{n+1} - u_n$.

$18 - 24 = -6$

$12 - 18 -6$

$6 - 12 -6$

Ainsi, les termes sont bien les premiers termes d'une suite arithmétique de raison -6 et de premier terme 24.
}


\question{}
$2$ ; $-3$ ; $\frac{9}{2}$ ; $-\frac{27}{4}$ sont-ils les premiers termes d'une suite géométrique? Si oui, en donner le premier terme et la raison.


\correction{
On calcule, pour les termes dont on dispose, $\frac{u_{n+1}}{u_n}$.


$-\dfrac{3}{2} = \dfrac{\frac{9}{2}}{-3} = \dfrac{-\frac{27}{4}}{\frac{9}{2}}$

Il s'agit donc bien des premiers termes d'une suite géométrique de premier terme 2 et de raison $\frac{-3}{2}$.
}

\question{}
Soit la suite géométrique $(u_n)$ de premier terme $u_0 = 7$ et de raison $q = -2$. 

\correction{
Le terme général d'une suite $(u_n)$ géométrique de premier terme $(u_0)$ et de raison $q$ est $u_n = u_0 \times q^n$.

Ici, $u_n = 7 \times (-2)^n$.
}

\subquestion{}
Calculer $u_1$, $u_4$ et $u_7$.

\correction{
\begin{itemize}
 \item $u_1 = 7 \times (-2) = -14$
 \item $u_4 = 7 \times (-2)^4 = 112$
 \item $u_7 = 7 \times (-2)^7 = -896$
\end{itemize}
}

\subquestion{}
Calculer $S_{4,12} = u_4 + u_5 + \ldots + u_{12}$.

\correction{
On rappelle que pour calculer la somme de termes consécutifs d'une suite géométrique du rang $p$ au rang $n$, la formule est:

\begin{equation}
 S_{p,n} = u_p \frac{1 - q^{n-p+1}}{1-q}
\end{equation}

Ici, $p = 4$, $n = 12$ d'où $n-p+1 = 9$, $q = -2$ et $u_4 = 112$.

On a donc:

\begin{equation}
 S_{4,12} = 112 \times \frac{1 - (-2)^{9}}{1-(-2)} = 19152
\end{equation}
}


\exo{}

On considère la suite $(u_n)$ définie pour tout entier naturel $n$ par:

$$\left\lbrace
\begin{array}{ll}
   u_0 = 13 \\
   u_{n+1} = 3 u_n - 14 \\
\end{array}
\right.$$

% $u_0 = 13$ et pour tout entier naturel $n$, $u_{n+1} = 3 u_n - 14$.

\question{}
Calculer $u_1$, $u_2$ et $u_3$.

\correction{
\begin{itemize}
 \item $u_0 = 13$
 \item $u_1 = 3u_0 - 14 = 25$
 \item $u_2 = 3u_1 - 14 = 61$
 \item $u_3 = 3u_2 - 14 = 169$
\end{itemize}
}

\question{}
On considère la suite $(v_n)$ définie pour tout entier naturel $n$, par $v_n = u_n - 7$.


\subquestion{}
Montrer que la suite $(v_n)$ est une suite géométrique dont on donnera le premier terme et la raison.

\correction{
Calculons $\frac{v_{n+1}}{v_n}$.

$\forall n \in \mathbb{N}, v_n + u_n - 7$, d'où $v_{n+1} = u_{n+1} - 7$

Donc $\frac{v_{n+1}}{v_n} = \frac{u_{n+1} - 7}{v_n}$

Or d'après l'énoncé, $u_{n+1} = 3 u_n - 14$

D'où $\frac{v_{n+1}}{v_n} = \frac{3 u_n - 14 - 7}{v_n} = \frac{3 u_n - 21}{v_n}$

On déduit de l'énoncé que $u_n = v_n + 7$. 

Donc $\frac{v_{n+1}}{v_n} = \frac{3 (v_n + 7) - 21}{v_n}$

Finalement:

\Answer{$\dfrac{v_{n+1}}{v_n} = 3$}

$(v_n)$ est donc géométrique de raison 3 et de premier terme $v_0 = u_0 - 7 = 6$.
}

\subquestion{}
En déduire l'expression de $v_n$ en fonction de $n$.

\correction{
On en déduit son terme général: $v_n = 6 \times 3^n$
}

\subquestion{}
En déduire que $u_n = 6 \times 3^n + 7$.

\correction{
$u_n = v_n + 7 = 6 \times 3^n + 7$
}

\subquestion{}
Déterminer $\lim u_n$ et $\lim v_n$.

\correction{
$(v_n)$ est géométrique de raison $q = 3 > 1$ et de premier terme $v_0 = 6 > 0$. 

\begin{equation*}
\lim_{n \to +\infty} v_n = + \infty 
\end{equation*}

De plus, $v_n > u_n$ quel que soit $n \in \mathbb{N}$, d'où: 

\begin{equation*}
 \lim_{n \to +\infty} u_n = + \infty 
\end{equation*}
}

\question{}
Soient les suites $U_n = \frac{1}{u_n}$ et $V_n = \frac{1}{v_n}$. Déterminer $\lim U_n$ et $\lim V_n$.

\correction{
 D'après les résultats précédents, $\lim U_n$ est de la forme $\frac{1}{+\infty}$ d'où $\lim U_n = 0^+$.

 D'après les résultats précédents, $\lim V_n$ est de la forme $\frac{1}{+\infty}$ d'où $\lim V_n = 0^+$.
}



\trait

\begin{center}
Fin.
\end{center}

\end{document}
