\documentclass[a4paper,12pt]{scrartcl}
\usepackage[utf8x]{inputenc}
\usepackage[T1]{fontenc} % avec T1 comme option  d'encodage c'est ben mieux, surtout pour taper du français.
%\usepackage{lmodern,textcomp} % fortement conseillé pour les pdf. On peut mettre autre chose : kpfonts, fourier,...
\usepackage[french]{babel} %Sans ça les guillemets, amarchpo
\usepackage{amsmath}
\usepackage{multicol}
\usepackage{amssymb}
\usepackage{tkz-tab}
\usepackage{exercice_sheet}

%\trait
%\section*{}
%\exo{}
%\question{}
%\subquestion{}

\date{}


% Title Page
\title{Devoir en classe 1, CG-2}

\author{Mathématiques}

\begin{document}

\maketitle

Sujet à rendre avec la copie.

\exo{Résoudre les équations suivantes:}

\question{}
$2x^2 +4x - 2 = 0$

$\Delta = b^2 - 4ac = 32 > 0$. L'équation a donc 2 solutions.

$\sqrt{\Delta} = 4 \sqrt{2}$

$x_1 = \dfrac{-4-4\sqrt{2}}{4} = 1-\sqrt{2}}$

$x_2 = \dfrac{-4+4\sqrt{2}}{4} = 1+\sqrt{2}}$

L'ensemble des solutions $S$ s'écrit donc $S = {1-\sqrt{2}};1+\sqrt{2}}}$

\question{}
$e^{2x} - 2e^x + 1 = 0$

$\Leftrightarrow (e^x - 1)^2 = 0$

$\Leftrightarrow e^x - 1 = 0$

$\Leftrightarrow e^x = 1$

$\Leftrightarrow x = 0$

\exo{Statistiques à une variable}

Une entreprise fabrique des conserves alimentaires dont l'étiquette annonce une masse de 250 grammes.
Les masses obtenues pour un échantillon de 500 conserves prises au hasard sont données dans le tableau suivant: 

\begin{table}[h]
\centering
\caption{Masses des 500 conserves.}
\label{table_conserves}
\begin{tabular}{|l|l|l|l|l|l|}
\hline
\textbf{Masse (en g) $x_{i}$}        & {[}235 ; 240{[} & {[}240 ; 245{[} & {[}245 ; 250{[} & {[}250 ; 255{[} & {[}255 ; 260{[} \\ \hline
\textbf{Nb. de conserves} & 33              & 67              & 217             & 132             & 51              \\ \hline
\end{tabular}
\end{table}

\question{}
À l'aide de la calculatrice, calculer, en utilisant les milieux des classes, la masse moyenne $\overline{x}$ ainsi que l'écart type $\sigma_x$ des conserves de cet échantillon.

On fournira les valeurs en grammes arrondies au dixième.

On trouve $\overline{x} = 248.5g$.

$\sigma_x = 5.07$

\question{}
Calculer le pourcentage des conserves alimentaires ayant une masse comprise entre 240 et 255 grammes. 

Il y a un total de 416 conserves dans cet intervalle. $\dfrac{416}{500} = 83.2\%$

\exo{Suites}

On place une somme d'argent notée $S_0$ au taux annuel de 5,5 \%, ce placement étant à intérêts composés.
Pour tout entier naturel $n$, $S_n$ désigne le capital disponible au bout de $n$ années.

\question{}
Justifier que la suite $(S_n)$ est géométrique. Préciser la raison $q$ de cette suite.

La somme sur le compte est chque année multipliée par le même nombre. Il s'agit donc d'une suite géométrique.

$q = 1.055$

\question{}
Exprimer $S_n$ en fonction de $S_0$ et de $n$.

$S_n = S_0 \times q^n$

\exo{}

Pour les besoins d'une usine qui fabrique des puces, l'entreprise TERRARE extrait du minerai rare. Sa production annuelle $X$ (en
tonnes) n'excède pas 2 tonnes et le coût total annuel de la production est noté $Y$ en milliers d'euros (1 k€ = 1000 €).
Les résultats des premières années d'exploitation sont consignés dans le tableau suivant:

\begin{table}[h]
\centering
\caption{}
\label{tableau_2}
\begin{tabular}{|l|l|l|l|l|l|}
\hline
année          & 2006  & 2007  & 2008  & 2009  & 2010  \\ \hline
$x_i$ (tonnes) & 0.52  & 0.77  & 1.01  & 1.36  & 1.81  \\ \hline
$y_i$ (k€)     & 186.7 & 230.9 & 283.1 & 381.3 & 558.9 \\ \hline
$z_i$&5.230&5.442&5.646&5.944&6.326 \\ \hline
\end{tabular}
\end{table}

\question{}
Le plan est muni d'un repère orthogonal.
Unités graphiques : 1 cm pour 0,1 unité sur l'axe des abscisses et 2 cm pour 100 unités sur l'axe des ordonnées.
Construire le nuage de points associé à cette série statistique sur votre copie. 

\question{}
La nature de l'activité et le graphique laissent penser qu'un ajustement exponentiel est approprié. On pose $z = \ln y$.

\subquestion{}
Compléter le tableau ci-dessus.
Arrondir à $10^{-3}$ les valeurs de $z_i$.

\subquestion{}
Déterminer le coefficient de corrélation linéaire $r$ entre $x$ et $z$. Arrondir à $10^{-3}$.

\subquestion{}\label{question}
À l'aide de la calculatrice, déterminer par la méthode des moindres carrés, une équation de la droite d'ajustement de $z$
en $x$. Les coefficients seront arrondis à $10^{-2}$.

\question{}
Estimations.

\subquestion{}
Déduire de la question précédente une expression de $y$ en fonction de $x$, de la forme $y = B \cdot e^{ax}$ où $a$ est un réel et où $B$ sera arrondi à l'entier le plus proche.

\subquestion{}
En déduire une estimation du coût de production pour 2 tonnes.


\trait

\begin{center}
Fin.
\end{center}

\end{document}
