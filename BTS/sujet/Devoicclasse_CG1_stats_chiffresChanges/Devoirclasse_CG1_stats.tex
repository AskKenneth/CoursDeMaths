\documentclass[a4paper,12pt]{scrartcl}
\usepackage[utf8x]{inputenc}
\usepackage[T1]{fontenc} % avec T1 comme option  d'encodage c'est ben mieux, surtout pour taper du français.
%\usepackage{lmodern,textcomp} % fortement conseillé pour les pdf. On peut mettre autre chose : kpfonts, fourier,...
\usepackage[french]{babel} %Sans ça les guillemets, amarchpo
\usepackage{amsmath}
\usepackage{multicol}
\usepackage{amssymb}
\usepackage{tkz-tab}
\usepackage{exercice_sheet}

%\trait
%\section*{}
%\exo{}
%\question{}
%\subquestion{}

\date{}


% Title Page
\title{Devoir en classe 1, CG-1}

\author{Mathématiques}

\begin{document}

\maketitle

\exo{Résoudre les équations suivantes:}

\question{}
$x^2-x-2=0$

\question{}
$x^2+2x+2=0$

\exo{Factoriser les expressions suivantes:}

\question{}
$f(x) = x^2 - 4x$

\question{}
$h(x) = x^2-x-2$

\exo{Statistiques à une variable:}

Nous avons mesuré la taille en centimètres de 11 personnes et réuni les données mesurées dans un tableau:

\begin{center}
\begin{tabular}{|l|l|l|l|l|l|l|l|l|l|l|l|}
\hline
$x_i$ & 164 & 175 & 178 & 181 & 183 & 186 & 189 & 189 & 194 & 197 & 200 \\ \hline
\end{tabular}
\end{center}

Indiquer:

\question{}
L'étendue

\question{}
La médiane $Q_2$

\question{}
Le premier quartile $Q_1$, le troisième quartile $Q_3$

\question{}
La moyenne $\overline{x}$

\question{}
La variance $V_x$ et l'écart-type $\sigma_x$. On rappelle que la variance peut se calculer comme suit: $V_x = \overline{x^2} - \overline{x}^2$.

\question{}
Quel pourcentage de cette population se situe dans l'intervalle $[\overline{x} - \sigma_x ; \overline{x} + \sigma_x]$?

\exo{Statistiques à deux variables}

On veut étudier s'il existe une dépendance entre le nombre de fruits sur un rameau d'un arbre fruitier et la masse de chaque fruit. 

On a relevé sur différents rameaux le nombre de fruits et le poids de chaque fruit. 

Les données relevées sont visibles dans le tableau ci-dessous:

\begin{center}
\begin{tabular}{|l|l|l|l|l|l|l|}
\hline
Nombre de fruits $x_i$       & 1   & 2   & 3   & 4   & 5  & 6  \\ \hline
Poids en grammes $y_i$ & 153 & 148 & 107 & 108 & 78 & 47 \\ \hline\end{tabular}
\end{center}

\question{}
En prenant 1cm pour une unité en abscisse et 1cm pour 20 grammes en ordonnée, tracer le nuage de points sur votre copie.

\question{}
Calculer les coordonnées du point moyen $G$ et le tracer sur le graphe.

\question{}
Méthode de Mayer

\subquestion{}
Calculer les coordonnées du point $G_1$, point moyen des trois premiers points ainsi que celles de $G_2$, point moyen des trois derniers points du nuage. 

\subquestion{}
Tracer $G_1$ et $G_2$ sur le graphe.

\subquestion{}
Indiquer l'équation de la droite $(G_1 G_2)$ et la tracer sur le graphe. 

\subquestion{}
En utilisant l'équation de cette droite, quelle masse peut-on attendre pour un fruit issu d'un rameau contenant 8 fruits?

\trait

\begin{center}
Fin.
\end{center}

\end{document}
