\documentclass[a4paper,12pt]{scrartcl}
\usepackage[utf8x]{inputenc}
\usepackage[T1]{fontenc} % avec T1 comme option  d'encodage c'est ben mieux, surtout pour taper du français.
%\usepackage{lmodern,textcomp} % fortement conseillé pour les pdf. On peut mettre autre chose : kpfonts, fourier,...
\usepackage[french]{babel} %Sans ça les guillemets, amarchpo
\usepackage{amsmath}
\usepackage{multicol}
\usepackage{amssymb}
\usepackage{tkz-tab}
\usepackage{exercice_sheet}

%\trait
%\section*{}
%\exo{}
%\question{}
%\subquestion{}

\date{}


% Title Page
\title{Devoir en classe, CG-1}

\author{\rotatebox{10}{\textsc{Mathématiques}} \\ 60 minutes}

\begin{document}

\maketitle

\section*{Exponentielle et logarithme}

\exo{Résoudre les équations suivantes:}

\question{}
$x^{2} + 1 = 0$

\question{}
$x^{2} - 1 = 0$

\question{}
$5e^{4x} +2 = 22$

\question{}
$e^{\frac{x}{3}} = 4$

\question{}
$\left( \dfrac{1}{4} \right)^{x} = 16$

\exo{Résoudre les inéquations suivantes:}

\question{}
$e^{2x} > 5$

\question{}
$\ln \left(x^{5}\right) \geqslant 10$

\question{}
$2e^{2x+3} < 2$

\question{}
$\ln(6x+2) \leqslant 5$

\exo{}

On pose $y = \ln(z+1)$  et $y = -x+3$.

\question{}
Montrer que $z$ peut s'écrire sous la forme $z = \alpha e^{\beta x} + \gamma$ où $\alpha$, $\beta$ et $\gamma$ sont des  constantes ($\alpha$ sera arrondi à l'entier le plus proche).

\question{}
Calculer $z$ pour $x = 5$. Donner la valeur exacte puis arrondir à $10^{-3}$ près.

\section*{Dérivation}

\exo{Calculer les dérivées des fonctions suivantes. Les simplifier lorsque c'est possible:}

\question{}
$f_1(x) = 5x^2 + 2$

\question{}
$f_2(x) = e^{5x^2 + 2}$

\question{}
$f_3(x) = \ln\left( \dfrac{1}{x^2 + 1} \right)$

\question{}
$f_4(x) = x \sqrt{x}$

\question{}
$f_5(x) = e^{x} \ln x$

\question{}
$f_6(x) = \sqrt{x^{3} + 1}$

\question{}
$f_7(x) = \dfrac{3x + 1}{x^{2} + x + 1}$

\question{}
$f_8(x) = 10^{12}$

\exo{}
Soit la fonction $f$ telle que $f(x) = e^{5x+3}$. 

\question{Montrer que $f'(x) = 5e^{5x+3}$.}

\question{Donner le signe de $f'(x)$ sur $\mathbb{R}$.}

\question{En déduire les variations de $f$ sur $\mathbb{R}$.}


\section*{Formulaire}

\section*{Exponentielle et logarithme}

$a$ et $b$ sont des nombres réels.

\begin{center}

\begin{tabular}{|c|c|}
\hline 
\textbf{fonction exponentielle} & \textbf{fonction logarithme népérien} \\ 
\hline 
$e^{a+b} = e^{a} \times e^{b}$ & $\ln(a \times b) = \ln a + \ln b$ \\ 
\hline 
$e^{a-b} = \dfrac{e^{a}}{e^{b}}$ & $\ln \left( \dfrac{a}{b} \right) = \ln a - \ln b$ \\ 
\hline 
$e^{-a} = \dfrac{1}{e^{a}}$ & $\ln \left( \dfrac{1}{a} \right) = - \ln a$ \\ 
\hline 
$(e^{a})^{b} = e^{a \times b}$ & $\ln(a^n) = n \ln a$, $n \in \mathbb{R}$, $a > 0$ \\ 
\hline 
$e^{1} = e$ & $\ln e = 1$ \\ 
\hline 
$e^{0} = 1$ & $\ln 1 = 0$ \\ 
\hline 
\end{tabular} 

\end{center}

\subsection*{Quelques propriétés}

\begin{itemize}

\item les 2 fonctions sont réciproques: $\ln \left( e^{a} \right) = a$ pour $a \in \mathbb{R}$; $e^{\ln a} = a$ pour $a > 0$ (\og $a$ strictement positif \fg{}).

\item ensemble de définition: 

$e^{a}$ peut être calculé quel que soit $a \in \mathbb{R}$: on dit que la fonction exponentielle est définie sur $\mathbb{R}$.

$\ln a$ peut être calculé uniquement pour $a > 0$. On dit donc que la fonction logarithme népérien est définie sur $]0;+\infty[$.


\item elles sont strictement croissantes sur leurs ensembles de définition, c'est-à-dire que:

$a > b > 0 \Leftrightarrow \ln a > \ln b$

et pour la fonction exponentielle:

$a > b \Leftrightarrow e^{a} > e^{b}$

\end{itemize}


\section*{Dérivation}

\subsection*{Dérivées de fonctions usuelles}

\begin{center}

\begin{tabular}{|c|c|}
\hline 
\textbf{fonction $f$} & \textbf{fonction dérivée $f'$} \\ 
\hline 
$k$, $k \in \mathbb{R}$ & $0$ \\ 
\hline 
$x$ & $1$ \\ 
\hline 
$x^2$ & $2x$ \\ 
\hline 
$x^{\alpha}$, $\alpha \in \mathbb{Z}$ & $\alpha x^{\alpha - 1}$ \\ 
\hline 
$e^x$ & $e^x$ \\ 
\hline 
$\ln x$ & $\dfrac{1}{x}$ \\ 
\hline 
$\sqrt{x}$ & $\dfrac{1}{2\sqrt{x}}$ \\ 
\hline 
\end{tabular} 
\end{center}

\subsection*{Dérivées de produits, quotients, etc.}

\begin{center}
\begin{tabular}{|c|c|}
\hline 
\textbf{fonction $f$} & \textbf{fonction dérivée $f'$} \\ 
\hline 
$u+v$ & $u'+v'$ \\ 
\hline 
$ku$ & $ku'$ \\ 
\hline 
$u \times v$ & $u'v+uv'$ \\ 
\hline 
$\dfrac{u}{v}$ & $\dfrac{u'v-uv'}{v^2}$ \\ 
\hline 
$\dfrac{1}{u}$ & $-\dfrac{u'}{u^2}$ \\ 
\hline 
\end{tabular} 
\end{center}

% \subsection*{Dérivées de fonctions composées}
% 
% 
% \begin{center}
% \begin{tabular}{|c|c|}
% \hline 
% \textbf{fonction $f$} & \textbf{fonction dérivée $f'$} \\ 
% \hline 
% $v \circ u$ ou $v(u(x))$ & $u' \times v' \circ u $ ou $u'(x) \times v'(u(x))$ \\ 
% \hline 
% $e^u$ & $u'e^u$ \\ 
% \hline 
% $\ln u$ & $\dfrac{u'}{u}$ \\ 
% \hline 
% $u^\alpha$ & $\alpha u' u^{\alpha - 1}$ \\ 
% \hline 
% \end{tabular}
% \end{center}


\trait

\begin{center}
Fin.
\end{center}

\end{document}
