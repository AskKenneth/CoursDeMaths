\documentclass[a4paper,12pt]{scrartcl}
\usepackage[utf8x]{inputenc}
\usepackage[T1]{fontenc} % avec T1 comme option  d'encodage c'est ben mieux, surtout pour taper du français.
%\usepackage{lmodern,textcomp} % fortement conseillé pour les pdf. On peut mettre autre chose : kpfonts, fourier,...
\usepackage[french]{babel} %Sans ça les guillemets, amarchpo
\usepackage{amsmath}
\usepackage{multicol}
\usepackage{amssymb}
\usepackage{tkz-tab}
\usepackage{exercice_sheet}%\trait

%\section*{}
%\exo{}
%\question{}
%\subquestion{}


% Title Page
\title{Exercices de révision}
%\subtitle{Factorisation, développement, racines carrées et bien d'autres...}
\author{}

\begin{document}

\maketitle

%\begin{multicols}{2}
\section*{Fractions}

\exo{Simplifier au maximum ces écritures fractionnaires}

\begin{multicols}{2}

\question{$A = \dfrac{12}{4}$}

\question{$B = \dfrac{4}{12}$}

\question{$C = \dfrac{5}{6} + \dfrac{7}{6}$}

\question{$D = \dfrac{1}{3} -  \dfrac{5}{12}$}

\question{$E = \dfrac{10}{7} -  \dfrac{8}{5}$}

\question{$F = \dfrac{9}{4} + \dfrac{7}{9}$}

\question{$G = \dfrac{2}{7} \times \dfrac{4}{7}$}

\question{$H = \dfrac{1}{4} \times \dfrac{1}{3}$}

\question{$I = \dfrac{\frac{2}{7}}{\frac{3}{4}}$}

\question{$J = \dfrac{\frac{5}{9}}{\frac{5}{8}}$}

\end{multicols}

\section*{Factorisations et développements}

\exo{Factoriser}

\begin{multicols}{2}
\question{$A = 4x+5x$}

\question{$B = 12x-4x+8x$}

\question{$C = 16x^2 + 16x + 4$}

\question{$D = 9h^2 - 4h^2$}

\question{$E = (4+x)^2 - 81$}

\question{$F = z^2 - 4z + 2$}

\question{$G = x^2 + 8x + 7$}

\question{$H = 4X^2 - 16X + 8$}
\end{multicols}

\clearpage
\exo{Développer et réduire}
\begin{multicols}{2}
\question{$A = 4(x+8)$}

\question{$B = (x+9)(x-9)$}

\question{$C = (x+9)(x-3)$}

\question{$D = (x+7)^2$}

\question{$E = (4+x)^2 - 81$}

\question{$F = (y-2)^2$}

\question{$G = (3z-2)^2$}

\question{$H = (2t+4)^2$}
\end{multicols}

\section*{Racines carrées}

\exo{Simplifier\footnote{Les écrire avec les nombres les plus petits possibles sous la racine... s'il reste une racine ce qui n'est pas forcément le cas.} les écritures suivantes:}

\begin{multicols}{2}

\question{$\sqrt{4}$}

\question{$\sqrt{8}$}

\question{$\sqrt{343}$}

\question{$\sqrt{208}$}

\question{$\sqrt{50000}$}

\question{$\sqrt{148}$}

\question{$\sqrt{1050}$}

\question{$\sqrt{200}$}
\end{multicols}

\section*{Polynômes du 2\textsuperscript{nd} degré}

\exo{Trouver les racines, factoriser lorsque c'est possible, et établir le tableau de signe des polynômes suivants}

\begin{multicols}{2}

\question{$x^2 + 5x + 1$}

\question{$x^2 + x + 5$}

\question{$7x^2 + 2x - 9$}

\question{$2x^2 - 9x + 4$}

\question{$2x^2 + 8x + 9$}

\question{$2x^2 + 3x + 2$}

\question{$x^2 + 12x + 11$}

\question{$8x^2 - 10x + 3$}

\question{$2x^2 + 7x + 6$}

\question{$-x^2 + 4x - 4$}
\end{multicols}

\section*{Équations, inéquations}

\exo{Résoudre ces équations:}

\begin{multicols}{2}

\question{$2x + 4 = -x + 2$}

\question{$8x^2 + 9x + 12 = -9x^2 - 5x + 16$}

\question{$\dfrac{x}{4} + 2 = x + 4$}

\question{$\dfrac{2x + 4}{6}  = \dfrac{7x + 1}{12}$}

\question{$x^2 + 8x = -12$}

\question{$2x^2 = -1$}

\question{$\dfrac{1-x}{x-1} = 3$}

\question{$(x-2)(x^2 + 4x + 3) = 0$}
\end{multicols}

\exo{Et maintenant, ces inéquations:}
\begin{multicols}{2}

\question{$x + 2 \geqslant -x-1$}

\question{$4x + 1 \geqslant 2x - 3$}

\question{$x^2 \leqslant 1$}

\question{$2x \geqslant 12 + x$}

\question{$x+1 \leqslant x$}

\question{$4(x+1)(x-2) \leqslant 0$}
\end{multicols}

\section*{Systèmes de 2 équations à 2 inconnues}

\exo{Résoudre les systèmes suivants:}

\begin{multicols}{2}
\question{$
\begin{cases}
3x+2y &= 42\\ 
2x+y &= 6
\end{cases}$}

\question{$
\begin{cases}
a+b &= 3\\ 
4a-b &= 2
\end{cases}$}

\question{$
\begin{cases}
4v+6w &= -10\\ 
2v+w &= -1
\end{cases}$}

\question{$
\begin{cases}
3\alpha+8\beta &= 2\\ 
2\alpha+7\beta &= 3
\end{cases}$}

\question{$
\begin{cases}
x+2y &= 2\\ 
\frac{x}{2}+y &= 3
\end{cases}$}

\question{$
\begin{cases}
3x+y &= 2\\ 
-6x-2y &= -4
\end{cases}$}

\end{multicols}

\exo{Véhicules}

Un parking contient 35 véhicules. Des tricycles et des voitures (4 roues donc). Le nombre total de roues dans le parking est de 132. 

Combien y a-t-il de véhicules de chaque type?

\exo{Droites}

\question{Quelle est l'équation de la droite $(D_1)$ passant par les points $M_1(-2;-20)$ et $M_2(4;22)$ ?}

\question{Et l'équation de $(D_2)$, qui elle passe par les points $N_1(2;2)$ et $N_2(-4;-1)$ ?}

\question{Quelles sont les coordonnées du point $I = (D_1) \cap (D_2)$ (qui est l'intersection des 2 droites)?}

%\end{multicols}
\end{document}
